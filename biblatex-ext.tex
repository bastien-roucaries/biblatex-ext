\documentclass[DIV=9]{scrartcl}
\usepackage{ifxetex,ifluatex}
\newif\iffontspec
\ifxetex
  \fontspectrue
\else\ifluatex
  \fontspectrue
\else
  \fontspecfalse
\fi\fi
\iffontspec
  \usepackage{fontspec}
\else
  \usepackage[T1]{fontenc}
  \usepackage[utf8]{inputenc}
\fi
\usepackage[french,ngerman,british]{babel}
\usepackage{csquotes}
\usepackage[useregional]{datetime2}
\usepackage{lmodern}
\usepackage[mono=false]{libertine}
\usepackage[libertine]{newtxmath}
\iffontspec
  \setmonofont[Scale=0.78]{Bitstream Vera Sans Mono}
\else
  \usepackage[scaled=0.78]{beramono}
\fi
\usepackage{microtype}
\usepackage{hyphenat}
\usepackage{bm}
\usepackage{booktabs}
\usepackage{multicol}
\usepackage[svgnames]{xcolor}
\usepackage[listings, breakable, skins]{tcolorbox}%
\usetikzlibrary{arrows.meta}
\usepackage[style=ext-authoryear-icomp, backend=biber]{biblatex}
\addbibresource{biblatex-examples.bib}
\addbibresource{biblatex-ext-examples.bib}

\makeatletter
\defbibcheck{examplebib}{%
  \xifinlist{\thefield{entrykey}}{\extblxdoc@examplebib@list}
    {}
    {\skipentry}}

\newcommand*{\exampleprintbib}[1]{%
  \nocite{#1}%
  \let\extblxdoc@examplebib@list\empty
  \def\do##1{\listeadd\extblxdoc@examplebib@list{\detokenize{##1}}}%
  \docsvlist{#1}%
  \printbibliography[check=examplebib, heading=none]}

\iffontspec
\def\textvisiblespace{%
  \raisebox{-2.2pt}{%
    \mbox{\kern.04em\vrule \@height.5ex \@width.12ex}%
    \vbox{\hrule \@width.2em \@height.12ex}%
    \hbox{\vrule \@height.5ex \@width.12ex}%
    \kern.04em}}
\fi
\makeatother


\usepackage{ltxdockit}
\usepackage{btxdockit}
\usepackage{cleveref}
\hypersetup{%
  colorlinks=true,
  allcolors=spot,
  bookmarksopen=false,
  bookmarksnumbered=false,
  plainpages=false}

\definecolor{highlight1}{RGB}{240, 0, 0}
\definecolor{highlight2}{RGB}{0,153, 153}
\definecolor{spot}{rgb}{0,0.2,0.6}

\lstdefinestyle{extblxstylegeneral}{%
  aboveskip    = {0\p@ \@plus 6\p@},
  belowskip    = {0\p@ \@plus 6\p@},
  tabsize      = 2,
  breaklines   = true,
  breakatwhitespace = true,
  keepspaces   = true,
  escapeinside = {(*@}{@*)},
  moredelim    = {[is][\ttfamily\bfseries\color{highlight1}]{|}{|}},
  moredelim    = {[is][\ttfamily\bfseries\color{highlight1}]{|1}{1|}},
  moredelim    = {[is][\ttfamily\bfseries\color{highlight2}]{|2}{2|}},
}

\lstdefinelanguage{extBibTeX}{%
  morekeywords=[1]{%
    @article,@artwork,@audio,@bibnote,@book,@bookinbook,@booklet,%
    @collection,@commentary,@customa,@customb,@customc,@customd,%
    @custome,@customf,@inbook,@incollection,@inproceedings,%
    @inreference,@image,@jurisdiction,@legal,@legislation,@letter,%
    @manual,@misc,@movie,@music,@mvcollection,@mvreference,%
    @mvproceedings,@mvbook,@online,@patent,@performance,@periodical,%
    @proceedings,@reference,@report,@review,@set,@software,@standard,%
    @suppbook,@suppcollection,@suppperiodical,@thesis,@unpublished,@video%
   },
   morekeywords=[2]{author,title,date},
   keywordstyle=[1]{\bfseries\spotcolor},
   keywordstyle=[2]{\spotcolor},
   sensitive=false,
}

\lstdefinestyle{extblxstylelatex}{%
  language  = {[LaTeX]TeX},
  style     = {extblxstylegeneral},
  moretexcs = {setlength,bibhang,addcomma,adddot,addperiod,addcolon,addspace,
    addnbspace,
    mkbibbold,mkbibemph,mkbibbrackets,mkbibparens,
    usebibmacro,newbibmacro,renewbibmacro,setunit,newunit,printfield,printlist,
    bibopenparen,bibcloseparen,bibopenbracket,bibclosebracket,
    iflistundef,iffieldundef,ifnameundef,
    ExecuteBibliographyOptions,
    DeclareFieldFormat,DeclareDelimFormat,DeclareDelimcontextAlias,
    DeclareInnerCiteDelim,UndeclareInnerCiteDelim,DeclareInnerCiteDelimAlias,
    DeclareOuterCiteDelim,UndeclareOuterCiteDelim,DeclareOuterCiteDelimAlias,
    introcitepunct,volnumdelim,maintitletitledelim,voltitledelim,sernumdelim,
    volnumdatedelim,locdatedelim,locpubdelim,publocdelim,pubdatedelim,
    extradateonlycompcitedelim,introcitesep,introcitewidth,introcitesep},
}

\lstdefinestyle{extblxstylebibtex}{%
  language  = {extBibTeX},
  style     = {extblxstylegeneral},
}


\newcommand*{\highlight}[2][1]{\textcolor{highlight#1}{#2}}
\newcommand*{\highlightbf}[2][1]{\textcolor{highlight#1}{\textbf{#2}}}
\DeclareFieldFormat{highlight1}{\textcolor{highlight1}{#1}}
\DeclareFieldFormat{highlight2}{\textcolor{highlight2}{#1}}

\newtcolorbox{bibexample}[1][]{enhanced,
                               colframe=spot!75!black, colback=spot!5!white,
                               #1}
\newtcblisting{biblatexcode}{enhanced,
                             colframe=black!75!white, colback=black!5!white,
                             listing only,
                             frame hidden,
                             breakable,
                             listing style=extblxstylelatex}
\newtcblisting{bibtexfile}{enhanced,
                           colframe=black!75!white, colback=black!5!white,
                           listing only,
                           frame hidden,
                           breakable,
                           listing style = extblxstylebibtex}

\newtcbinputlisting{\inputexamplebibfile}[2][]{%
  listing file={#2},
  enhanced,
  colframe=black!75!white, colback=black!5!white,
  listing only,
  frame hidden,
  breakable,
  listing style = extblxstylebibtex,
  #1}

\makeatletter
\AtUsedriver*{%
  \let\newblock\relax
  \let\abx@macro@bibindex\@empty
  \let\abx@macro@pageref\@empty}


% this is taken from ltxdockit.cls, which is not loaded
\newrobustcmd*{\fnurl}[1][]{\hyper@normalise\ltd@fnurl{#1}}
\def\ltd@fnurl#1#2{\footnote{#1\hyper@linkurl{\Hurl{#2}}{#2}}}

\newrobustcmd*{\email}{\hyper@normalise\ltd@email}
\def\ltd@email#1{\href{mailto:#1}{#1}}

% title
\providecommand*{\titlepage}{}
\providecommand*{\titlefont}{}
\renewrobustcmd*{\titlepage}[1]{\setkeys{ltd@ttp}{#1}}
\renewcommand*{\titlefont}{\bfseries}
\define@key{ltd@ttp}{title}{\def\ltd@title@title{#1}}
\define@key{ltd@ttp}{subtitle}{\def\ltd@title@subtitle{#1}}
\define@key{ltd@ttp}{url}{\def\ltd@title@url{#1}}
\define@key{ltd@ttp}{author}{\def\ltd@title@author{#1}}
\define@key{ltd@ttp}{link}{\def\ltd@title@link{#1}}
\define@key{ltd@ttp}{revision}{\def\ltd@title@revision{#1}}
\define@key{ltd@ttp}{date}{\def\ltd@title@date{#1}}

\providecommand*{\printtitlepage}{}
\renewrobustcmd*{\printtitlepage}{%
  \begingroup
  \centering\titlefont
  \begingroup\LARGE
    \ifundef\ltd@title@url
      {\ltd@title@title}
      {\href{\ltd@title@url}{\ltd@title@title}}%
    \par
  \endgroup
  \vspace{0.25\baselineskip}
  \begingroup\large
    \ltd@title@subtitle\par
  \endgroup
  \vspace{0.25\baselineskip}
  \begin{multicols}{2}
  \raggedleft
    \ltd@title@author\par
    \expandafter\email\expandafter{\ltd@title@link}\par
  \raggedright
    Version \ltd@title@revision\par\ltd@title@date
  \end{multicols}
  \endgroup}

\BeforeStartingTOC[toc]{\begin{multicols}{2}}
\AfterStartingTOC[toc]{\end{multicols}}

\newrobustcmd*{\tex}{\TeX}
\newrobustcmd*{\etex}{\mbox{e-TeX}}
\newrobustcmd*{\pdftex}{pdf\-\tex}
\newrobustcmd*{\xetex}{Xe\-\tex}
\newrobustcmd*{\luatex}{Lua\-\tex}
\newrobustcmd*{\latex}{\LaTeX}%{La\kern-0.07em TeX}
\newrobustcmd*{\pdflatex}{pdf\-\latex}
\newrobustcmd*{\xelatex}{Xe\-\latex}
\newrobustcmd*{\lualatex}{Lua\-\latex}
\newrobustcmd*{\miktex}{Mik\-\tex}
\newrobustcmd*{\texlive}{\tex~live}
\newrobustcmd*{\bibtex}{Bib\kern-0.07em TeX}
\newrobustcmd*{\lppl}{\latex{} Project Public License}
\newrobustcmd*{\pdf}{\acr{PDF}}
\newrobustcmd*{\utf}{\mbox{\acr{UTF}-8}}

\pdfstringdefDisableCommands{%
  \def\tex{TeX}%
  \def\etex{e-TeX}%
  \def\xetex{XeTeX}%
  \def\latex{LaTeX}%
  \def\xelatex{XeLaTeX}%
  \def\bibtex{BibTeX}%
  \def\lppl{LaTeX Project Public License}%
  \def\pdf{PDF}%
  \def\utf{UTF-8}%
}

\let\accentcolour\spotcolor

\addtokomafont{section}{\accentcolour}
\addtokomafont{subsection}{\accentcolour}
\addtokomafont{subsubsection}{\accentcolour}

\renewcommand*{\verbatimfont}{\ttfamily}
\renewcommand*{\displayverbfont}{\ttfamily}
\renewcommand*{\marglistfont}{\accentcolour\sffamily\small}
\renewcommand*{\margnotefont}{\sffamily\small}
\renewcommand*{\optionlistfont}{\accentcolour\sffamily\displayverbfont}
\renewcommand*{\ltxsyntaxfont}{\ttfamily}
\renewcommand*{\ltxsyntaxlabelfont}{\accentcolour\displayverbfont}
\renewcommand*{\changelogfont}{\normalfont}
\renewcommand*{\changeloglabelfont}{\accentcolour\sffamily\bfseries}
\newcommand*{\stylelistlabelfont}{\accentcolour\sffamily\small}
\newcommand*{\bibfieldformatfont}{\sffamily}
\newcommand*{\bibfieldformatlabelfont}{\accentcolour\bibfieldformatfont\small}

\newenvironment*{stylelist}
  {\list{}{%
     \setlength{\labelwidth}{\marglistwidth}%
     \setlength{\labelsep}{\marglistsep}%
     \setlength{\leftmargin}{0pt}%
     \renewcommand*{\makelabel}[1]{\hss\stylelistlabelfont##1}}%
   \def\styleitem##1{%
     \item[{##1}]%
     \ltd@pdfbookmark{##1}{##1}}}
  {\endlist}

\newenvironment*{bibfieldformatlist}
  {\list{}{%
     \setlength{\labelwidth}{\marglistwidth}%
     \setlength{\labelsep}{\marglistsep}%
     \setlength{\leftmargin}{0pt}%
     \renewcommand*{\makelabel}[1]{\hss\bibfieldformatlabelfont##1}}%
   \def\bibfieldformatitem##1{%
     \item[{##1}]%
     \ltd@pdfbookmark{##1}{##1}}}
  {\endlist}

\newenvironment*{keymarglistbook}
  {\marglist
   \setlength{\itemsep}{0pt}%
   \raggedright
   \let\@@item\item
   \def\keyitem[##1]{%
     \@@item[{##1}]%
     \ltd@pdfbookmark{##1}{##1}}}
  {\endmarglist}

% modified for correct measurements
\def\ltd@option@i#1#2#3#4[#5]{%
  \item[#2]%
  \ltd@pdfbookmark{#1}{#1}%
  \begingroup\raggedright
  \ltd@textverb{=}%
  \settowidth\@tempdimb{\ltd@textverb{=}}%
  \settowidth\@tempdimc{\optionlistfont#2}%
  \ifdim\@tempdimc>\marglistwidth
    \@tempdimc=\dimexpr\@tempdimc-\marglistwidth\relax
  \else
    \@tempdimc=0pt
  \fi
  \@tempdima=\dimexpr\linewidth-\@tempdimb-\@tempdimc\relax
  \ifblank{#4}
    {}
    {\settowidth\@tempdimb{default: #4}%
     \@tempdima=\dimexpr\@tempdima-\@tempdimb-2em\relax}%
  \parbox[t]{\@tempdima}{\raggedright #3}%
  \ifblank{#4}
    {}
    {\hfill default:~#4}%
  \ifblank{#5}
    {}
    {\marginpar{\margnotefont #5}}%
  \par\endgroup
  \nobreak\vspace{\itemsep}}

\def\ltd@item@ii#1#2{%
  \ltd@itemsave
  \ifhmode
    \itemsep\z@
  \else
    \ltd@itembreak
  \fi
  \item[#1]%
  \ifblank{#2}
    {}
    {\phantomsection\label{exltd@itm@#2}}%
  \ltd@itemrest
  \ifblank{#2}{}{\ltd@pdfbookmark{#2}{#1}}}

\blx@inputonce{ext-biblatex-aux.def}{auxiliary code for ext-biblatex}{}{}{}{}
\newcommand*{\biblatexversion}{\extblx@requiredbiblatexversion}
\newcommand*{\biblatexdate}{\extblx@requiredbiblatexdate}

\AtEndPreamble{%
  \deflength{\marglistwidth}{(\oddsidemargin+2cm)*85/100}}

\newcommand*{\tikztextmark}[2]{%
  \tikz[remember picture,baseline,inner sep=0pt]\node [anchor=base] (#1) {#2};}

% *{<cmd>}{<x-shift>}{<y-shift>}
% unstarred version for commands defined by biblatex-ext
% starred version for standard biblatex commands
\def\punctarrow{%
  \@ifstar\punctarrow@ii\punctarrow@i}

\def\punctarrow@i{%
  \def\punctarrow@cmdfont{\bfseries}%
  \def\punctarrow@hyperref##1##2{%
    \hyperref[exltd@itm@##1]{##2}%
  }%
  \punctarrow@iii
}

\def\punctarrow@ii{
  \def\punctarrow@cmdfont{\itshape}%
  \let\punctarrow@hyperref\@secondoftwo
  \punctarrow@iii
}

\def\punctarrow@iii#1#2#3{%
  \ifdimcomp{#2}{<}{0pt}
    {\def\punctarrow@nodeanchor{east}}
    {\def\punctarrow@nodeanchor{west}}%
  \draw[spot,thick,latex-,rounded corners] (#1) |- ++ (#2,#3)
    node[anchor=\punctarrow@nodeanchor,text = black]
    {\punctarrow@hyperref{#1}{\punctarrow@cmdfont\cmd{#1}}};%
}
\makeatother

% By Stefan Kottwitz, see https://tex.stackexchange.com/a/799/35864
\newcommand*\justify{%
  \fontdimen2\font=0.4em% interword space
  \fontdimen3\font=0.2em% interword stretch
  \fontdimen4\font=0.1em% interword shrink
  \fontdimen7\font=0.1em% extra space
  \hyphenchar\font=`\-% allowing hyphenation
}

% no \mbox here, we may have to break things
\renewrobustcmd*{\sty}[1]{{\verbatimfont\justify #1}}
\renewrobustcmd*{\bibfield}[1]{\sty{#1}}
\renewrobustcmd*{\opt}[1]{\sty{#1}}
\newrobustcmd*{\bibmacro}[1]{\sty{#1}}
\renewrobustcmd*{\bibtype}[1]{\sty{@#1}}
\renewrobustcmd*{\cmd}[1]{\sty{\textbackslash #1}}
\newrobustcmd*{\bibfieldformat}[1]{{\bibfieldformatfont #1}}

\newcommand*{\ctan}{\mkbibacro{CTAN}}
\newcommand*{\gitbaseurl}{https://github.com/moewew/biblatex-ext}
\newcommand*{\extblxversion}{0.3}
\newcommand*{\biber}{Biber}
\newcommand*{\gitissuelink}[1]{%
  \href{\gitbaseurl/issues/#1}{issue \##1 on github}}

\newrobustcmd*{\CSdelim}{%
  \textcolor{spot}{\margnotefont context sensitive}}
\newrobustcmd*{\CSdelimMark}{%
  \leavevmode\marginpar{\CSdelim}}

\titlepage{%
  title    = {The \sty{biblatex-ext} Bundle},
  subtitle = {Extensions for the \sty{biblatex} standard styles},
  url      = {\gitbaseurl},
  author   = {Moritz Wemheuer},
  link     = {mwemheu@gmail.com},%
  revision = {\extblxversion},
  date     = {\DTMDate{2018-06-04}},
}

\hypersetup{%
  pdftitle    = {The biblatex-ext Bundle},
  pdfsubject  = {Extensions for the \sty{biblatex} standard styles},
  pdfauthor   = {Moritz Wemheuer},
  pdfkeywords = {latex, biblatex, bibtex, bibliography, references, citation},
}

\hyphenation{%
  star-red
  un-star-red
  bib-lio-gra-phy
  white-space
  bib-open-paren
  bib-close-paren
  bib-open-bracket
  bib-close-bracket
  main-title-after-title
  jour-vol-delim
}

\begin{document}

\printtitlepage
\tableofcontents

\section{Introduction}\label{sec:int}
\subsection{About}
The \sty{biblatex-ext} bundle provides an extended version of the standard
styles that come with \sty{biblatex}.
For each standard style this bundle provides a style with the same name
prefixed with \texttt{ext-} which can be used as a drop-in replacement for the
standard style -- for example, the replacement for \sty{authoryear-icomp}
is called \sty{ext-authoryear-icomp}.

The aim of the styles of this bundle is to offer a simple interface to change
some of the stylistic decisions made for the standard styles that would
otherwise need cumbersome and tedious redefinitions.
Additionally, some customisation features that were not deemed appropriate
for inclusion in the \sty{biblatex} kernel are provided.
Other than that the styles are as close to the standard styles as possible.
They do not attempt to offer options or commands for customisations that
are already fairly simple to achieve with the tools provided by the standard
styles.\footnote{Please be gentle and allow for a lot of wiggle room for what
exactly \enquote{simple} means. And don't get mad if the principle is not
followed at all times and the occasional solution for things that already are
\enquote{simple enough} pop up here and there.}

The initial motivation for this bundle was what has now become the option
\opt{in\-name\-before\-title}.
It is fairly straighforward to print the name of the editor of an
\bibtype{incollection} before the \bibfield{booktitle}.
But if one does not want to resort to clever tricks,%
\footnote{See \url{https://tex.stackexchange.com/q/122218/} and
\url{https://tex.stackexchange.com/q/173638/} for example.}
the modifications needed to do this in a stable, safe and clean manner by
redefining the bibliography drivers can easily amass hundred lines of code --
code you may not want to see in your preamble.
Some contributed \sty{biblatex} styles already place the editor in the
desired position, but you may not want to commit to the other changes implied
by switching to one of those styles.
Especially styles written for the sole purpose of implementing the requirements
of a particular style guide may have to go to great lengths to do so and are
therefore not as easily modified as the standard styles.
The styles of this bundle, on the other hand, try to stay as close to the
standard styles as possible both in output and implementation to allow you
to customise the styles with minimum additional effort over the standard styles.


A few words of warning:
The styles of this bundle are only really useful if you want to use one of their
features to avoid having to go through the lengthy and tedious redefinitions the
standard styles would require.
Before you get involved in modifying the standard styles or one of the styles
of this bundle, you may want to have a look at the host of other styles
available for \sty{biblatex},\fnurl{https://www.ctan.org/topic/biblatex}
maybe you are lucky and the style you are looking for has already been
implemented by someone else.
As was the intention, most methods to modify the standard styles are also
applicable to the styles of this bundle, but you may break some of their
features if you happen to modify something that the \sty{ext-} styles redefine
themselves.
While many contributed \sty{biblatex} styles are examples of good \sty{biblatex}
style coding, this can not be said of all of the files included in this bundle.
Especially the citation styles for compact citations have to work harder
to implement the citation delimiter feature properly.
So if you want to get inspired for your \sty{biblatex} coding, be warned that
terrible things lurk in the \sty{.cbx} files.
The standard \sty{.cbx} files will give you a much better impression of how
things should be done.


\subsection{Requirements}\label{sec:req}
The use of the styles requires a current version of the \sty{biblatex} package.
At the time of writing the latest version of \sty{biblatex} is
\biblatexversion{} (dated \biblatexdate),
that version is required for the styles to work properly.
A warning will be issued if you use an older version.
You may choose to ignore that warning, but the styles can not be guaranteed to
work properly in that case:
you might get other warnings or errors, and some features of the style might
just fail silently.

Use of the \biber{} backend is strongly encouraged.
Most of the new features of this bundle do not require \biber{} explicitly in
their implementation, but many \sty{biblatex} standard features only work
properly with \biber{}.

\subsection{Installation}\label{sec:install}
This style bundle is available on \ctan.%
\fnurl{https://ctan.org/pkg/biblatex-ext}
The current release is also available in \miktex{} and \texlive{} 2018 or
newer.
If at all possible you should install this bundle via your \tex{} distribution
(using \miktex{} Console\fnurl{https://miktex.org/howto/miktex-console} or
\sty{tlmgr} for \texlive\fnurl{https://www.tug.org/texlive/pkginstall.html}).
If you must install this package manually, get the files from \ctan{}
and install the \sty{.bbx}, \sty{.cbx} and \sty{.def} files preferably to
\path{tex/latex/biblatex-ext} of your local or home \TeX{} tree, the other
files (\path{CHANGES.md}, \path{README.md}, \path{biblatex-ext.tex},
\path{biblatex-ext.pdf} and \path{biblatex-ext-examples.bib}) go to
\path{doc/latex/biblatex-ext}, you may have to refresh your file name database
afterwards, so \tex{} can find the files.

\subsection{License}\label{sec:license}
Permission is granted to copy, distribute and\slash or modify this software
under the terms of the \lppl, version 1.3c%
\fnurl{https://www.latex-project.org/lppl/lppl-1-3c.txt}
or (at your option) any later version%
\fnurl{https://www.latex-project.org/lppl.txt}.
This bundle is maintained by Moritz Wemheuer (\textcopyright 2017--2018).


\subsection{Feedback}\label{sec:feedback}
You can use the \sty{biblatex-ext} project page on GitHub%
\footnote{\url{\gitbaseurl}} to report bugs and
submit suggestions and feature requests, or you can do so via email.

If you do not want to report a bug or request a feature, but are simply in need
of assistance, you might want to consider posting your question on the
\texttt{comp.text.tex} newsgroup or \tex{} -- \latex{} Stack Exchange.%
\fnurl{https://tex.stackexchange.com/questions/tagged/biblatex}

\section{Use}\label{sec:use}
The \sty{biblatex-ext} bundle is a collection of \sty{biblatex} style files.
You can load the styles exactly as you would load the standard styles:
\begin{biblatexcode}
\usepackage[style=(*@\prm{style}@*)]{biblatex}
\end{biblatexcode}
The naming of the styles follows the scheme
\mbox{\texttt{ext-}\prm{standard style}}, e.g.,
the style corresponding to \sty{authoryear-icomp} is called
\sty{ext-authoryear-icomp}.

This manual assumes familiarity with the concepts of \sty{biblatex} and does
not attempt to explain any of the standard \sty{biblatex} features, so you may
want to have the \sty{biblatex} documentation%
\fnurl{http://mirrors.ctan.org/macros/latex/contrib/biblatex/doc/biblatex.pdf}
at hand to fully appreciate what is going on.
If you are new to \sty{biblatex} the three hundred odd pages of the manual
can be hard to stomach at first, so you may want to have a look at a more
gentle introduction.
Of the many introductory texts and tutorials available on the internet%
\fnurl{https://tex.stackexchange.com/q/13509/35864}
the author particularly likes
Knut Hegna and Dag Langmyhr's \emph{Local Guide to \sty{biblatex}}%
\fnurl{http://dag.at.ifi.uio.no/public/doc/biblatex-guide.pdf}
and Paul Stanley's \emph{\sty{biblatex} -- An Easier Read}%
\fnurl{https://github.com/PaulStanley/biblatex-tutorial/releases}.
French speakers may want to consider Maïeul Rouquette's
\foreignlanguage{french}{\emph{(Xe)\LaTeX{} Appliqué aux sciences humaines}}%
available on \ctan\fnurl{https://ctan.org/pkg/latex-sciences-humaines}.
If you read German you may be interested in Dominik Waßenhoven's two-part
series \foreignlanguage{ngerman}{\emph{Bibliographien erstellen mit
\sty{biblatex}}}\fnurl{http://biblatex.dominik-wassenhoven.de/dtk.shtml}
in \foreignlanguage{ngerman}{\emph{Die \TeX nische Komödie}}
2/2008\fnurl{http://www.dante.de/DTK/Ausgaben/2008-2.pdf} (pp.~53--75)
and 4/2008\fnurl{http://www.dante.de/DTK/Ausgaben/2008-4.pdf} (pp.~31--51).
Please keep in mind that some of these texts were written a while ago and
that \sty{biblatex} is actively developed: technical details may have changed
and new features make some things easier.

\section{Styles}\label{sec:styles}
This bundle provides an extended version of each standard style as well as a
few new styles.
\subsection{Standard styles}\label{sec:styles:standard}
Please refer to the \sty{biblatex} documentation%
\fnurl{http://mirrors.ctan.org/macros/latex/contrib/biblatex/doc/biblatex.pdf}
and the style examples%
\fnurl{http://mirrors.ctan.org/macros/latex/contrib/biblatex/doc/examples/}
for a more detailed description of each standard style.
The relations between the styles are exactly as in their standard counterparts.
The \texttt{ext-} styles only build on top of the standard files.
\begin{stylelist}
\styleitem{ext-numeric}
An extended version of the standard \sty{numeric} style for citations with
numeric labels.
\begin{bibexample}[]
[1, 2, 5, 6, 7]
\end{bibexample}

\styleitem{ext-numeric-comp}
An extended version of the standard \sty{numeric-comp} style.
Similar to \sty{ext-numeric}, but citation labels are
compressed to give \enquote{[1--3]} instead of \enquote{[1, 2, 3]}.
\begin{bibexample}[]
[1, 2, 5--7]
\end{bibexample}

\styleitem{ext-numeric-verb}
An extended version of the standard \sty{numeric-verb} style.
This style is similar to the \sty{numeric} style, but each citation label
is in its own set of brackets: \enquote{[1], [2], [3]}.
\begin{bibexample}[]
[1], [3], [5], [6], [7]
\end{bibexample}

\styleitem{ext-alphabetic}
An extended version of the standard \sty{alphabetic} style for citations with
alphabetic labels derived from author name and year.
\begin{bibexample}[]
[SR98, Knu86c]
\end{bibexample}

\styleitem{ext-alphabetic-verb}
An extended version of the standard \sty{alphabetic-verb} style.
This style is based on \sty{ext-alphabetic}, but like
\sty{ext-numeric-verb} places each citation label in its own pair of
brackets: \enquote{[SR98], [Knu86c]}.
\begin{bibexample}[]
[SR98], [Knu86c]
\end{bibexample}

\styleitem{ext-authoryear}
An extended version of the standard \sty{authoryear} style for citations
using author name and year.
\begin{bibexample}
Sigfridsson and Ryde 1998
\end{bibexample}

\styleitem{ext-authoryear-comp}
An extended version of the standard \sty{authoryear-comp} style.
The style is based on the author-year citations of \sty{ext-authoryear},
but several works by the same author are compressed by not displaying the
author multiple times: \enquote{Knuth 1984, 1986} instead of
\enquote{Knuth 1984; Knuth 1986}.
\begin{bibexample}
Knuth 1984, 1986
\end{bibexample}

\styleitem{ext-authoryear-ibid}
An extended version of the standard \sty{authoryear-ibid} style.
This style is similar to \sty{ext-authoryear}, but repeated citations are
replaced with \enquote{ibidem}.
\begin{bibexample}
Knuth 1984\quad ibid.
\end{bibexample}

\styleitem{ext-authoryear-icomp}
An extended version of the standard \sty{authoryear-icomp} style.
This style combines the two styles \sty{ext-authoryear-comp} and
\sty{ext-authoryear-ibid}.
\begin{bibexample}
Knuth 1984, 1986 \quad Sigfridsson and Ryde 1998 \quad ibid.
\end{bibexample}

\styleitem{ext-authortitle}
An extended version of the standard \sty{authortitle} style for citations
by author and title.
\begin{bibexample}
Maron, \emph{Animal Triste}
\end{bibexample}

\styleitem{ext-authortitle-comp}
An extended version of the standard \sty{authortitle-comp} style.
This style is based on \sty{ext-authortitle} and compresses several citations
by the same author just like \sty{ext-author\-year-comp}:
\enquote{Aristotle, \emph{Physics}, \emph{Poetics}} instead of
\enquote{Aristotle, \emph{Physics}; Aristotle, \emph{Poetics}}.
\begin{bibexample}
Aristotle, \emph{Physics}, \emph{Poetics}
\end{bibexample}

\styleitem{ext-authortitle-ibid}
An extended version of the standard \sty{authortitle-ibid} style.
This style is similar to \sty{ext-authortitle}, but replaces repeated
citations of the same work with \enquote{ibidem}.
\begin{bibexample}
Maron, \emph{Animal Triste}\quad ibid.
\end{bibexample}

\styleitem{ext-authortitle-icomp}
An extended version of the standard \sty{authortitle-icomp} style.
This style combines \sty{ext-authortitle-comp} and
\sty{ext-authortitle-ibid}.
\begin{bibexample}
Aristotle, \emph{Physics}, \emph{Poetics}\quad Maron, \emph{Animal Triste}\quad
ibid.
\end{bibexample}

\styleitem{ext-authortitle-terse}
An extended version of the standard \sty{authortitle-terse} style.
This style is similar to \sty{ext-authortitle}, but the title is omitted in
citations if there is only one work by the relevant author.
\begin{bibexample}
Sigfridsson and Ryde \quad Aristotle, \emph{Physics}; Aristotle, \emph{Poetics}
\end{bibexample}

\styleitem{ext-authortitle-tcomp}
An extended version of the standard \sty{authortitle-tcomp} style.
This style combines \sty{ext-authortitle-terse} and
\sty{ext-authortitle-comp}.
\begin{bibexample}
Sigfridsson and Ryde \quad Aristotle, \emph{Physics}, \emph{Poetics}
\end{bibexample}

\styleitem{ext-authortitle-ticomp}
An extended version of the standard \sty{authortitle-ticomp} style.
This style combines \sty{ext-authortitle-terse},
\sty{ext-authortitle-comp} and \sty{ext-authortitle-ibid}.
\begin{bibexample}
Aristotle, \emph{Physics}, \emph{Poetics} \quad Sigfridsson and Ryde \quad ibid.
\end{bibexample}

\styleitem{ext-verbose}
An extended version of the standard \sty{verbose} style.
This style shows the full bibliographic reference the first time a work is
cited.

\styleitem{ext-verbose-ibid}
An extended version of the standard \sty{verbose-ibid} style.
Based on \sty{ext-verbose}, repeated citations to the same work are replaced
with \enquote{ibidem}.

\styleitem{ext-verbose-note}
An extended version of the standard \sty{verbose-note} style.
Based on \sty{ext-verbose} and intended for use in footnotes, subsequent
citations link back to the footnote the entry was cited at first and in full.

\styleitem{ext-verbose-inote}
An extended version of the standard \sty{verbose-inote} style.
Similar to \sty{verbose-note}, but repeated citations to the same work are
replaced with \enquote{ibidem}.

\styleitem{ext-verbose-trad1}
An extended version of the standard \sty{verbose-trad1} style.
This style makes extensive use of scholarly abbreviations and is otherwise
similar to \sty{ext-verbose}.

\styleitem{ext-verbose-trad2}
An extended version of the standard \sty{verbose-trad2} style.
The style is similar to \sty{ext-verbose-inote} and
uses scholarly abbreviations to shorten citations.

\styleitem{ext-verbose-trad3}
An extended version of the standard \sty{verbose-trad3} style.
This style is similar to \sty{ext-verbose-trad2}.
\end{stylelist}

\subsection{New styles}\label{sec:styles:new}
The following styles are not mere extensions of a particular standard style,
instead they implement new combinations of the concepts available in the
standard styles.
\begin{stylelist}
\styleitem{ext-authoryear-terse}
An author-year citation style that suppresses years for author lists with only
one work in the bibliography.
This style is like \sty{ext-authortitle-terse}, but it is based on
\sty{ext-authoryear} and not on \sty{ext-authortitle}.
\begin{bibexample}
Sigfridsson and Ryde \quad Knuth 1984 \quad Knuth 1986
\end{bibexample}

\styleitem{ext-authoryear-tcomp}
A compact author-year citation style that suppresses years for author lists
with only one work in the bibliography.
This style is like \sty{ext-authortitle-tcomp}, but it is based on
\sty{ext-authoryear} and not on \sty{ext-authortitle}.
\begin{bibexample}
Sigfridsson and Ryde \quad Knuth 1984, 1986
\end{bibexample}

\styleitem{ext-authoryear-ticomp}
A compact author-year citation style with \enquote{ibidem} function that
suppresses years for author lists with only one work in the bibliography.
This style is like \sty{ext-authortitle-ticomp}, but it is based on
\sty{ext-authoryear} and not on \sty{ext-authortitle}.
\begin{bibexample}
Sigfridsson and Ryde \quad ibid. \quad Knuth 1984, 1986
\end{bibexample}

\end{stylelist}


\section{Options}\label{sec:opt}
All options of the \sty{biblatex} package are supported and each style supports
the options of its standard counterpart.

\subsection{General options}\label{sec:opt:gen}
Additionally, all styles support the following options in global, per-type and
per-entry scope.
The default values are such that the styles can be used as drop-in replacement
for the standard files without significant changes in output.

\begin{optionlist}
\boolitem[true]{articlein}
Whether or not to display \enquote{in:} before the journal information in
\bibtype{article} entries.
All other entry types are not affected by this option.

\begingroup
\makeatletter
\togglefalse{bbx:doi}
\DeclareFieldFormat{highlighthere}{%
  \iffieldequalstr{entrykey}{sigfridsson}
    {\textcolor{highlight1}{#1}}
    {\iffieldequalstr{entrykey}{westfahl:space}
       {\textcolor{highlight2}{#1}}
       {#1}}}
\renewbibmacro*{in:}{%
  \blx@begunit\blx@endunit
  \printtext[highlighthere]{%
    \printtext{%
      \bibstring{in}\intitlepunct}}}

\begin{bibexample}[title={\kvopt{articlein}{true}}]
\toggletrue{bbx:articlein}
\exampleprintbib{sigfridsson,westfahl:space}
\end{bibexample}

\begin{bibexample}[title={\kvopt{articlein}{false}}]
\togglefalse{bbx:articlein}
\exampleprintbib{sigfridsson,westfahl:space}
\end{bibexample}
\makeatother
\endgroup

\boolitem[false]{citeinxref}
This option controls if \bibtype{inbook}, \bibtype{incollection} and
\bibtype{inproceedings} entries that are tied to a parent entry with
\bibfield{xref} or \bibfield{crossref} should cite their parent in the
bibliography if the parent is listed in the bibliography as a separate entry.
With the default setting \kvopt{citeinxref}{false} the parent is not cited, the
entry is shown as in the standard styles. If the option is set to
\opt{true}, the block following the \enquote{in:} is replaced by a citation
to the parent entry.
The option itself does \emph{not} cause the parent entry to be added to the
bibliography, this needs to happen either explicitly by citing the parent
(this includes \cmd{nocite}) or implicitly via the \opt{minxrefs} or
\opt{mincrossrefs} option.

\nocite{westfahl:frontier}
\begin{bibexample}[title={\kvopt{citeinxref}{true}}]
\makeatletter
\renewbibmacro*{crosscite}[1]{%
  \printtext[highlight1]{%
    \iftoggle{bbx:citeinxref}
      {\iffieldundef{crossref}
         {\iffieldundef{xref}
            {\usebibmacro{#1}}
            {\printtext{\bbx@cite@inxref{\thefield{xref}}}}}
         {\printtext{\bbx@cite@inxref{\thefield{crossref}}}}}
      {\usebibmacro{#1}}}}
\makeatother
\toggletrue{bbx:citeinxref}
\exampleprintbib{westfahl:space}
\end{bibexample}

\begin{bibexample}[title={\kvopt{citeinxref}{false}}]
\makeatletter
\renewbibmacro*{crosscite}[1]{%
  \printtext[highlight1]{%
    \iftoggle{bbx:citeinxref}
      {\iffieldundef{crossref}
         {\iffieldundef{xref}
            {\usebibmacro{#1}}
            {\printtext{\bbx@cite@inxref{\thefield{xref}}}}}
         {\printtext{\bbx@cite@inxref{\thefield{crossref}}}}}
      {\usebibmacro{#1}}}}
\makeatother
\togglefalse{bbx:citeinxref}
\exampleprintbib{westfahl:space}
\end{bibexample}

\boolitem[false]{innamebeforetitle}
Whether or not the \bibfield{editor} should be moved before the
\bibfield{booktitle} field for \bibtype{inbook}, \bibtype{incollection} and
\bibtype{inproceedings} entries.

\begin{bibexample}[title={\kvopt{innamebeforetitle}{true}}]
\makeatletter
\renewbibmacro*{bbx:in:editor}[1]{%
  \ifboolexpr{
    test \ifuseeditor
    and
    not test {\ifnameundef{editor}}
  }
    {\printtext[highlight1]{%
       \ifboolexpr{togl {bbx:innameidem} and test {\bbx@ineditoridem}}
         {\bibstring[\mkibid]{idem\thefield{gender}}}
         {\printnames[ineditor]{editor}}%
       \setunit{\printdelim{editortypedelim}}%
       \usebibmacro{#1}}%
     \clearname{editor}}
    {}}
\makeatother
\toggletrue{bbx:innamebeforetitle}
\exampleprintbib{pines}
\end{bibexample}

\begin{bibexample}[title={\kvopt{innamebeforetitle}{false}}]
\renewbibmacro*{byeditor+others}{%
  \ifnameundef{editor}
    {}
    {\printtext[highlight1]{%
       \usebibmacro{byeditor+othersstrg}%
       \setunit{\addspace}%
       \printnames[byeditor]{editor}%
       \newunit}
     \clearname{editor}}%
  \usebibmacro{byeditorx}%
  \usebibmacro{bytranslator+others}}
\togglefalse{bbx:innamebeforetitle}
\exampleprintbib{pines}
\end{bibexample}

\boolitem[false]{innameidem}
Whether or not the \bibfield{editor} of the \bibfield{booktitle}
for the entry types \bibtype{inbook}, \bibtype{incollection} and
\bibtype{inproceedings} is replaced by \enquote{idem} in case the
\bibfield{editor} and \bibfield{author} name lists coincide.
This option only has an effect if \opt{innamebeforetitle} is set to \opt{true}.

\begin{bibexample}[title={\kvopt{innameidem}{true}}]
\togglefalse{bbx:isbn}
\makeatletter
\renewbibmacro*{bbx:in:editor}[1]{%
  \ifboolexpr{
    test \ifuseeditor
    and
    not test {\ifnameundef{editor}}
  }
    {\printtext[highlight1]{%
       \ifboolexpr{togl {bbx:innameidem} and test {\bbx@ineditoridem}}
         {\bibstring[\mkibid]{idem\thefield{gender}}}
         {\printnames[ineditor]{editor}}%
       \setunit{\printdelim{editortypedelim}}%
       \usebibmacro{#1}%
       \clearname{editor}}}
    {}}
\makeatother
\toggletrue{bbx:innamebeforetitle}\toggletrue{bbx:innameidem}
\exampleprintbib{gaonkar:in}
\end{bibexample}

\begin{bibexample}[title={\kvopt{innameidem}{false}}]
\togglefalse{bbx:isbn}
\makeatletter
\renewbibmacro*{bbx:in:editor}[1]{%
  \ifboolexpr{
    test \ifuseeditor
    and
    not test {\ifnameundef{editor}}
  }
    {\printtext[highlight1]{%
       \ifboolexpr{togl {bbx:innameidem} and test {\bbx@ineditoridem}}
         {\bibstring[\mkibid]{idem\thefield{gender}}}
         {\printnames[ineditor]{editor}}%
       \setunit{\printdelim{editortypedelim}}%
       \usebibmacro{#1}%
       \clearname{editor}}}
    {}}
\makeatother
\toggletrue{bbx:innamebeforetitle}\togglefalse{bbx:innameidem}
\exampleprintbib{gaonkar:in}
\end{bibexample}


\boolitem[false]{maintitleaftertitle}
Whether or not the \bibfield{maintitle} is printed after the
\bibfield{title} or \bibfield{booktitle} of the work.
If \opt{maintitleaftertitle} is \opt{true}, the \bibfield{volume} field will be
printed with the \bibfield{volumeof} format.

\begingroup
\makeatletter
\renewbibmacro*{maintitle+title}{%
  \blx@begunit\blx@endunit
   \iftoggle{bbx:maintitleaftertitle}
     {}
     {\iffieldsequal{maintitle}{title}
        {\clearfield{maintitle}%
         \clearfield{mainsubtitle}%
         \clearfield{maintitleaddon}}
        {\printtext[highlight1]{%
           \iffieldundef{maintitle}
             {}
             {\usebibmacro{maintitle}%
              \newunit\newblock
              \iffieldundef{volume}
                {}
                {\printfield{volume}%
                 \printfield{part}%
                 \setunit{\maintitletitledelim}}}}}}%
  \printtext[highlight2]{\usebibmacro{title}\blx@begunit\blx@endunit}%
  \printunit{}%
   \iftoggle{bbx:maintitleaftertitle}
     {\iffieldsequal{maintitle}{title}
        {\clearfield{maintitle}%
         \clearfield{mainsubtitle}%
         \clearfield{maintitleaddon}}
        {\iffieldundef{maintitle}
           {}
           {\setunit{\titlemaintitledelim}%
            \printtext[highlight1]{%
              \iffieldundef{volume}
                {}
                {\printfield[volumeof]{volume}%
                 \printfield{part}%
                 \setunit{\addspace}%
                 \bibstring{ofseries}%
                 \setunit{\addspace}}%
              \usebibmacro{maintitle}}}}
       {}}%
  \newunit}%
\renewcommand*{\maintitletitledelim}{\highlight{\addcolon\space}}
\makeatother
\begin{bibexample}[title={\kvopt{maintitleaftertitle}{true}}]
\togglefalse{bbx:isbn}
\toggletrue{bbx:maintitleaftertitle}
\exampleprintbib{knuth:ct:a}
\end{bibexample}

\begin{bibexample}[title={\kvopt{maintitleaftertitle}{false}}]
\togglefalse{bbx:isbn}
\togglefalse{bbx:maintitleaftertitle}
\exampleprintbib{knuth:ct:a}
\end{bibexample}
\endgroup
\end{optionlist}

\subsection{Style-specific options}\label{sec:opt:style}
The \opt{dashed} option of the \sty{authoryear}- and \sty{authortitle}-like
bibliography styles allows for finer control over the dashes now.
\begin{optionlist}
\optitem[true]{dashed}{\opt{true}, \opt{false}, \opt{fullhash},
                       \opt{bibnamehash}}

This option controls whether or not recurring lists of authors/editors in the
bibliography are replaced with a dash.
The standard values \opt{true} and \opt{false} are still valid and give the
exact same output as in the standard styles.
This means that the output is fully compatible with the standard styles.
The new values \opt{fullhash} and \opt{bibnamehash} differ in how exactly they
determine if a list of authors/editors is the same as the previous.
\begin{valuelist}
\item[true] An alias for \opt{fullhash}.
\item[false] Disable this feature.
\item[bibnamehash] Replace recurring name lists with a dash.
                   Compare name lists using \bibfield{bib\-name\-hash}, taking
                   into account only names that are actually listed in the
                   bibliography account.
\item[fullhash] Replace recurring name lists with a dash.
                Compare name lists using \bibfield{fullhash}, taking into
                account all names in the list, even those that are truncated
                and do not appear in the bibliography.
\end{valuelist}


\begin{refsection}
\makeatletter
Assuming \kvopt{maxnames}{1} and no name list disambiguation
(\kvopt{uniquelist}{false}), the four entries
\inputexamplebibfile[listing options={linerange={1-16,18-23,25},
                                      style = extblxstylebibtex}]
                    {biblatex-ext-examples.bib}
give
\renewcommand*{\bibnamedash}{\textbf{\textemdash\addspace}}
\DeclareFieldFormat{highlighthere}{%
  \iffieldequalstr{entrykey}{elk:einio}
    {\highlight[1]{#1}}
    {\iffieldequalstr{entrykey}{appleby:abl}
       {\highlight[2]{#1}}
       {#1}}}
\renewbibmacro*{author}{%
  \printtext[highlighthere]{%
  \ifboolexpr{
    test \ifuseauthor
    and
    not test {\ifnameundef{author}}
  }
    {\usebibmacro{bbx:dashcheck}
       {\printtext{\bibnamedash}}
       {\usebibmacro{bbx:savehash}%
        \printnames{author}}%
        \iffieldundef{authortype}
          {\setunit{\printdelim{nameyeardelim}}}
          {\setunit{\printdelim{authortypedelim}}}%
     \iffieldundef{authortype}
       {}
       {\usebibmacro{authorstrg}%
        \setunit{\printdelim{nameyeardelim}}}}%
    {\global\undef\bbx@lasthash
     \usebibmacro{labeltitle}%
     \setunit*{\printdelim{nonameyeardelim}}}}%
  \usebibmacro{date+extradate}}
\let\ExecuteBibliographyOptions\@gobble
\nocite{elk:bronto,elk:einio,appleby:abl,appleby:civ}
\begin{bibexample}[title={\kvopt{dashed}{false}}]
\csuse{extblx@opt@dashed@false}
\printbibliography[heading=none]
\end{bibexample}

\begin{bibexample}[title={\kvopt{dashed}{bibnamehash}}]
\csuse{extblx@opt@dashed@bibnamehash}
\printbibliography[heading=none]
\end{bibexample}

\begin{bibexample}[title={\kvopt{dashed}{fullhash}}]
\csuse{extblx@opt@dashed@fullhash}
\printbibliography[heading=none]
\end{bibexample}
\makeatother
\end{refsection}
With \kvopt{dashed}{bibnamehash} the dash replaces the name list if they are
indistinguishable in the bibliography, while with \kvopt{dashed}{fullhash} the
lists are only replaced if they are indistinguishable in the data source.


The \opt{introcite} option is available for the bibliography styles of the
\sty{authoryear} and \sty{authortitle} family.
It can not be used with citation styles of the \sty{verbose} family.
\optitem[false]{introcite}{\opt{false}, \opt{plain}, \opt{label}}

This option controls whether or not the citation label is repeated in the
bibliography.
There are two possible output formats.
\begin{valuelist}
\item[false] Do not show the citation label in the bibliography.
\item[plain] Show the citation label at the beginning of an entry.
\item[label] Show the citation label as the label of a list similar to the
             \sty{numeric} or \sty{alphabetic} styles.
\end{valuelist}

The difference between \opt{plain} and \opt{label} is that the former simply
prints the citation label at the beginning of the entry, while the latter
prints the citation label similar to the item labels in a list or the numeric
labels in a \sty{numeric} bibliography.

\begingroup
\togglefalse{bbx:doi}
\setlength{\introcitewidth}{5.5\biblabelsep}
\DeclareFieldFormat{bbx:introcite}{\highlight{#1}}
\renewcommand*{\introcitepunct}{\highlight{\addcolon}\space}
\makeatletter
\begin{bibexample}[title={\kvopt{introcite}{false}}]
\csuse{extblx@opt@dashed@false}
\csletcs{extblx@introcite}{extblx@opt@introcite@false}
\exampleprintbib{sigfridsson,knuth:ct:a,knuth:ct:b}
\end{bibexample}

\begin{bibexample}[title={\kvopt{introcite}{plain}}]
\csuse{extblx@opt@dashed@false}
\csletcs{extblx@introcite}{extblx@opt@introcite@plain}
\exampleprintbib{sigfridsson,knuth:ct:a,knuth:ct:b}
\end{bibexample}

\begin{bibexample}[title={\kvopt{introcite}{label}}]
\csuse{extblx@opt@dashed@false}
\csletcs{extblx@introcite}{extblx@opt@introcite@label}
\exampleprintbib{sigfridsson,knuth:ct:a,knuth:ct:b}
\end{bibexample}
\makeatother
\endgroup

The label produced by the \opt{plain} option can be customised as follows.
\begin{ltxsyntax}
\csitem{introcitepunct} The punctuation insterted between the label and the
  rest of the entry with \kvopt{introcite}{plain}.
 The default value is a colon followed by a space.
\begin{bibexample}
\togglefalse{bbx:doi}
\renewcommand*{\introcitepunct}{\highlight{\textbf{\addcolon}\textvisiblespace}}
\csuse{extblx@opt@dashed@false}
\csletcs{extblx@introcite}{extblx@opt@introcite@plain}
\exampleprintbib{sigfridsson}
\end{bibexample}
\end{ltxsyntax}
\begin{keymarglistbook}
  \keyitem[bbx:introcite:plain:keeprelated] This toggle controls whether or
  not the citation label is also repeated for default related entries.
  The default value \opt{false} suppresses the label for related entries.
\begin{bibexample}[title={\texttt{\string\togglefalse\{%
  bbx:introcite:plain:keeprelated\}} (default)}]
\togglefalse{bbx:doi}
\csletcs{extblx@introcite}{extblx@opt@introcite@plain}
\exampleprintbib{vizedom:related}
\end{bibexample}
\begin{bibexample}[title={\texttt{\string\toggletrue\{%
  bbx:introcite:plain:keeprelated\}}}]
\toggletrue{bbx:introcite:plain:keeprelated}
\renewbibmacro*{related:init}{%
  \csundef{bbx:relatedloop}%
  \iftoggle{bbx:introcite:plain:keeprelated}{%
    \DeclareFieldFormat{bbx:introcite}{\highlight{##1}}%
    \renewcommand*{\introcitepunct}{\highlight{\addcolon}\space}%
  }{\renewbibmacro{introcite:plain}{}}}
\csuse{extblx@opt@dashed@false}
\csletcs{extblx@introcite}{extblx@opt@introcite@plain}
\exampleprintbib{vizedom:related}
\end{bibexample}
\end{keymarglistbook}

The \opt{label} option can be configured to not allow the label to run into
the remaining bibliography entry thus creating the appearance of a tabular-like
bibliography.
The citation label is not broken across lines, instead it moves the entry text
into the next line with \cmd{introcitebreak} if the width of the citation is
greater than \len{introcitewidth}.

\begin{ltxsyntax}
\lenitem{introcitewidth} The maximum width of the citation label.
  The initial value is 8 times \len{biblabelsep}.
\lenitem{introcitesep} This length sets the minimal space between the end of the
  citation label and the beginning of the rest of the entry.
  The initial value is \len{biblabelsep}.
\csitem{introcitebreak} The command to execute if a citation label exceeds
  \len{introcitewidth}. The default is \cs{leavevmode}\cs{newline}.
\end{ltxsyntax}

\makeatletter
\togglefalse{bbx:doi}
\begingroup
\setlength{\introcitewidth}{4.2\biblabelsep}
\setlength{\introcitesep}{3.8\biblabelsep}
\begin{bibexample}[enhanced, title={Lengths for \kvopt{introcite}{label}},
overlay={%
  \draw[highlight1, line width=.24mm,|-|] (frame.west)++(5.5mm,-.38cm) --
    node [at end, below=1pt] {\len{introcitewidth}} ++
    (\introcitewidth,0);
  \draw[highlight2, line width=.24mm,|-|] (frame.west)++
    (5.5mm+\introcitewidth,-.8mm) -- node [at start, above=2pt]
    {\len{introcitesep}} ++ (\introcitesep,0);
  \draw[highlight2, line width=.24mm,|-|] (frame.west)++
    (5.5mm+\introcitewidth,2.32cm) -- ++ (\introcitesep,0);
}
]
\csuse{extblx@opt@dashed@false}
\csletcs{extblx@introcite}{extblx@opt@introcite@label}
\exampleprintbib{sigfridsson,coleridge,geer}
\end{bibexample}
\endgroup

\begin{bibexample}[enhanced, title={\kvopt{introcite}{label} with empty
  \cs{introcitebreak}}]
\renewcommand*{\introcitebreak}{}
\setlength{\introcitewidth}{1.8cm}
\csuse{extblx@opt@dashed@false}
\csletcs{extblx@introcite}{extblx@opt@introcite@label}
\exampleprintbib{coleridge,geer}
\end{bibexample}

\begin{bibexample}[enhanced, title={\kvopt{introcite}{label} with
  \len{introcitewidth} set to zero and \len{introcitesep} equal to
  \len{bibhang}}]
\setlength{\introcitewidth}{0pt}
\setlength{\introcitesep}{\bibhang}
\csuse{extblx@opt@dashed@false}
\csletcs{extblx@introcite}{extblx@opt@introcite@label}
\exampleprintbib{sigfridsson,geer}
\end{bibexample}
\makeatother

The appearance of the citation label can be customised mostly as if it were
produced by a true citation command called \cmd{bbx:introcite}.
The delimiter context is \sty{bbx:introcite}, the inner citation delimiters
can be accessed as \sty{bbx:introcite} as well.
The label does not have outer citation delimiters, you can use the wrapper
field format \sty{bbx:introcite} instead. In fact this approach is more
versatile than the outer citation delimiter feature (see the discussion in
\cref{sec:opt:citedelims}).

\begin{bibexample}[title={Example customisations for \kvopt{introcite}{plain}}]
\begin{lstlisting}[style=extblxstylelatex]
\DeclareFieldFormat{bbx:introcite}{\mkbibbrackets{#1}}
\DeclareDelimFormat[bbx:introcite]{nameyeardelim}{\addcomma\space}
\UndeclareInnerCiteDelim{bbx:introcite}
\renewcommand*{\introcitepunct}{\quad}
\end{lstlisting}
\tcblower
\DeclareFieldFormat{bbx:introcite}{\mkbibbrackets{#1}}
\DeclareDelimFormat[bbx:introcite]{nameyeardelim}{\addcomma\space}
\UndeclareInnerCiteDelim{bbx:introcite}
\renewcommand*{\introcitepunct}{\quad}
\csuse{extblx@opt@dashed@false}
\csletcs{extblx@introcite}{extblx@opt@introcite@plain}
\exampleprintbib{sigfridsson}
\end{bibexample}

\begin{bibexample}[title={Example customisations for \kvopt{introcite}{label}}]
\begin{lstlisting}[style=extblxstylelatex]
\DeclareFieldFormat{bbx:introcite}{\mkbibbold{#1}}
\DeclareDelimcontextAlias{bbx:introcite}{textcite}
\DeclareInnerCiteDelim{bbx:introcite}{\bibopenparen}{\bibcloseparen}
\setlength{\introcitewidth}{0pt}
\setlength{\introcitesep}{\bibhang}
\end{lstlisting}
\tcblower
\DeclareFieldFormat{bbx:introcite}{\mkbibbold{#1}}
\DeclareDelimcontextAlias{bbx:introcite}{textcite}
\DeclareInnerCiteDelim{bbx:introcite}{\bibopenparen}{\bibcloseparen}
\setlength{\introcitewidth}{0pt}
\setlength{\introcitesep}{\bibhang}
\csuse{extblx@opt@dashed@false}
\csletcs{extblx@introcite}{extblx@opt@introcite@label}
\exampleprintbib{coleridge,geer}
\end{bibexample}
\csgundef{blx@delimcontextalias@bbx:introcite}
% because \DeclareDelimcontextAlias is global ... at the moment, this can go
% with biblatex >= 3.12

If you are using an author-year citation style together with
\kvopt{introcite}{label} or \kvopt{introcite}{label}, you may be interested in
combining this with \kvopt{bibstyle}{ext-authortitle} instead to move the year
back to the end of the entry.
\makeatletter
\begin{bibexample}[enhanced, title={\kvopt{introcite}{label} with
  \kvopt{style}{ext-authoryear} and \kvopt{bibstyle}{ext-authortitle}}]
\setlength{\introcitewidth}{0pt}
\setlength{\introcitesep}{\bibhang}
\bbx@opt@mergedate@false
\renewbibmacro*{author}{%
  \ifboolexpr{
    test \ifuseauthor
    and
    not test {\ifnameundef{author}}
  }
    {\usebibmacro{bbx:dashcheck}
       {\bibnamedash}
       {\printnames{author}%
        \setunit{\printdelim{authortypedelim}}%
        \usebibmacro{bbx:savehash}}%
     \usebibmacro{authorstrg}}
    {\global\undef\bbx@lasthash}}
\csuse{extblx@opt@dashed@false}
\csletcs{extblx@introcite}{extblx@opt@introcite@label}
\exampleprintbib{sigfridsson,geer}
\end{bibexample}
\makeatother
\end{optionlist}

\section{Further Customisations}\label{sec:opt:cust}
Aside from the new options mentioned in the last section the styles of this
bundle also offer additional field formats, punctuation and delimiter commands,
a new citation delimiter interface and a few new bibliography macros.

The citation delimiter interface is a novel feature of \sty{biblatex-ext},
but for the other subsections familiarity with the underlying \sty{biblatex}
concepts is assumed.
Some of this is easier understood by looking at the source code directly,
so it might not be a bad idea to have \sty{ext-standard.bbx} open when perusing
this section of the manual.

\subsection{Field formats}\label{sec:opt:field}
In a few places where the standard styles employ hard-coded formatting
directives the styles of this bundle offer customisable formats instead.
Field formats can be modified with \cmd{DeclareFieldFormat}.

\begin{bibfieldformatlist}
\bibfieldformatitem{biblabeldate} The format for the labeldate in the
  bibliography for \sty{authoryear}-like styles. The default is to wrap
  the date in round brackets.
  \begin{bibexample}
  \togglefalse{bbx:doi}
  \DeclareFieldFormat{biblabeldate}{\highlight{\bibopenparen}\highlight[2]{#1}%
    \highlight{\bibcloseparen}}
  \exampleprintbib{sigfridsson}
  \end{bibexample}

\bibfieldformatitem{biblistlabeldate} Like \bibfieldformat{biblabeldate},
  but for bibliography lists created by \cmd{printbiblist}.
  The default is to use the same format as \bibfieldformat{biblabeldate}.

\bibfieldformatitem{issuedate} The format of the \bibfield{issue} and
  \bibfield{date} information for \bibtype{article}s. By default this block is
  wrapped in round brackets.
  \begin{bibexample}
  \makeatletter\bbx@opt@mergedate@false\makeatother
  \togglefalse{bbx:doi}
  \DeclareFieldFormat{issuedate}{\highlight{\bibopenparen}\highlight[2]{#1}%
    \highlight{\bibcloseparen}}
  \exampleprintbib{sigfridsson}
  \end{bibexample}

\bibfieldformatitem{volumeof} The format for the \bibfield{volume} of a
  \bibfield{maintitle} used when \kvopt{maintitleaftertitle}{true}.
  \begin{bibexample}
  \togglefalse{bbx:isbn}\toggletrue{bbx:maintitleaftertitle}
  \DeclareFieldFormat{volumeof}{\highlight{\bibstring{volume}}~%
    \highlight[2]{#1}}
  \exampleprintbib{knuth:ct:a}
  \end{bibexample}

\bibfieldformatitem{titlecase:title} The standard styles follow an
  all-or-nothing approach when it comes to title casing. The field format
  \bibfieldformat{titlecase} intended to enable sentence case with
  \cmd{MakeSentenceCase*} is applied to all title-like fields alike. Finer
  control over the title casing of each field could require involved code.%
  \fnurl{https://tex.stackexchange.com/a/22981/}
  The field format \bibfieldformat{titlecase:title} is applied to the fields
  \bibfield{title} and \bibfield{subtitle}.
  By default this field format is an alias for \bibfieldformat{titlecase}.

\bibfieldformatitem{titlecase:booktitle} Like \bibfieldformat{titlecase:title},
  but controls the title casing of the \bibfield{booktitle} and
  \bibfield{booksubtitle} fields.

\bibfieldformatitem{titlecase:maintitle} Like \bibfieldformat{titlecase:title},
  but controls the title casing of the \bibfield{maintitle} and
  \bibfield{mainsubtitle} fields.

\bibfieldformatitem{titlecase:journaltitle} Like
  \bibfieldformat{titlecase:title}, but controls the title casing of the
  \bibfield{journaltitle} and \bibfield{journalsubtitle} fields.

\bibfieldformatitem{titlecase:issuetitle} Like \bibfieldformat{titlecase:title},
  but controls the title casing of the \bibfield{issuetitle} and
  \bibfield{issuesubtitle} fields.

The \bibfieldformat{citetitle} field format can be used to change the title
case in author-title citations, so there is no
\bibfieldformat{citetitle:labeltitle}.

\begin{bibexample}[title={Default output for \bibfieldformat{titlecase}
  field formats}]%
\DeclareFieldFormat{titlecase:title}{\highlight[1]{#1}}
\DeclareFieldFormat{titlecase:journaltitle}{\highlight[2]{#1}}
\togglefalse{bbx:doi}
\exampleprintbib{shore}
\end{bibexample}

\begin{bibexample}[title={Example changes to \bibfieldformat{titlecase} field
  formats}]%
\begin{lstlisting}[style=extblxstylelatex]
\DeclareFieldFormat{titlecase:title}{|1\MakeSentenceCase*{#1}1|}
\DeclareFieldFormat{titlecase:journaltitle}{|2#12|}
\end{lstlisting}
\tcblower
\DeclareFieldFormat{titlecase:title}{\highlight[1]{\MakeSentenceCase*{#1}}}
\DeclareFieldFormat{titlecase:journaltitle}{\highlight[2]{#1}}
\togglefalse{bbx:doi}
\exampleprintbib{shore}
\end{bibexample}
\end{bibfieldformatlist}

\subsection{Punctuation}\label{sec:opt:punct}
The package provides the following commands to modify the delimiters and
punctuation between fields.
Normal punctuation commands should be redefined with \cmd{renewcommand},
while context-sensitive commands marked with \CSdelim{} should be redefined
with \cmd{DeclareDelimFormat}.
A short overview over common punctuation commands defined by \sty{biblatex-ext}
as well as standard \sty{biblatex} in an example bibliography
can be found in \cref{sec:punctinuse}.
\begin{ltxsyntax}
\csitem{innametitledelim}\CSdelimMark
Similar to \cmd{nametitledelim}, but for names after the \enquote{in:} if
\opt{innamebeforetitle} is \opt{true}.
The default value is that of \cmd{nametitledelim} for all contexts.
Since the definition of \cmd{nametitledelim} is different for the delimiter
contexts \opt{bib} and \opt{biblist}, you may have to use the optional argument
to redefine the delimiter.\footnote{The author admits that it is somewhat
pointless to make \cmd{innametitledelim} context sensitive.
But the obvious parallels with \cmd{nametitledelim} were too tempting.
It is probably too late now.}
\begin{bibexample}
\toggletrue{bbx:innamebeforetitle}
\togglefalse{bbx:isbn}
\DeclareDelimFormat[bib,biblist]{innametitledelim}{\highlight{\textbf{%
  \addperiod}\textvisiblespace}\bibsentence}% <- hacky & hard-coded!
\exampleprintbib{pines}
\end{bibexample}
Since \cmd{nametitledelim} and \cmd{innametitledelim} are independent,
the following output is easily achieved.
Note that the optional argument to \cmd{DeclareDelimFormat} is used
to make sure the definitions apply to the bibliography and bibliography lists
contexts, this is necessary because these contexts have special pre-defined
values that would otherwise not be redefined.
\begin{bibexample}[title={Example customisations for \cs{innametitledelim}}]
\begin{lstlisting}[style=extblxstylelatex]
\ExecuteBibliographyOptions{innamebeforetitle=true}
\DeclareDelimFormat[bib,biblist]{nametitledelim}{|1\addcolon\space1|}
\DeclareDelimFormat[bib,biblist]{innametitledelim}{|2\addcomma\space2|}
\end{lstlisting}
\tcblower
\toggletrue{bbx:innamebeforetitle}
\DeclareDelimFormat[bib,biblist]{nametitledelim}{\highlightbf[1]{\addcolon}%
  \space}
\DeclareDelimFormat[bib,biblist]{innametitledelim}{\highlightbf[2]{\addcomma}%
  \space}
\exampleprintbib{gaonkar,gaonkar:in}
\end{bibexample}

\csitem{maintitletitledelim}
The punctuation between the \bibfield{maintitle} and \bibfield{title} or
\bibfield{booktitle} of a work if \opt{maintitleaftertitle} is \opt{false}.
The default is \cs{newunitpunct}.
\begin{bibexample}
\togglefalse{bbx:isbn}
\renewcommand*{\maintitletitledelim}{\highlight{\textbf{\addperiod}%
  \textvisiblespace}\bibsentence}% <- hacky & hard-coded!
\exampleprintbib{knuth:ct:a}
\end{bibexample}

\csitem{voltitledelim}
The punctuation between the \bibfield{volume} and \bibfield{title} or
\bibfield{booktitle} of a work if \opt{maintitleaftertitle} is \opt{false}.
The default is a colon followed by a space.
\begin{bibexample}
\togglefalse{bbx:isbn}
\renewcommand*{\voltitledelim}{\highlight{\textbf{\addcolon}%
  \textvisiblespace}\bibsentence}
\exampleprintbib{knuth:ct:a}
\end{bibexample}

\csitem{titlemaintitledelim}
The punctuation between the \bibfield{title} or \bibfield{booktitle} and
\bibfield{maintitle} of a work if \opt{maintitleaftertitle} is \opt{true}.
The default is \cmd{newunitpunct}.
\begin{bibexample}
\togglefalse{bbx:isbn}
\toggletrue{bbx:maintitleaftertitle}
\renewcommand*{\titlemaintitledelim}{\highlight{\textbf{\addperiod}%
  \textvisiblespace}\bibsentence}%<- hacky & hard-coded!
\exampleprintbib{knuth:ct:a}
\end{bibexample}

\csitem{jourvoldelim}
The delimiter between the \bibfield{journaltitle} and \bibfield{volume} fields
for \bibtype{article} if no \bibfield{series} is present.
The default is a space.
\begin{bibexample}
\togglefalse{bbx:doi}
\renewcommand*{\jourvoldelim}{\highlight{\textvisiblespace}}
\makeatletter\bbx@opt@mergedate@false\makeatother
\exampleprintbib{sigfridsson}
\end{bibexample}

\csitem{jourserdelim}
The delimiter between the \bibfield{journaltitle} and \bibfield{series} fields
for \bibtype{article}.
The default is \cmd{newunitpunct}.
\begin{bibexample}
\togglefalse{bbx:doi}
\renewcommand*{\jourserdelim}{\highlight{\textbf{\addperiod}\textvisiblespace}%
                              \bibsentence% <- hacky ...
                              }% <- hard-coded!
\makeatletter\bbx@opt@mergedate@false\makeatother
\exampleprintbib{reese,shore}
\end{bibexample}

\csitem{servoldelim}
The delimiter between the \bibfield{series} and \bibfield{volume} fields
for \bibtype{article}.
The default is \cmd{jourvoldelim}.
\begin{bibexample}
\togglefalse{bbx:doi}
\renewcommand*{\servoldelim}{\highlight{\textvisiblespace}}% <- hard-coded!
\makeatletter\bbx@opt@mergedate@false\makeatother
\exampleprintbib{reese,shore}
\end{bibexample}

\csitem{volnumdatedelim}
The delimiter between the \bibfield{volume}, \bibfield{number} block and the
date information for \bibtype{article}.
The default is a space.
\begin{bibexample}
\togglefalse{bbx:doi}
\renewcommand*{\volnumdatedelim}{\highlight{\textvisiblespace}}
\makeatletter\bbx@opt@mergedate@false\makeatother
\exampleprintbib{sigfridsson}
\end{bibexample}


\csitem{volnumdelim}
The delimiter between \bibfield{volume} and \bibfield{number} for
\bibtype{article}.
The default is a dot.
\begin{bibexample}
\togglefalse{bbx:doi}
\renewcommand*{\volnumdelim}{\highlightbf{\adddot}}
\exampleprintbib{sigfridsson}
\end{bibexample}

\csitem{sernumdelim}
The delimiter between \bibfield{series} and \bibfield{number} for
\bibtype{book}- and \bibtype{inbook}-like entries.
The default is a space.
\begin{bibexample}
\renewcommand*{\sernumdelim}{\highlight{\textvisiblespace}}
\exampleprintbib{coleridge}
\end{bibexample}

\csitem{locdatedelim}
The delimiter between \bibfield{location} and \bibfield{date}.
The default is a comma followed by a space.
\begin{bibexample}
\renewcommand*{\locdatedelim}{\highlight{\textbf{\addcomma}\textvisiblespace}}
\makeatletter\bbx@opt@mergedate@false\makeatother
\exampleprintbib{jaffe}
\end{bibexample}

\csitem{locpubdelim}
The delimiter between \bibfield{location} and \bibfield{publisher}\slash%
\bibfield{organization}\slash\bibfield{institution}.
The default is a colon followed by a space.
\begin{bibexample}
\makeatletter\bbx@opt@mergedate@false\makeatother
\renewcommand*{\locpubdelim}{\highlight{\textbf{\addcolon}\textvisiblespace}}
\exampleprintbib{knuth:ct:a}
\end{bibexample}


\csitem{publocdelim}
The delimiter between \bibfield{publisher}\slash\bibfield{organization}\slash
\bibfield{institution} and \bibfield{location}.
The default is a comma followed by a space.
This delimiter is not used by the default style, since the standard order of
fields is \bibfield{location}, \bibfield{publisher}\slash
\bibfield{organization}\slash\bibfield{institution}, \bibfield{date}.
You could use \cs{publocdelim} if you changed the order of these fields to
\bibfield{publisher}\slash\bibfield{organization}\slash\bibfield{institution},
\bibfield{location} \bibfield{date} with
\begin{biblatexcode}
\renewbibmacro*{pubinstorg+location+date}[1]{%
  \printlist{#1}%
  \setunit*{|1\publocdelim1|}%
  \printlist{location}%
  \setunit*{|2\locdatedelim2|}%
  \usebibmacro{date}%
  \newunit}
\end{biblatexcode}
\begin{bibexample}
\renewcommand*{\publocdelim}{\highlight[1]{\textbf{\addcomma}\textvisiblespace}}
\renewcommand*{\locdatedelim}{\highlight[2]{%
  \textbf{\addcomma}\textvisiblespace}}
\renewbibmacro*{pubinstorg+location+date}[1]{%
  \printlist{#1}%
  \setunit*{\publocdelim}%
  \printlist{location}%
  \setunit*{\locdatedelim}%
  \usebibmacro{date}%
  \newunit}
\makeatletter\bbx@opt@mergedate@false\makeatother
\exampleprintbib{knuth:ct:a}
\end{bibexample}


\csitem{pubdatedelim}
The delimiter between \bibfield{publisher}\slash\bibfield{organization}\slash
\bibfield{institution} and \bibfield{date}.
The default is a comma followed by a space.
\begin{bibexample}
\renewcommand*{\pubdatedelim}{\highlight{\textbf{\addcomma}\textvisiblespace}}
\makeatletter\bbx@opt@mergedate@false\makeatother
\exampleprintbib{knuth:ct:a}
\end{bibexample}

\csitem{extradateonlycompcitedelim}\CSdelimMark
Similar to \cmd{compcitedelim}, but indended for use between compressed
citations where the second is an \bibfield{extradate} only.
The default is a comma (\emph{not} followed by a space).
\end{ltxsyntax}
\begin{bibexample}
\DeclareDelimFormat{extradateonlycompcitedelim}{\highlightbf{\addcomma}}
\cite{knuth:ct:b,knuth:ct:c}
\end{bibexample}

\subsection{Delimiters for citation commands}\label{sec:opt:citedelims}
The delimiters for citation commands provided by the styles of this bundle
offer a simple way to customise the bracketing of citation commands.

The citation commands \cmd{cite}, \cmd{parencite} and \cmd{textcite} come with
two sets of delimiters: A pair of \emph{outer delimiters} wrapped around the
resulting citation in its entirety and a pair of \emph{inner delimiters} that
sets off certains bits of the citation label from other information.
An example for outer delimiters would be the round brackets of \cmd{parencite}
for \sty{authoryear}-like styles or the square brackets of \cmd{cite} for
\sty{numeric}- or \sty{alphabetic}-like styles.
Inner delimiters would be the round brackets in \cmd{textcite} around
the year for \sty{authoryear} or around the title for \sty{authortitle}.
See \cref{tab:citationdelims} for more details.

The delimiters are set up to work as paired delimiters, but you are free to
use non-matching pairs or to leave the opening or closing delimiter empty.
If you want to add punctuation, the context-sensitive delimiters
\sty{nameyeardelim}, \sty{nametitledelim} and friends as well as
the context-insensitive \cmd{postnotedelim} and friends are more approriate.
Although the outer delimiters can be set up using \cmd{DeclareCiteCommand}'s
optional \prm{wrapper} argument for most styles, this is not possible for all
styles. Inner delimiters can be set up with \cmd{DeclareFieldFormat} in some
styles, but other styles need more intricate implementations.
This means that the commands discussed here can be used to place the citations
between delimiters, but not natively to pass the result of a citation to a
wrapper command as an argument.

\begin{table}[btph]
\centering
\caption[Outer and inner citation delimiters by style]{\highlight[1]{Outer} and
\highlight[2]{inner} citation delimiters by style. If the style does not use
the delimiters by default, $\langle$ and $\rangle$ are substituted in the
appropriate place.}
\label{tab:citationdelims}
\begin{tabular}{@{}llll@{}}
\toprule
             & \multicolumn{3}{c}{Citation command}\\
             \cmidrule(lr){2-4}
Style family & \cmd{cite} & \cmd{parencite} & \cmd{textcite}\\
\midrule
\sty{alphabetic} & \highlightbf[1]{[}Knu84\highlightbf[1]{]} &
  \highlightbf[1]{[}Knu84\highlightbf[1]{]} & \highlight[1]{$\bm{\langle}$}Knuth
  \highlightbf[2]{[}2\highlightbf[2]{]}\highlight[1]{$\bm{\rangle}$}\\
\sty{numeric} & \highlightbf[1]{[}2\highlightbf[1]{]} &
  \highlightbf[1]{[}2\highlightbf[1]{]} & \highlight[1]{$\bm{\langle}$}Knuth
  \highlightbf[2]{[}2\highlightbf[2]{]}\highlight[1]{$\bm{\rangle}$}\\
\sty{authortitle} & \highlight[1]{$\bm{\langle}$}Knuth,
  \highlight[2]{$\bm{\langle}$}\emph{\TeX book}\highlight[2]{$\bm{\rangle}$}%
  \highlight[1]{$\bm{\rangle}$} & \highlightbf[1]{(}Knuth,
  \highlight[2]{$\bm{\langle}$}\emph{\TeX book}\highlight[2]{$\bm{\rangle}$}%
  \highlightbf[1]{)} & \highlight[1]{$\bm{\langle}$}Knuth \highlightbf[2]{(}%
  \emph{\TeX book}\highlightbf[2]{)}\highlight[1]{$\bm{\rangle}$}\\
\sty{authoryear} & \highlight[1]{$\bm{\langle}$}Knuth
  \highlight[2]{$\bm{\langle}$}1984\highlight[2]{$\bm{\rangle}$}%
  \highlight[1]{$\bm{\rangle}$} & \highlightbf[1]{(}Knuth
  \highlight[2]{$\bm{\langle}$}1984\highlight[2]{$\bm{\rangle}$}%
  \highlightbf[1]{)} & \highlight[1]{$\bm{\langle}$}Knuth
  \highlightbf[2]{(}1984\highlightbf[2]{)}\highlight[1]{$\bm{\rangle}$}\\
\bottomrule
\end{tabular}
\end{table}

\begin{ltxsyntax}
\cmditem{DeclareOuterCiteDelim}{cite command}{opening delimiter}
        {closing delimiter}

Sets up the outer delimiters for the citation command
\cmd{$\langle$\emph{cite command}$\rangle$}. The name of the \prm{cite command}
is given without leading backslash in the argument, it normally corresponds to
the delimiter context.

You may use almost any input for \prm{opening delimiter} and
\prm{closing delimiter} as long as typesetting of
\enquote{\prm{opening delimiter}text\prm{closing delimiter}} does not lead to
errors when arbitrary grouping such as
\enquote{\{\prm{opening delimiter}\allowbreak text\}\allowbreak
\prm{closing delimiter}}
or \enquote{\prm{opening delimiter}\allowbreak text\allowbreak
\{\prm{closing delimiter}\}}
is introduced.
It can not be guaranteed that the opening and closing delimiters are executed
at the same level of grouping, let alone in the same group.
As mentioned above, this approach is not suitable to wrap the citation up in a
wrapper command, i.e.\ to pass the entire output of the citation command as
argument to a macro.

Instead of hard-coded \texttt{(}, \texttt{)}, \texttt{[} and \texttt{]} their
\sty{biblatex} counterparts \cmd{bibopenparen}, \cmd{bibcloseparen},
\cmd{bibopenbracket} and \cmd{bibclosebracket} are preferable, since these
commands respond to nesting and check if opening brackets are always closed.

\cmditem{DeclareOuterCiteDelimAlias}{cite alias}{cite command}
\cmditem*{DeclareOuterCiteDelimAlias*}{cite alias}{cite command}

Use the outer delimiters of \cmd{$\langle$\emph{cite command}$\rangle$} for
\cmd{$\langle$\emph{cite alias}$\rangle$} as well.
The unstarred version uses \cmd{def} assignment while the starred version uses
\cmd{let}. This means that the starred version copies the values of the
definitions at the time of executing the aliasing command,
whereas the alias created by the unstarred version will only evaluate the
delimiters whenever the citation command is called.

\cmditem{UndeclareOuterCiteDelim}{cite command}

Completely remove the definitions of the outer delimiters for
\cmd{$\langle$\emph{cite command}$\rangle$}.

\cmditem{DeclareInnerCiteDelim}{cite command}{opening delimiter}
        {closing delimiter}

Sets up the inner delimiters for the citation command
\cmd{$\langle$\emph{cite command}$\rangle$}.

This command is similar to \cmd{DeclareOuterCiteDelim} and the same
restrictions for the arguments apply.

\cmditem{DeclareInnerCiteDelimAlias}{cite alias}{cite command}
\cmditem*{DeclareInnerCiteDelimAlias*}{cite alias}{cite command}

Use the inner delimiters of \cmd{$\langle$\emph{cite command}$\rangle$} for
\cmd{$\langle$\emph{cite alias}$\rangle$} as well.
The unstarred version uses \cmd{def} assignment while the starred version uses
\cmd{let}. This means that the starred version copies the values of the
definitions at the time of executing the aliasing command,
whereas the alias created by the unstarred version will only evaluate the
delimiters whenever the citation command is called.

\cmditem{UndeclareInnerCiteDelim}{cite command}

Completely remove the definitions of the inner delimiters for
\cmd{$\langle$\emph{cite command}$\rangle$}.

The \sty{authoryear} and \sty{authortitle} styles have \cmd{parencite},
e.g.\ \parencite{knuth:ct:a}, \parencite{sigfridsson}, set up with
\begin{biblatexcode}
\DeclareOuterCiteDelim{parencite}{\bibopenparen}{\bibcloseparen}
\DeclareInnerCiteDelim{parencite}{}{}
\end{biblatexcode}
and \cmd{textcite}, e.g.\ \textcite{knuth:ct:a}, \textcite{sigfridsson}, with
\begin{biblatexcode}
\DeclareOuterCiteDelim{textcite}{}{}
\DeclareInnerCiteDelim{textcite}{\bibopenparen}{\bibcloseparen}
\end{biblatexcode}
If you wanted \cmd{parencite} of \sty{authoryear} to look like
\enquote{[Sigfridsson and Worman (1998)]} you would use
\begin{biblatexcode}
\DeclareOuterCiteDelim{parencite}{|1\bibopenbracket1|}{|1\bibclosebracket1|}
\DeclareInnerCiteDelim{parencite}{|2\bibopenparen2|}{|2\bibcloseparen2|}
\end{biblatexcode}
\citereset
\begin{bibexample}
\DeclareOuterCiteDelim{parencite}{\highlightbf[1]{\bibopenbracket}}{%
  \highlightbf[1]{\bibclosebracket}}
\DeclareInnerCiteDelim{parencite}{\highlightbf[2]{\bibopenparen}}{%
  \highlightbf[2]{\bibcloseparen}}
\parencite{sigfridsson}\quad\parencite{worman,geer}\quad
\parencite{knuth:ct:a,knuth:ct:b,knuth:ct:c}
\end{bibexample}
\end{ltxsyntax}

\subsection{Selected bibliography macros}\label{sec:opt:bibmacros}
The following macros are defined in \texttt{ext-standard.bbx} and may make
certain things easier to customise.
Many of these macros are replacements for bare \cmd{printfield} or
\cmd{printlist} in the bibliography drivers, or pack a frequently-used
sequence of commands into one central place.

\begin{keymarglistbook}
\keyitem[barevolume+volumes]
A bibliography macro to print the \bibfield{volume}, \bibfield{part} and
\bibfield{volumes} fields for \bibtype{mvbook}-, \bibtype{book}- and
\bibtype{inbook}-like entry types.
If \bibfield{maintitle} is defined, the \bibfield{volume} and \bibfield{part}
fields will be printed by \bibmacro{maintitle+title} or
\bibmacro{maintitle+booktitle} instead.

\begin{bibexample}
\renewbibmacro*{barevolume+volumes}{%
  \printtext[highlight1]{%
    \iffieldundef{maintitle}
      {\printfield{volume}%
       \printfield{part}}
      {}%
    \newunit
    \printfield{volumes}}}
\exampleprintbib{knuth:ct,matuz:doody}
\end{bibexample}

\keyitem[edition]
A bibliography macro to print the \bibfield{edition} field, this avoids a direct
\lstinline|\printfield{edition}| in the bibliography drivers.

\keyitem[in:editor(+others)]
The bibliography macro to print the \bibfield{editor} before the
\bibfield{booktitle} for \bibtype{inbook}, \bibtype{incollection} and
\bibtype{inproceedings} when \opt{innamebeforetitle} is set to \opt{true}.
The \cmd{printname} uses the name format \bibfield{ineditor}.

\keyitem[language]
A bibliography macro to print the \bibfield{language} field, this avoids a
direct \lstinline|\printlist{language}| in the bibliography drivers.

\keyitem[note]
A bibliography macro to print the \bibfield{note} field, this avoids a direct
\lstinline|\printfield{note}| in the bibliography drivers.

\keyitem[pubinstorg+location+date]
A general-purpose bibliography macro to catch
\bibmacro{publisher+location+date},
\bibmacro{institution+location+date} and \bibmacro{organization+location+date}.
This bibliography macro has one mandatory argument: the name of a list field,
sensible values are \bibfield{publisher}, \bibfield{institution} and
\bibfield{organization}.

\begin{biblatexcode}
\newbibmacro*{pubinstorg+location+date}[1]{%
  \printlist{location}%
  \iflistundef{|#1|}
    {\setunit*{\locdatedelim}}
    {\setunit*{\locpubdelim}}%
  \printlist{|#1|}%
  \setunit*{\pubdatedelim}%
  \usebibmacro{date}%
  \newunit}

\renewbibmacro*{|publisher|+location+date}{%
  \usebibmacro{pubinstorg+location+date}{|publisher|}}

\renewbibmacro*{|institution|+location+date}{%
  \usebibmacro{pubinstorg+location+date}{|institution|}}

\renewbibmacro*{|organization|+location+date}{%
  \usebibmacro{pubinstorg+location+date}{|organization|}}
\end{biblatexcode}

\keyitem[type+number]
A bibliography macro to print the \bibfield{type} and \bibfield{number} fields.
\end{keymarglistbook}

\section{Revision History}\label{sec:log}
The GitHub repository of this project uses release tags, so you can compare
the changes in source code there.\footnote{\url{\gitbaseurl/compare/}}
See also \sty{CHANGES.md}.
\begin{changelog}
\begin{release}{0.3}{2018-06-04}
\item Added \bibfieldformat{titlecase:\dots title} field formats%
  \see{sec:opt:field}
\item Added \sty{bbx:introcite:plain:keeprelated} toggle%
  \see{sec:opt:style}
\item Added \cs{jourvoldelim}, \cs{jourserdelim} and \cs{servoldelim}%
  \see{sec:opt:punct}
\item Improve documentation
\end{release}
\begin{release}{0.2}{2018-03-28}
\item Rework lengths for \kvopt{introcite}{label}\see{sec:opt:style}
\item Fixed meaning of \cs{maintitletitledelim}\see{sec:opt:punct}
\item Added \cs{voltitledelim}\see{sec:opt:punct}
\end{release}
\begin{release}{0.1a}{2018-03-20}
\item Fixed inner citation delimiters for \sty{ext-authoryear},
  \sty{ext-authortitle} and their \sty{-ibid} versions\see{sec:opt:citedelims}
\item Fixed \cmd{smartcite} delimiters\see{sec:opt:citedelims}
\end{release}
\begin{release}{0.1}{2018-03-18}
\item First public release
\end{release}
\end{changelog}

\begin{bibexample}[breakable]
\printbibliography[title={Example References}]
\end{bibexample}

\clearpage
\appendix
\section{Punctuation Commands in Use}\label{sec:punctinuse}
The following example bibliography shows some common punctuation commands
in use.
Commands provided by standard \sty{biblatex} are marked in
\textit{\cmd{italics}}, new commands defined by \sty{biblatex-ext} are in
\textbf{\cmd{bold}}.
\begin{bibexample}[title={Punctuation and delimiters
  (defined by \textit{standard \sty{biblatex}} and
  \textbf{\sty{biblatex-ext}})},
  top=6mm,bottom=4mm,remember,
  overlay={
    % geer
    \punctarrow*{nameyeardelim}{-.8mm}{6mm}
    \punctarrow*{nametitledelim}{2.5mm}{6mm}
    % companion
    \punctarrow*{multinamedelim}{-2.5mm}{6mm}
    \punctarrow*{finalnamdelim}{2.5mm}{4mm}
    \punctarrow{locpubdelim}{-2.5mm}{-5mm}
    \punctarrow{pubdatedelim}{2.5mm}{-4mm}
    % knuth:ct:a
    \punctarrow{maintitletitledelim}{-2.5mm}{-.8cm}
    \punctarrow{voltitledelim}{2.5mm}{-4mm}
    % moore
    \punctarrow*{intitlepunct}{-4mm}{-4mm}
    \punctarrow*{bibpagespunct}{-2.5mm}{-4mm}
    \punctarrow*{bibrangedash}{.2cm}{-4mm}
    % sigfridsson
    \punctarrow{jourvoldelim}{-.2cm}{-4mm}
    \punctarrow{volnumdelim}{.2cm}{-4mm}
    % vizedom:related
    \punctarrow*{translatortypedelim}{2.5mm}{6mm}
    \punctarrow*{begrelateddelim}{.8cm}{-4mm}
    % westfahl:frontier
    \punctarrow*{editortypedelim}{2.5mm}{6mm}
    \punctarrow*{newunitpunct}{-.2cm}{-4mm}
    % westfahl:space
    \punctarrow*{bibnamedash}{2.5mm}{5mm}
    \punctarrow*{subtitlepunct}{2.5mm}{5mm}
  }
]
\frenchspacing
\list{}
  {\setlength{\leftmargin}{\bibhang}%
   \setlength{\itemindent}{-\leftmargin}%
   \setlength{\itemsep}{8mm}%
   \setlength{\parsep}{\bibparsep}}
\item\nocite{geer}
  Geer, Ingrid de\tikztextmark{nameyeardelim}{\textvisiblespace}(1985)%
  \tikztextmark{nametitledelim}{.} \enquote{Earl, Saint, Bishop, Skald --
  and Music. The Orkney Earldom of the Twelfth Century. A Musicological Study}.
  PhD thesis. Uppsala: Uppsala Universitet.

\item\nocite{companion}
  Goossens, Michel\tikztextmark{multinamedelim}{,} Frank Mittelbach
  \tikztextmark{finalnamdelim}{and} Alexander Samarin (1994).
  \emph{The LaTeX Companion}. 1st ed. Reading, Mass.%
  \tikztextmark{locpubdelim}{:} Addison-Wesley\tikztextmark{pubdatedelim}{,}
  1994. 528 pp.

\item\nocite{knuth:ct:a}
  Knuth, Donald E. (1984). \emph{Computers \& Typesetting}%
  \tikztextmark{maintitletitledelim}{.} Vol. A\tikztextmark{voltitledelim}{:}
  \emph{The \TeX{} book}. Reading, Mass.: Addison-Wesley, 1984.

\item\nocite{moore}
  Moore, Gordon E. (1965). \enquote{Cramming more components onto integrated
  circuits}. In\tikztextmark{intitlepunct}{:} \emph{Electronics} 38.8%
  \tikztextmark{bibpagespunct}{,} pp. 114\tikztextmark{bibrangedash}{--}117.

\item\nocite{sigfridsson}
  Sigfridsson, Emma and Ulf Ryde (1998). \enquote{Comparison of methods for
  deriving atomic charges from the electrostatic potential and moments}.
  In: \emph{Journal of Computational
  Chemistry}\tikztextmark{jourvoldelim}{\textvisiblespace}19%
  \tikztextmark{volnumdelim}{.}4 (1998), pp. 377--395.

\item\nocite{vizedom:related}
  Vizedom, Monika B. and Gabrielle L. Cafee%
  \tikztextmark{translatortypedelim}{,} trans. (1960).
  \emph{The Rites of Passage.} University of Chicago Press%
  \tikztextmark{begrelateddelim}{.} Trans. of Arnold van Gennep.
  \emph{Les rites de passage.}
  Paris: Nourry, 1909.

\item\nocite{westfahl:frontier}
  Westfahl, Gary\tikztextmark{editortypedelim}{,} ed. (2000a).
  \emph{Space and Beyond. The Frontier Theme in Science Fiction}%
  \tikztextmark{newunitpunct}{.} Westport, Conn. and London: Greenwood, 2000.

\item\nocite{westfahl:space}
  \tikztextmark{bibnamedash}{--} (2000b). \enquote{The True Frontier%
  \tikztextmark{subtitlepunct}{.} Confronting and Avoiding the Realities of
  Space in American Science Fiction Films}. In: \emph{Space and Beyond.
  The Frontier Theme in Science Fiction}. Ed. by Gary Westfahl. Westport, Conn.
  and London: Greenwood, 2000, pp. 55--65.
\endlist
\end{bibexample}

\end{document}
\endinput
