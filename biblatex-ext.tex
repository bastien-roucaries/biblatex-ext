\documentclass[DIV=9]{scrartcl}
\usepackage{ifxetex,ifluatex}
\newif\iffontspec
\ifxetex
  \fontspectrue
\else\ifluatex
  \fontspectrue
\else
  \fontspecfalse
\fi\fi
\iffontspec
  \usepackage{fontspec}
\else
  \usepackage[T1]{fontenc}
  \usepackage[utf8]{inputenc}
  \input glyphtounicode
  \pdfgentounicode=1
\fi
\usepackage[french,ngerman,british]{babel}
\usepackage{csquotes}
\usepackage[useregional]{datetime2}
\iffontspec\else
  \usepackage{lmodern}
\fi
\usepackage[mono=false]{libertinus}
\iffontspec
  \setmonofont[Scale=0.78]{Bitstream Vera Sans Mono}
  \usepackage{unicode-math}
  \setmathfont[Scale=MatchUppercase]{libertinusmath-regular.otf}
\else
  \usepackage[libertine]{newtxmath}
  \usepackage[scaled=0.78]{beramono}
  \usepackage{bm}
\fi
\usepackage{microtype}
\usepackage{hyphenat}
\usepackage{booktabs}
\usepackage{multicol}
\usepackage[svgnames]{xcolor}
\usepackage{longtable}
\usepackage{array}
\newcolumntype{L}[1]{%
  >{\raggedright\let\newline\\\arraybackslash\hspace{0pt}}p{#1}}
\usepackage[listings, breakable, skins]{tcolorbox}%
\usetikzlibrary{arrows.meta}
\usetikzlibrary{shapes.geometric}
\usepackage[style=ext-authoryear-iecomp, backend=biber]{biblatex}
\addbibresource{biblatex-examples.bib}
\addbibresource{biblatex-ext-examples.bib}
\usepackage{biblatex-ext-tabular}
\usepackage{biblatex-ext-oa}
% Normally, the following packages should not be loaded explicitly.
% Instead, one of them (and *only* one) should be loaded via
% biblatex-ext-oa's options.
\usepackage{biblatex-ext-oasymb-tikz}
% We need to undefine \oasymbol to be able to load the other packages as well.
\undef\oasymbol
\undef\DefineOASymbol
\usepackage{biblatex-ext-oasymb-l3draw}
% We need to undefine \oasymbol to be able to load the other packages as well.
\undef\oasymbol
\undef\DefineOASymbol
\usepackage{biblatex-ext-oasymb-pict2e}

\makeatletter
\defbibcheck{examplebib}{%
  \xifinlist{\thefield{entrykey}}{\extblxdoc@examplebib@list}
    {}
    {\skipentry}}

\newcommand*{\exampleprintbib}[1]{%
  \nocite{#1}%
  \let\extblxdoc@examplebib@list\empty
  \def\do##1{\listeadd\extblxdoc@examplebib@list{\detokenize{##1}}}%
  \docsvlist{#1}%
  \printbibliography[check=examplebib, heading=none]}

\newcommand*{\exampleprintbibtab}[1]{%
  \nocite{#1}%
  \let\extblxdoc@examplebib@list\empty
  \def\do##1{\listeadd\extblxdoc@examplebib@list{\detokenize{##1}}}%
  \docsvlist{#1}%
  \printbibtabular[check=examplebib, heading=none]}


\usepackage{ltxdockit}
\usepackage{btxdockit}
\usepackage{cleveref}
\hypersetup{%
  colorlinks=true,
  allcolors=spot,
  bookmarksopen=false,
  bookmarksnumbered=false,
  plainpages=false}

\definecolor{highlight1}{RGB}{240, 0, 0}
\definecolor{highlight2}{RGB}{0,153, 153}
\definecolor{spot}{rgb}{0,0.2,0.6}

\lstdefinestyle{extblxstylegeneral}{%
  aboveskip    = {0\p@ \@plus 6\p@},
  belowskip    = {0\p@ \@plus 6\p@},
  tabsize      = 2,
  breaklines   = true,
  breakatwhitespace = true,
  keepspaces   = true,
  escapeinside = {(*@}{@*)},
  moredelim    = {[is][\ttfamily\bfseries\color{highlight1}]{|}{|}},
  moredelim    = {[is][\ttfamily\bfseries\color{highlight1}]{|1}{1|}},
  moredelim    = {[is][\ttfamily\bfseries\color{highlight2}]{|2}{2|}},
}

\lstdefinelanguage{extBibTeX}{%
  morekeywords=[1]{%
    @article,@artwork,@audio,@bibnote,@book,@bookinbook,@booklet,%
    @collection,@commentary,@customa,@customb,@customc,@customd,%
    @custome,@customf,@inbook,@incollection,@inproceedings,%
    @inreference,@image,@jurisdiction,@legal,@legislation,@letter,%
    @dataset,
    @manual,@misc,@movie,@music,@mvcollection,@mvreference,%
    @mvproceedings,@mvbook,@online,@patent,@performance,@periodical,%
    @proceedings,@reference,@report,@review,@set,@software,@standard,%
    @suppbook,@suppcollection,@suppperiodical,@thesis,@unpublished,@video%
   },
   morekeywords=[2]{author,title,date,journal,volume,number,pages,doi,
                    eprint,eprinttype,eprintclass},
   keywordstyle=[1]{\bfseries\spotcolor},
   keywordstyle=[2]{\spotcolor},
   sensitive=false,
}

\lstdefinestyle{extblxstylelatex}{%
  language  = {[LaTeX]TeX},
  style     = {extblxstylegeneral},
  moretexcs = {
    dimexpr,arraybackslash,newcolumntype,anchor,driver,plain,plainlang,
    anchorlang,
    thefield,
    setlength,bibhang,addcomma,adddot,addperiod,addcolon,addspace,
    addnbspace,
    mkbibbold,mkbibemph,mkbibbrackets,mkbibparens,
    usebibmacro,newbibmacro,renewbibmacro,setunit,newunit,printfield,printlist,
    bibopenparen,bibcloseparen,bibopenbracket,bibclosebracket,
    iflistundef,iffieldundef,ifnameundef,
    ExecuteBibliographyOptions,
    defbibtabular,defbibtabulartwocolumn,defbibenvironment,
    DeclareFieldFormat,DeclareDelimFormat,DeclareDelimcontextAlias,
    DeclareInnerCiteDelims,UndeclareInnerCiteDelims,DeclareInnerCiteDelimsAlias,
    DeclareOuterCiteDelims,UndeclareOuterCiteDelims,DeclareOuterCiteDelimsAlias,
    introcitepunct,volnumdelim,maintitletitledelim,voltitledelim,sernumdelim,
    volnumdatedelim,locdatedelim,locpubdelim,publocdelim,pubdatedelim,
    extradateonlycompcitedelim,introcitesep,introcitewidth,introcitesep,
    DeclareOpenAccessFieldUrl,DeclareOpenAccessEprintUrl,
    DeclareOpenAccessEprintAlias,DeclareOpenAccessUrlFieldPriority},
}

\lstdefinestyle{extblxstylebibtex}{%
  language  = {extBibTeX},
  style     = {extblxstylegeneral},
}


\newcommand*{\highlight}[2][1]{\textcolor{highlight#1}{#2}}
\newcommand*{\highlightbf}[2][1]{\textcolor{highlight#1}{\textbf{#2}}}
\DeclareFieldFormat{highlight1}{\textcolor{highlight1}{#1}}
\DeclareFieldFormat{highlight2}{\textcolor{highlight2}{#1}}

\newtcolorbox{bibexample}[1][]{enhanced,
                               colframe=spot!75!black, colback=spot!5!white,
                               #1}
\newtcblisting{biblatexcode}{enhanced,
                             colframe=black!75!white, colback=black!5!white,
                             listing only,
                             frame hidden,
                             breakable,
                             listing style=extblxstylelatex}
\newtcblisting{bibtexfile}{enhanced,
                           colframe=black!75!white, colback=black!5!white,
                           listing only,
                           frame hidden,
                           breakable,
                           listing style = extblxstylebibtex}

\newtcbinputlisting{\inputexamplebibfile}[2][]{%
  listing file={#2},
  enhanced,
  colframe=black!75!white, colback=black!5!white,
  listing only,
  frame hidden,
  breakable,
  listing style = extblxstylebibtex,
  #1}

\newtcolorbox{warnbox}[1][]{%
  enhanced,
  colframe=red, colback=red!5!white,
  underlay={%
    \path[draw=none] (interior.south east)
                     rectangle node{\warnsymbol}
                     ([xshift=15mm,yshift=.1cm]interior.north east);},
  #1}

\newtcolorbox{remindbox}[1][]{%
  enhanced,
  colframe=yellow, colback=yellow!5!white,
  underlay={%
    \path[draw=none] (interior.south east)
                     rectangle node{\dbendsymbol}
                     ([xshift=15mm,yshift=.1cm]interior.north east);},
  #1}

\AtUsedriver*{%
  \let\newblock\relax
  \let\abx@macro@bibindex\@empty
  \let\abx@macro@pageref\@empty}


% this is taken from ltxdockit.cls, which is not loaded
\newrobustcmd*{\fnurl}[1][]{\hyper@normalise\ltd@fnurl{#1}}
\def\ltd@fnurl#1#2{\footnote{#1\hyper@linkurl{\Hurl{#2}}{#2}}}

\newrobustcmd*{\email}{\hyper@normalise\ltd@email}
\def\ltd@email#1{\href{mailto:#1}{#1}}

% title
\providecommand*{\titlepage}{}
\providecommand*{\titlefont}{}
\renewrobustcmd*{\titlepage}[1]{\setkeys{ltd@ttp}{#1}}
\renewcommand*{\titlefont}{\bfseries}
\define@key{ltd@ttp}{title}{\def\ltd@title@title{#1}}
\define@key{ltd@ttp}{subtitle}{\def\ltd@title@subtitle{#1}}
\define@key{ltd@ttp}{url}{\def\ltd@title@url{#1}}
\define@key{ltd@ttp}{author}{\def\ltd@title@author{#1}}
\define@key{ltd@ttp}{link}{\def\ltd@title@link{#1}}
\define@key{ltd@ttp}{revision}{\def\ltd@title@revision{#1}}
\define@key{ltd@ttp}{date}{\def\ltd@title@date{#1}}

\providecommand*{\printtitlepage}{}
\renewrobustcmd*{\printtitlepage}{%
  \begingroup
  \centering\titlefont
  \begingroup\LARGE
    \ifundef\ltd@title@url
      {\ltd@title@title}
      {\href{\ltd@title@url}{\ltd@title@title}}%
    \par
  \endgroup
  \vspace{0.25\baselineskip}
  \begingroup\large
    \ltd@title@subtitle\par
  \endgroup
  \expandafter\url\expandafter{\ltd@title@url}\par
  \begin{multicols}{2}
  \raggedleft
    \ltd@title@author\par
    \expandafter\email\expandafter{\ltd@title@link}\par
  \raggedright
    Version \ltd@title@revision\par\ltd@title@date
  \end{multicols}
  \endgroup}

\BeforeStartingTOC[toc]{\begin{multicols}{2}}
\AfterStartingTOC[toc]{\end{multicols}}
\BeforeTOCHead[toc]{\pdfbookmark[1]{\contentsname}{contents}}

\newrobustcmd*{\tex}{\TeX}
\newrobustcmd*{\etex}{\mbox{e-TeX}}
\newrobustcmd*{\pdftex}{pdf\-\tex}
\newrobustcmd*{\xetex}{Xe\-\tex}
\newrobustcmd*{\luatex}{Lua\-\tex}
\newrobustcmd*{\latex}{\LaTeX}%{La\kern-0.07em TeX}
\newrobustcmd*{\pdflatex}{pdf\-\latex}
\newrobustcmd*{\xelatex}{Xe\-\latex}
\newrobustcmd*{\lualatex}{Lua\-\latex}
\newrobustcmd*{\miktex}{Mik\-\tex}
\newrobustcmd*{\texlive}{\tex~live}
\newrobustcmd*{\bibtex}{Bib\kern-0.07em TeX}
\newrobustcmd*{\lppl}{\latex{} Project Public License}
\newrobustcmd*{\pdf}{\acr{PDF}}
\newrobustcmd*{\utf}{\mbox{\acr{UTF}-8}}
\newrobustcmd*{\pgftikz}{PGF/Ti\emph{k}Z}

\pdfstringdefDisableCommands{%
  \def\tex{TeX}%
  \def\etex{e-TeX}%
  \def\xetex{XeTeX}%
  \def\latex{LaTeX}%
  \def\xelatex{XeLaTeX}%
  \def\bibtex{BibTeX}%
  \def\lppl{LaTeX Project Public License}%
  \def\pdf{PDF}%
  \def\utf{UTF-8}%
}

\let\accentcolour\spotcolor

\addtokomafont{section}{\accentcolour}
\addtokomafont{subsection}{\accentcolour}
\addtokomafont{subsubsection}{\accentcolour}

\renewcommand*{\verbatimfont}{\ttfamily}
\renewcommand*{\displayverbfont}{\ttfamily}
\renewcommand*{\marglistfont}{\accentcolour\sffamily\small}
\renewcommand*{\margnotefont}{\sffamily\small}
\renewcommand*{\optionlistfont}{\accentcolour\sffamily\displayverbfont}
\renewcommand*{\ltxsyntaxfont}{\ttfamily}
\renewcommand*{\ltxsyntaxlabelfont}{\accentcolour\displayverbfont}
\renewcommand*{\changelogfont}{\normalfont}
\renewcommand*{\changeloglabelfont}{\accentcolour\sffamily\bfseries}
\newcommand*{\stylelistlabelfont}{\accentcolour\sffamily\small}
\newcommand*{\bibfieldformatfont}{\sffamily}
\newcommand*{\bibfieldformatlabelfont}{\accentcolour\bibfieldformatfont\small}

\newenvironment*{stylelist}
  {\list{}{%
     \setlength{\labelwidth}{\marglistwidth}%
     \setlength{\labelsep}{\marglistsep}%
     \setlength{\leftmargin}{0pt}%
     \renewcommand*{\makelabel}[1]{\hss\stylelistlabelfont##1}}%
   \def\styleitem##1{%
     \item[{##1}]%
     \ltd@pdfbookmark{##1}{##1}}}
  {\endlist}

\newenvironment*{bibfieldformatlist}
  {\list{}{%
     \setlength{\labelwidth}{\marglistwidth}%
     \setlength{\labelsep}{\marglistsep}%
     \setlength{\leftmargin}{0pt}%
     \renewcommand*{\makelabel}[1]{\hss\bibfieldformatlabelfont##1}}%
   \def\bibfieldformatitem##1{%
     \item[{##1}]%
     \ltd@pdfbookmark{##1}{##1}}}
  {\endlist}

\newenvironment*{keymarglistbook}
  {\marglist
   \setlength{\itemsep}{0pt}%
   \raggedright
   \let\@@item\item
   \def\keyitem[##1]{%
     \@@item[{##1}]%
     \ltd@pdfbookmark{##1}{##1}}}
  {\endmarglist}

% modified for correct measurements
\def\ltd@option@i#1#2#3#4[#5]{%
  \item[#2]%
  \ltd@pdfbookmark{#1}{#1}%
  \begingroup\raggedright
  \ltd@textverb{=}%
  \settowidth\@tempdimb{\ltd@textverb{=}}%
  \settowidth\@tempdimc{\optionlistfont#2}%
  \ifdim\@tempdimc>\marglistwidth
    \@tempdimc=\dimexpr\@tempdimc-\marglistwidth\relax
  \else
    \@tempdimc=0pt
  \fi
  \@tempdima=\dimexpr\linewidth-\@tempdimb-\@tempdimc\relax
  \ifblank{#4}
    {}
    {\settowidth\@tempdimb{default: #4}%
     \@tempdima=\dimexpr\@tempdima-\@tempdimb-2em\relax}%
  \parbox[t]{\@tempdima}{\raggedright #3}%
  \ifblank{#4}
    {}
    {\hfill default:~#4}%
  \ifblank{#5}
    {}
    {\marginpar{\margnotefont #5}}%
  \par\endgroup
  \nobreak\vspace{\itemsep}}

% inject a label additional to the bookmark so we can link stuff
\let\exltd@pdfbookmark\ltd@pdfbookmark
\def\ltd@pdfbookmark#1#2{%
  \phantomsection\label{exltd@itm@#1}%
  \exltd@pdfbookmark{#1}{#2}}

\blx@inputonce{ext-biblatex-aux.def}{auxiliary code for ext-biblatex}
  {}{}{}{}
\newcommand*{\biblatexversion}{\extblx@requiredbiblatexversion}
\def\exltd@isofydate#1/#2/#3{#1-#2-#3}
\newcommand*{\biblatexdate}{\extblx@requiredbiblatexdate}
\expandafter\def\expandafter\biblatexdate\expandafter{%
  \expandafter\exltd@isofydate\extblx@requiredbiblatexdate}

\AtEndPreamble{%
  \deflength{\marglistwidth}{(\oddsidemargin+2cm)*85/100}}

\newcommand*{\tikztextmark}[2]{%
  \tikz[remember picture,baseline,inner sep=0pt]\node [anchor=base] (#1) {#2};}

% *{<cmd>}{<x-shift>}{<y-shift>}
% unstarred version for commands defined by biblatex-ext
% starred version for standard biblatex commands
\def\punctarrow{%
  \@ifstar\punctarrow@ii\punctarrow@i}

\def\punctarrow@i{%
  \def\punctarrow@cmdfont{\bfseries}%
  \def\punctarrow@hyperref##1##2{%
    \hyperref[exltd@itm@##1]{##2}%
  }%
  \punctarrow@iii
}

\def\punctarrow@ii{
  \def\punctarrow@cmdfont{\itshape}%
  \let\punctarrow@hyperref\@secondoftwo
  \punctarrow@iii
}

\def\punctarrow@iii#1#2#3{%
  \ifdimcomp{#2}{<}{0pt}
    {\def\punctarrow@nodeanchor{east}}
    {\def\punctarrow@nodeanchor{west}}%
  \draw[spot,thick,latex-,rounded corners] (#1) |- ++ (#2,#3)
    node[anchor=\punctarrow@nodeanchor,text = black]
    {\punctarrow@hyperref{#1}{\punctarrow@cmdfont\cmd{#1}}};%
}

% By Stefan Kottwitz, see https://tex.stackexchange.com/a/799/35864
\newcommand*\justify{%
  \fontdimen2\font=0.4em% interword space
  \fontdimen3\font=0.2em% interword stretch
  \fontdimen4\font=0.1em% interword shrink
  \fontdimen7\font=0.1em% extra space
  \hyphenchar\font=`\-% allowing hyphenation
}

% no \mbox here, we may have to break things
\renewrobustcmd*{\sty}[1]{{\verbatimfont\justify #1}}
\newrobustcmd*{\blxstyle}[1]{{\verbatimfont\justify #1}}
\newrobustcmd*{\filename}[1]{{\verbatimfont\justify #1}}
\renewrobustcmd*{\bibfield}[1]{\sty{#1}}
\renewrobustcmd*{\opt}[1]{\sty{#1}}
\newrobustcmd*{\optval}[1]{\sty{#1}}
\newrobustcmd*{\bibmacro}[1]{\sty{#1}}
\renewrobustcmd*{\bibtype}[1]{\sty{@#1}}
\renewrobustcmd*{\cmd}[1]{\sty{\textbackslash #1}}
\let\cs\cmd
\newrobustcmd*{\bibfieldformat}[1]{{\bibfieldformatfont #1}}

\newrobustcmd*{\gencode}[1]{{\verbatimfont\justify #1}}

\def\exltd@buildhypercmd@i#1{%
  \expandafter\newrobustcmd\expandafter*\expandafter
    {\csname hyper#1@i\endcsname}[2][]{%
    \hyperref[exltd@itm@##1]{\csname #1\endcsname{##2}}}
  \csdef{hyper#1}{\@dblarg{\csname hyper#1@i\endcsname}}}

\def\exltd@buildhypercmd{\forcsvlist{\exltd@buildhypercmd@i}}

\exltd@buildhypercmd{cmd,opt,bibmacro,len,blxstyle,sty,bibfield,bibfieldformat,
  filename}

\newrobustcmd*{\hyperkvopt}[2]{{%
  \verbatimfont\hyperref[exltd@itm@#1]{#1}\penalty\@M
  \hskip 0em plus 0.15em\relax
  =\penalty\hyphenpenalty
  \hskip 0em plus 0.15em\relax #2}}

\newcommand*{\ctan}{\mkbibacro{CTAN}}
\newcommand*{\gitbaseurl}{https://github.com/moewew/biblatex-ext}
\newcommand*{\extblxversion}{0.8}
\newcommand*{\biber}{Biber}
\newcommand*{\gitissuelink}[1]{%
  \href{\gitbaseurl/issues/#1}{issue \##1 on github}}

\newrobustcmd*{\CSdelim}{%
  \textcolor{spot}{\margnotefont\footnotesize context sensitive}}
\newrobustcmd*{\CSdelimMark}{%
  \leavevmode\marginpar{\CSdelim}}

\newcommand*{\mpdl}{$\langle$}
\newcommand*{\mpdr}{$\rangle$}

\iffontspec
  \newcommand*{\hmpdl}{$\langle$}
  \newcommand*{\hmpdr}{$\rangle$}
\else
  \newcommand*{\hmpdl}{$\bm{\langle}$}
  \newcommand*{\hmpdr}{$\bm{\rangle}$}
\fi

\def\textvisiblespace{%
  \raisebox{-2.2pt}{%
    \mbox{\kern.04em\vrule \@height.5ex \@width.12ex}%
    \vbox{\hrule \@width.2em \@height.12ex}%
    \hbox{\vrule \@height.5ex \@width.12ex}%
    \kern.04em}}

% Bourbaki dangerous bend symbol by Heiko Oberdiek
% https://tex.stackexchange.com/users/16967/heiko-oberdiek
% https://tex.stackexchange.com/a/262510/35864
\newcommand*{\dbendsymbol@tikz}{%
  \begin{tikzpicture}[
    line cap=but,
    line join=round,
    x=1.2em,
    line width=2pt,
    y=2*(height("Z")-\pgflinewidth)*(1-sin(10)),
    rotate=-10,
    rounded corners=1.5pt,]
    \draw (.5,.5) node[scale=2,draw,diamond,fill=yellow,color=yellow] {};
    \draw (1, 0) -- (0, 0) -- (1, 1) -- (0, 1);
  \end{tikzpicture}}

\newsavebox{\dbendsymbol@box}
\sbox{\dbendsymbol@box}{\dbendsymbol@tikz}
\newcommand*{\dbendsymbol}{\usebox{\dbendsymbol@box}}

\newcommand{\warnsymbol}{%
  \makebox[2em][c]{%
    \makebox[0pt][c]{\raisebox{.26em}{\LARGE\bfseries!}}%
    \makebox[0pt][c]{%
      \fontsize{40}{40}\selectfont\color{red}$\bigtriangleup$}}}

\newcommand*{\marginwarn}{%
  \leavevmode\marginpar{\raisebox{-1.5em}{\warnsymbol}}}

\makeatother

\titlepage{%
  title    = {The \sty{biblatex-ext} Bundle},
  subtitle = {Extensions for the \sty{biblatex} standard styles},
  url      = {\gitbaseurl},
  author   = {Moritz Wemheuer},
  link     = {mwemheu@gmail.com},%
  revision = {\extblxversion},
  date     = {\DTMDate{2019-08-25}},
}

\hypersetup{%
  pdftitle    = {The biblatex-ext Bundle},
  pdfsubject  = {Extensions for the \sty{biblatex} standard styles},
  pdfauthor   = {Moritz Wemheuer},
  pdfkeywords = {latex, biblatex, bibtex, bibliography, references, citation},
}

\hyphenation{%
  star-red
  un-star-red
  bib-lio-gra-phy
  white-space
  bib-open-paren
  bib-close-paren
  bib-open-bracket
  bib-close-bracket
  main-title-after-title
  jour-vol-delim
  in-name-before-title
  tikz-picture
  Define-OA-Symbol
}


\begin{document}

\printtitlepage
\tableofcontents

\section{Introduction}\label{sec:int}
\subsection{About}
The \sty{biblatex-ext} bundle provides an extended version of the standard
styles that come with \sty{biblatex}.
For each standard style this bundle provides a style with the same name
prefixed with \blxstyle{ext-} which can be used as a drop-in replacement for the
standard style~-- for example, the replacement for \blxstyle{authoryear-icomp}
is called \hyperblxstyle{ext-authoryear-icomp}.

The aim of the styles of this bundle is to offer a simple interface to change
some of the stylistic decisions made for the standard styles that would
otherwise need cumbersome and tedious redefinitions.
Additionally, some customisation features that were not deemed appropriate
for inclusion in the \sty{biblatex} kernel are provided.
Other than that the styles are as close to the standard styles as possible.
They do not attempt to offer options or commands for customisations that
are already fairly simple to achieve with the tools provided by the standard
styles.\footnote{Please be gentle and allow for a lot of wiggle room for what
exactly \enquote{simple} means. And don't get mad if the principle is not
followed at all times and the occasional solution for things that already are
\enquote{simple enough} pop up here and there.}

The initial motivation for this bundle was what has now become the option
\hyperopt{innamebeforetitle}.
It is fairly straighforward to print the name of the editor of an
\bibtype{incollection} before the \bibfield{booktitle}.
But if one does not want to resort to clever tricks,%
\footnote{See \url{https://tex.stackexchange.com/q/122218/} and
\url{https://tex.stackexchange.com/q/173638/} for example.}
the modifications needed to do this in a stable, safe and clean manner by
redefining the bibliography drivers can easily amass hundred lines of code~--
code you may not want to see in your preamble.
Some contributed \sty{biblatex} styles already place the editor in the
desired position, but you may not want to commit to the other changes implied
by switching to one of those styles.
Especially styles written for the sole purpose of implementing the requirements
of a particular style guide may have to go to great lengths to do so and are
therefore not as easily modified as the standard styles.
The styles of this bundle, on the other hand, try to stay as close to the
standard styles as possible both in output and implementation to allow you
to customise the styles with minimum additional effort over the standard styles.


A few words of warning:
The styles of this bundle are only really useful if you want to use one of their
features to avoid having to go through the lengthy and tedious redefinitions the
standard styles would require.
Before you get involved in modifying the standard styles or one of the styles
of this bundle, you may want to have a look at the host of other styles
available for \sty{biblatex},\fnurl{https://www.ctan.org/topic/biblatex}
maybe you are lucky and the style you are looking for has already been
implemented by someone else.
As was the intention, most methods to modify the standard styles are also
applicable to the styles of this bundle, but you may break some of their
features if you happen to modify something that the \blxstyle{ext-} styles
redefine themselves.
While many contributed \sty{biblatex} styles are examples of good \sty{biblatex}
style coding, this can not be said of all of the files included in this bundle.
Especially the citation styles for compact citations have to work harder
to implement the citation delimiter feature properly.
So if you want to get inspired for your \sty{biblatex} coding, be warned that
terrible things lurk in the \filename{.cbx} files.
The standard \filename{.cbx} files will give you a much better impression of how
things should be done.


\subsection{Requirements}\label{sec:req}
The use of the styles requires a current version of the \sty{biblatex} package.
At the time of writing the latest version of \sty{biblatex} is
\biblatexversion{} (dated \biblatexdate),
that version is required for the styles to work properly.
A warning will be issued if you use an older version.
You may choose to ignore that warning, but the styles can not be guaranteed to
work properly in that case:
you might get other warnings or errors, and some features of the style might
just fail silently.

Use of the \biber{} backend is strongly encouraged.
Most of the new features of this bundle do not require \biber{} explicitly in
their implementation, but many \sty{biblatex} standard features only work
properly with \biber{}.

\subsection{Compatibility with earlier versions}\label{sec:backw}
In an ideal world all changes to the styles in this bundle would be fully
backwards compatible. Unfortunately we do not live in this ideal world.
Most changes to this bundle should introduce new features that have little
relevant impact on existing functionality. But the occasional change
may have more lasting consequences and may cause existing customisations to
fail.
In fact all changes have the potential to break someone's code, since
\sty{biblatex} styles have little to no distinction between public and internal
code.
Still, some changes are more likely to be problematic than others and those
should be kept to a minimum. They can, however, be required to move development
forward.

If possible, backwards compatiblity measures that allow old code to
continue to work are in place. Sometimes that is not feasible and in those
cases the next best solution is to warn if old code is used, so please check
the log file for warnings~-- especially after an update.
But at times not even that is possible, so it is always a good idea to consult
the change history after an update and to examine the output with extra
scrutiny.
Higher-level changes are found in the revision history in \cref{sec:log},
more technical changes and commentary can be found in \filename{CHANGES.md},
the full set of changes can be investigated by comparing the tagged and
publicly available source code.

\subsection{Installation}\label{sec:install}
This style bundle is available on \ctan.%
\fnurl{https://ctan.org/pkg/biblatex-ext}
The current release is also available in \miktex{} and \texlive{} 2018 or
newer.
If at all possible you should install this bundle via your \tex{} distribution
(using \miktex{} Console\fnurl{https://miktex.org/howto/miktex-console} or
\gencode{tlmgr} for \texlive
\fnurl{https://www.tug.org/texlive/pkginstall.html}).
If you must install this package manually, get the files from \ctan{}
and install the \filename{.bbx}, \filename{.cbx}, \filename{.def},
\filename{.sty} and \filename{.lua} files preferably to
\path{tex/latex/biblatex-ext} of your local or home \TeX{} tree%
\fnurl{https://texfaq.org/FAQ-inst-wlcf}, the other
files (\path{CHANGES.md}, \path{README.md}, \path{biblatex-ext.tex},
\path{biblatex-ext.pdf} and \path{biblatex-ext-examples.bib}) go into
\path{doc/latex/biblatex-ext}. You may have to refresh your file name database
afterwards, so \tex{} can find the files.

\subsection{License}\label{sec:license}
Permission is granted to copy, distribute and\slash or modify this software
under the terms of the \lppl, version 1.3c%
\fnurl{https://www.latex-project.org/lppl/lppl-1-3c.txt}
or (at your option) any later version%
\fnurl{https://www.latex-project.org/lppl.txt}.
This bundle is maintained by Moritz Wemheuer (\textcopyright 2017--2019).


\subsection{Feedback}\label{sec:feedback}
You can use the \sty{biblatex-ext} project page on GitHub%
\footnote{\url{\gitbaseurl}} to report bugs and
submit suggestions and feature requests, or you can do so via email.

If you do not want to report a bug or request a feature, but are simply in need
of assistance, you might want to consider posting your question on the
\texttt{comp.text.tex} newsgroup or \tex{}~--~\latex{} Stack Exchange.%
\fnurl{https://tex.stackexchange.com/questions/tagged/biblatex}
If after a consultation on one of these sites it turns out you have happened
upon a bug or that you can formulate a request for a new feature that could be
useful, please head over to GitHub or send me an email.

\section{Use}\label{sec:use}
The \sty{biblatex-ext} bundle is a collection of \sty{biblatex} style files.
You can load the styles exactly as you would load the standard styles:
\begin{biblatexcode}
\usepackage[style=(*@\prm{style}@*)]{biblatex}
\end{biblatexcode}
The naming of the styles follows the scheme
\mbox{\blxstyle{ext-}\prm{standard style}}, e.g.,
the style corresponding to \blxstyle{authoryear-icomp} is called
\hyperblxstyle{ext-authoryear-icomp}.

This manual assumes familiarity with the concepts of \sty{biblatex} and does
not attempt to explain any of the standard \sty{biblatex} features, so you may
want to have the \sty{biblatex} documentation%
\fnurl{http://mirrors.ctan.org/macros/latex/contrib/biblatex/doc/biblatex.pdf}
at hand to fully appreciate what is going on.
If you are new to \sty{biblatex} the three hundred odd pages of the manual
can be hard to stomach at first, so you may want to have a look at a more
gentle introduction.
Of the many introductory texts and tutorials available on the internet%
\fnurl{https://tex.stackexchange.com/q/13509/35864}
the author particularly likes
Knut Hegna and Dag Langmyhr's \emph{Local Guide to \sty{biblatex}}%
\fnurl{http://dag.at.ifi.uio.no/public/doc/biblatex-guide.pdf}
and Paul Stanley's \emph{\sty{biblatex}~-- An Easier Read}%
\fnurl{https://github.com/PaulStanley/biblatex-tutorial/releases}.
French speakers may want to consider Maïeul Rouquette's
\foreignlanguage{french}{\emph{(Xe)\LaTeX{} Appliqué aux sciences humaines}}
available on \ctan\fnurl{https://ctan.org/pkg/latex-sciences-humaines}.
If you read German you may be interested in Dominik Waßenhoven's two-part
series \foreignlanguage{ngerman}{\emph{Bibliographien erstellen mit
\sty{biblatex}}} in \foreignlanguage{ngerman}{\emph{Die \TeX nische Komödie}}
2/2008\fnurl{https://archiv.dante.de/DTK/PDF/komoedie_2008_2.pdf} (pp.~53--75)
and 4/2008\fnurl{https://archiv.dante.de/DTK/PDF/komoedie_2008_4.pdf} (pp.~31--51).
Please keep in mind that some of these texts were written a while ago and
that \sty{biblatex} is actively developed: technical details may have changed
and new features make some things easier.

\clearpage
\section{Styles}\label{sec:styles}
This bundle provides an extended version of each standard style as well as a
few new styles.
\subsection{Standard styles}\label{sec:styles:standard}
Please refer to the \sty{biblatex} documentation%
\fnurl{http://mirrors.ctan.org/macros/latex/contrib/biblatex/doc/biblatex.pdf}
and the style examples%
\fnurl{http://mirrors.ctan.org/macros/latex/contrib/biblatex/doc/examples/}
for a more detailed description of each standard style.
The relations between the styles are exactly as in their standard counterparts.
The \blxstyle{ext-} styles only build on top of the standard files.
\begin{stylelist}
\styleitem{ext-numeric}
An extended version of the standard \blxstyle{numeric} style for citations with
numeric labels.
\begin{bibexample}[]
[1, 2, 5, 6, 7]
\end{bibexample}

\styleitem{ext-numeric-comp}
An extended version of the standard \blxstyle{numeric-comp} style.
Similar to \blxstyle{ext-numeric}, but citation labels are
compressed to give \enquote{[1--3]} instead of \enquote{[1, 2, 3]}.
\begin{bibexample}[]
[1, 2, 5--7]
\end{bibexample}

\styleitem{ext-numeric-verb}
An extended version of the standard \blxstyle{numeric-verb} style.
This style is similar to the \blxstyle{numeric} style, but each citation label
is in its own set of brackets: \enquote{[1], [2], [3]}.
\begin{bibexample}[]
[1], [3], [5], [6], [7]
\end{bibexample}

\styleitem{ext-alphabetic}
An extended version of the standard \blxstyle{alphabetic} style for citations
with alphabetic labels derived from author name and year.
\begin{bibexample}[]
[SR98, Knu86c]
\end{bibexample}

\styleitem{ext-alphabetic-verb}
An extended version of the standard \blxstyle{alphabetic-verb} style.
This style is based on \blxstyle{ext-alphabetic}, but like
\blxstyle{ext-numeric-verb} places each citation label in its own pair of
brackets: \enquote{[SR98], [Knu86c]}.
\begin{bibexample}[]
[SR98], [Knu86c]
\end{bibexample}

\styleitem{ext-authoryear}
An extended version of the standard \blxstyle{authoryear} style for citations
using author name and year.
\begin{bibexample}
Sigfridsson and Ryde 1998
\end{bibexample}

\styleitem{ext-authoryear-comp}
An extended version of the standard \blxstyle{authoryear-comp} style.
The style is based on the author-year citations of \blxstyle{ext-authoryear},
but several works by the same author are compressed by not displaying the
author multiple times: \enquote{Knuth 1984, 1986} instead of
\enquote{Knuth 1984; Knuth 1986}.
\begin{bibexample}
Knuth 1984, 1986
\end{bibexample}

\styleitem{ext-authoryear-ibid}
An extended version of the standard \blxstyle{authoryear-ibid} style.
This style is similar to \blxstyle{ext-authoryear}, but repeated citations are
replaced with \enquote{ibidem}.
\begin{bibexample}
Knuth 1984\quad ibid.
\end{bibexample}

\styleitem{ext-authoryear-icomp}
An extended version of the standard \blxstyle{authoryear-icomp} style.
This style combines the two styles \blxstyle{ext-authoryear-comp} and
\blxstyle{ext-authoryear-ibid}.
\begin{bibexample}
Knuth 1984, 1986 \quad Sigfridsson and Ryde 1998 \quad ibid.
\end{bibexample}

\styleitem{ext-authortitle}
An extended version of the standard \blxstyle{authortitle} style for citations
by author and title.
\begin{bibexample}
Maron, \emph{Animal Triste}
\end{bibexample}

\styleitem{ext-authortitle-comp}
An extended version of the standard \blxstyle{authortitle-comp} style.
This style is based on \blxstyle{ext-authortitle} and compresses several
citations by the same author just like \blxstyle{ext-author\-year-comp}:
\enquote{Aristotle, \emph{Physics}, \emph{Poetics}} instead of
\enquote{Aristotle, \emph{Physics}; Aristotle, \emph{Poetics}}.
\begin{bibexample}
Aristotle, \emph{Physics}, \emph{Poetics}
\end{bibexample}

\styleitem{ext-authortitle-ibid}
An extended version of the standard \blxstyle{authortitle-ibid} style.
This style is similar to \blxstyle{ext-authortitle}, but replaces repeated
citations of the same work with \enquote{ibidem}.
\begin{bibexample}
Maron, \emph{Animal Triste}\quad ibid.
\end{bibexample}

\styleitem{ext-authortitle-icomp}
An extended version of the standard \blxstyle{authortitle-icomp} style.
This style combines \blxstyle{ext-authortitle-comp} and
\blxstyle{ext-authortitle-ibid}.
\begin{bibexample}
Aristotle, \emph{Physics}, \emph{Poetics}\quad Maron, \emph{Animal Triste}\quad
ibid.
\end{bibexample}

\styleitem{ext-authortitle-terse}
An extended version of the standard \blxstyle{authortitle-terse} style.
This style is similar to \blxstyle{ext-authortitle}, but the title is omitted
in citations if there is only one work by the relevant author.
\begin{bibexample}
Sigfridsson and Ryde \quad Aristotle, \emph{Physics}; Aristotle, \emph{Poetics}
\end{bibexample}

\styleitem{ext-authortitle-tcomp}
An extended version of the standard \blxstyle{authortitle-tcomp} style.
This style combines \blxstyle{ext-authortitle-terse} and
\blxstyle{ext-authortitle-comp}.
\begin{bibexample}
Sigfridsson and Ryde \quad Aristotle, \emph{Physics}, \emph{Poetics}
\end{bibexample}

\styleitem{ext-authortitle-ticomp}
An extended version of the standard \blxstyle{authortitle-ticomp} style.
This style combines \blxstyle{ext-authortitle-terse},
\blxstyle{ext-authortitle-comp} and \blxstyle{ext-authortitle-ibid}.
\begin{bibexample}
Aristotle, \emph{Physics}, \emph{Poetics} \quad Sigfridsson and Ryde \quad ibid.
\end{bibexample}

\styleitem{ext-verbose}
An extended version of the standard \blxstyle{verbose} style.
This style shows the full bibliographic reference the first time a work is
cited.

\styleitem{ext-verbose-ibid}
An extended version of the standard \blxstyle{verbose-ibid} style.
Based on \blxstyle{ext-verbose}, repeated citations to the same work are
replaced with \enquote{ibidem}.

\styleitem{ext-verbose-note}
An extended version of the standard \blxstyle{verbose-note} style.
Based on \blxstyle{ext-verbose} and intended for use in footnotes, subsequent
citations link back to the footnote the entry was cited at first and in full.

\styleitem{ext-verbose-inote}
An extended version of the standard \blxstyle{verbose-inote} style.
Similar to \blxstyle{verbose-note}, but repeated citations to the same work are
replaced with \enquote{ibidem}.

\styleitem{ext-verbose-trad1}
An extended version of the standard \blxstyle{verbose-trad1} style.
This style makes extensive use of scholarly abbreviations and is otherwise
similar to \blxstyle{ext-verbose}.

\styleitem{ext-verbose-trad2}
An extended version of the standard \blxstyle{verbose-trad2} style.
The style is similar to \blxstyle{ext-verbose-inote} and
uses scholarly abbreviations to shorten citations.

\styleitem{ext-verbose-trad3}
An extended version of the standard \blxstyle{verbose-trad3} style.
This style is similar to \blxstyle{ext-verbose-trad2}.
\end{stylelist}

\subsection{New styles}\label{sec:styles:new}
The following styles are not mere extensions of a particular standard style,
instead they implement new combinations or extensions of the concepts available
in the standard styles.
\begin{stylelist}
\styleitem{ext-authoryear-ecomp}
An author-year citation that is even more compact than
\blxstyle{ext-authoryear-comp}, it compresses lists of \bibfield{extradate}
letters from \enquote{Knuth 1986a,b,c} to \enquote{Knuth 1986a\bibrangedash c}.
\begin{bibexample}
Sigfridsson and Ryde 1998 \quad Knuth 1986a\bibrangedash c
\end{bibexample}

\styleitem{ext-authoryear-iecomp}
An author-year citation that is even more compact than
\blxstyle{ext-authoryear-icomp}, it compresses lists of \bibfield{extradate}
letters from \enquote{Knuth 1986a,b,c} to \enquote{Knuth 1986a\bibrangedash c}.
\begin{bibexample}
Sigfridsson and Ryde 1998 \quad ibid.\quad Knuth 1986a\bibrangedash c
\end{bibexample}

\styleitem{ext-authoryear-terse}
An author-year citation style that suppresses years for author lists with only
one work in the bibliography.
This style is like \blxstyle{ext-authortitle-terse}, but it is based on
\blxstyle{ext-authoryear} and not on \blxstyle{ext-authortitle}.
\begin{bibexample}
Sigfridsson and Ryde \quad Knuth 1984 \quad Knuth 1986
\end{bibexample}

\styleitem{ext-authoryear-tcomp}
A compact author-year citation style that suppresses years for author lists
with only one work in the bibliography.
This style is like \blxstyle{ext-authortitle-tcomp}, but it is based on
\blxstyle{ext-authoryear} and not on \blxstyle{ext-authortitle}.
\begin{bibexample}
Sigfridsson and Ryde \quad Knuth 1984, 1986
\end{bibexample}

\styleitem{ext-authoryear-tecomp}
A compact author-year citation style that combines
\blxstyle{ext-authoryear-tcomp} and \blxstyle{ext-authoryear-ecomp}.
\begin{bibexample}
Sigfridsson and Ryde \quad Knuth 1984, 1986a\bibrangedash c
\end{bibexample}

\styleitem{ext-authoryear-ticomp}
A compact author-year citation style with \enquote{ibidem} function that
suppresses years for author lists with only one work in the bibliography.
This style is like \blxstyle{ext-authortitle-ticomp}, but it is based on
\blxstyle{ext-authoryear} and not on \blxstyle{ext-authortitle}.
\begin{bibexample}
Sigfridsson and Ryde \quad ibid. \quad Knuth 1984, 1986
\end{bibexample}

\styleitem{ext-authoryear-teicomp}
A compact author-year citation style that combines
\blxstyle{ext-authoryear-ticomp} and \blxstyle{ext-authoryear-iecomp}.
\begin{bibexample}
Sigfridsson and Ryde \quad ibid. \quad Knuth 1984, 1986a\bibrangedash c
\end{bibexample}

\end{stylelist}

\clearpage
\section{Options}\label{sec:opt}
All options of the \sty{biblatex} package are supported and each style supports
the options of its standard counterpart.

\subsection{General options}\label{sec:opt:gen}
Additionally, all styles support the following options in global, per-type and
per-entry scope.
The default values are such that the styles can be used as drop-in replacement
for the standard files without significant changes in output.

\begin{optionlist}
\boolitem[true]{articlein}
Whether or not to display \enquote{in:} before the journal information in
\bibtype{article} entries.
All other entry types are not affected by this option.
If it is desired to remove the \enquote{in:} for more entry types or
a more specific behaviour is required, then it is still going to be
necessary to modify the bibmacro~\bibmacro{in:}.

\begingroup
\makeatletter
\togglefalse{bbx:doi}
\DeclareFieldFormat{highlighthere}{%
  \iffieldequalstr{entrykey}{sigfridsson}
    {\textcolor{highlight1}{#1}}
    {\iffieldequalstr{entrykey}{westfahl:space}
       {\textcolor{highlight2}{#1}}
       {#1}}}
\renewbibmacro*{in:}{%
  \blx@begunit\blx@endunit
  \printtext[highlighthere]{%
    \printtext{%
      \bibstring{in}\intitlepunct}}}

\begin{bibexample}[title={\kvopt{articlein}{true}}]
\toggletrue{bbx:articlein}
\exampleprintbib{sigfridsson,westfahl:space}
\end{bibexample}

\begin{bibexample}[title={\kvopt{articlein}{false}}]
\togglefalse{bbx:articlein}
\exampleprintbib{sigfridsson,westfahl:space}
\end{bibexample}
\makeatother
\endgroup

\boolitem[false]{citexref}
This option controls if \bibtype{inbook}, \bibtype{incollection} and
\bibtype{inproceedings} entries that are tied to a parent entry with
\bibfield{xref} or \bibfield{crossref} should cite their parent in the
bibliography \emph{if} the parent is listed in the bibliography as a
separate entry.
With the default setting \kvopt{citexref}{false} the parent is not
cited, the entry is shown as in the standard styles.
If the option is set to \optval{true}, the block following the
\enquote{in:} is replaced by a citation to the parent entry.

\begin{remindbox}
The option itself does \emph{not} cause the parent entry to be added
to the bibliography automatically.
This needs to happen either explicitly by citing the parent
(possibly with \cmd{nocite})
or implicitly via the \opt{minxrefs} or \opt{mincrossrefs} option.
\end{remindbox}

\nocite{westfahl:frontier}
\begin{bibexample}[title={\kvopt{citexref}{true}}]
\makeatletter
\renewbibmacro*{crosscite}[1]{%
  \printtext[highlight1]{%
    \iftoggle{bbx:citexref}
      {\iffieldundef{crossref}
         {\iffieldundef{xref}
            {\usebibmacro{#1}}
            {\printtext{\bbx@xrefcite{\thefield{xref}}}}}
         {\printtext{\bbx@xrefcite{\thefield{crossref}}}}}
      {\usebibmacro{#1}}}}
\makeatother
\toggletrue{bbx:citexref}
\exampleprintbib{westfahl:space}
\end{bibexample}

\begin{bibexample}[title={\kvopt{citexref}{false}}]
\makeatletter
\renewbibmacro*{crosscite}[1]{%
  \printtext[highlight1]{%
    \iftoggle{bbx:citexref}
      {\iffieldundef{crossref}
         {\iffieldundef{xref}
            {\usebibmacro{#1}}
            {\printtext{\bbx@xrefcite{\thefield{xref}}}}}
         {\printtext{\bbx@xrefcite{\thefield{crossref}}}}}
      {\usebibmacro{#1}}}}
\makeatother
\togglefalse{bbx:citexref}
\exampleprintbib{westfahl:space}
\end{bibexample}

\boolitem[false]{innamebeforetitle}
Whether or not the \bibfield{editor} should be moved before the
\bibfield{booktitle} field for \bibtype{inbook}, \bibtype{incollection} and
\bibtype{inproceedings} entries.

\begin{bibexample}[title={\kvopt{innamebeforetitle}{true}}]
\makeatletter
\renewbibmacro*{bbx:in:editor}[1]{%
  \ifboolexpr{
    test \ifuseeditor
    and
    not test {\ifnameundef{editor}}
  }
    {\printtext[highlight1]{%
       \ifboolexpr{togl {bbx:innameidem} and test {\bbx@ineditoridem}}
         {\bibstring[\mkibid]{idem\thefield{gender}}}
         {\printnames[ineditor]{editor}}%
       \setunit{\printdelim{editortypedelim}}%
       \usebibmacro{#1}}%
     \clearname{editor}}
    {}}
\makeatother
\toggletrue{bbx:innamebeforetitle}
\exampleprintbib{pines}
\end{bibexample}

\begin{bibexample}[title={\kvopt{innamebeforetitle}{false}}]
\renewbibmacro*{byeditor+others}{%
  \ifnameundef{editor}
    {}
    {\printtext[highlight1]{%
       \usebibmacro{byeditor+othersstrg}%
       \setunit{\addspace}%
       \printnames[byeditor]{editor}%
       \newunit}
     \clearname{editor}}%
  \usebibmacro{byeditorx}%
  \usebibmacro{bytranslator+others}}
\togglefalse{bbx:innamebeforetitle}
\exampleprintbib{pines}
\end{bibexample}

\boolitem[false]{innameidem}
Whether or not the \bibfield{editor} of the \bibfield{booktitle}
for the entry types \bibtype{inbook}, \bibtype{incollection} and
\bibtype{inproceedings} is replaced by \enquote{idem} in case the
\bibfield{editor} and \bibfield{author} name lists coincide.
This option only has an effect if \hyperopt{innamebeforetitle} is set to
\optval{true}.

\begin{bibexample}[title={\kvopt{innameidem}{true}}
  {\small (and \kvopt{innamebeforetitle}{true})}]
\togglefalse{bbx:isbn}
\makeatletter
\renewbibmacro*{bbx:in:editor}[1]{%
  \ifboolexpr{
    test \ifuseeditor
    and
    not test {\ifnameundef{editor}}
  }
    {\printtext[highlight1]{%
       \ifboolexpr{togl {bbx:innameidem} and test {\bbx@ineditoridem}}
         {\bibstring[\mkibid]{idem\thefield{gender}}}
         {\printnames[ineditor]{editor}}%
       \setunit{\printdelim{editortypedelim}}%
       \usebibmacro{#1}%
       \clearname{editor}}}
    {}}
\makeatother
\toggletrue{bbx:innamebeforetitle}\toggletrue{bbx:innameidem}
\exampleprintbib{gaonkar:in}
\end{bibexample}

\begin{bibexample}[title={\kvopt{innameidem}{false}}
  {\small (and \kvopt{innamebeforetitle}{true})}]
\togglefalse{bbx:isbn}
\makeatletter
\renewbibmacro*{bbx:in:editor}[1]{%
  \ifboolexpr{
    test \ifuseeditor
    and
    not test {\ifnameundef{editor}}
  }
    {\printtext[highlight1]{%
       \ifboolexpr{togl {bbx:innameidem} and test {\bbx@ineditoridem}}
         {\bibstring[\mkibid]{idem\thefield{gender}}}
         {\printnames[ineditor]{editor}}%
       \setunit{\printdelim{editortypedelim}}%
       \usebibmacro{#1}%
       \clearname{editor}}}
    {}}
\makeatother
\toggletrue{bbx:innamebeforetitle}\togglefalse{bbx:innameidem}
\exampleprintbib{gaonkar:in}
\end{bibexample}


\boolitem[false]{maintitleaftertitle}
Whether or not the \bibfield{maintitle} is printed after the
\bibfield{title} or \bibfield{booktitle} of the work.
If \opt{maintitleaftertitle} is \optval{true}, the \bibfield{volume} field will
be printed with the \hyperbibfieldformat{volumeof} format.

\begingroup
\makeatletter
\renewbibmacro*{maintitle+title}{%
  \blx@begunit\blx@endunit
   \iftoggle{bbx:maintitleaftertitle}
     {}
     {\iffieldsequal{maintitle}{title}
        {\clearfield{maintitle}%
         \clearfield{mainsubtitle}%
         \clearfield{maintitleaddon}}
        {\printtext[highlight1]{%
           \iffieldundef{maintitle}
             {}
             {\usebibmacro{maintitle}%
              \newunit\newblock
              \iffieldundef{volume}
                {}
                {\printfield{volume}%
                 \printfield{part}%
                 \setunit{\maintitletitledelim}}}}}}%
  \printtext[highlight2]{\usebibmacro{title}\blx@begunit\blx@endunit}%
  \printunit{}%
   \iftoggle{bbx:maintitleaftertitle}
     {\iffieldsequal{maintitle}{title}
        {\clearfield{maintitle}%
         \clearfield{mainsubtitle}%
         \clearfield{maintitleaddon}}
        {\iffieldundef{maintitle}
           {}
           {\setunit{\titlemaintitledelim}%
            \printtext[highlight1]{%
              \iffieldundef{volume}
                {}
                {\printfield[volumeof]{volume}%
                 \printfield{part}%
                 \setunit{\addspace}%
                 \bibstring{ofseries}%
                 \setunit{\addspace}}%
              \usebibmacro{maintitle}}}}
       {}}%
  \newunit}%
\renewcommand*{\maintitletitledelim}{\highlight{\addcolon\space}}
\makeatother
\begin{bibexample}[title={\kvopt{maintitleaftertitle}{true}}]
\togglefalse{bbx:isbn}
\toggletrue{bbx:maintitleaftertitle}
\exampleprintbib{knuth:ct:a}
\end{bibexample}

\begin{bibexample}[title={\kvopt{maintitleaftertitle}{false}}]
\togglefalse{bbx:isbn}
\togglefalse{bbx:maintitleaftertitle}
\exampleprintbib{knuth:ct:a}
\end{bibexample}
\endgroup
\end{optionlist}

\subsection{Style-specific options}\label{sec:opt:style}
\begin{optionlist}
\optitem[true]{dashed}{\optval{true}, \optval{false}, \optval{fullhash},
                       \optval{bibnamehash}}

The \opt{dashed} option of the \hyperblxstyle[ext-authoryear]{authoryear}- and
\hyperblxstyle[ext-authortitle]{authortitle}-like bibliography styles allows for
finer control over the dashes than in the standard styles.
The option is available globally and additionally on a per-type and per-entry
level.

This option controls whether or not recurring lists of authors/editors in the
bibliography are replaced with a dash.
The standard values \optval{true} and \optval{false} are still valid and give
the exact same output as in the standard styles.
This means that the output is fully compatible with the standard styles.
The new values \optval{fullhash} and \optval{bibnamehash} differ in how exactly
they determine if a list of authors/editors is the same as the previous.
\begin{valuelist}
\item[true] An alias for \optval{fullhash}.
\item[false] Disable this feature.
\item[bibnamehash] Replace recurring name lists with a dash.
                   Compare name lists using \bibfield{bib\-name\-hash}, taking
                   into account only names that are actually listed in the
                   bibliography account.
\item[fullhash] Replace recurring name lists with a dash.
                Compare name lists using \bibfield{fullhash}, taking into
                account all names in the list, even those that are truncated
                and do not appear in the bibliography.
\end{valuelist}


\begin{refsection}
\makeatletter
Assuming \kvopt{maxnames}{1} and no name list disambiguation
(\kvopt{uniquelist}{false}), the four entries
\inputexamplebibfile[listing options={linerange={1-16,18-23,25-25},
                                      style = extblxstylebibtex}]
                    {biblatex-ext-examples.bib}
give
\renewcommand*{\bibnamedash}{\textbf{\textemdash\addspace}}
\DeclareFieldFormat{highlighthere}{%
  \iffieldequalstr{entrykey}{elk:einio}
    {\highlight[1]{#1}}
    {\iffieldequalstr{entrykey}{appleby:abl}
       {\highlight[2]{#1}}
       {#1}}}
\renewbibmacro*{author}{%
  \printtext[highlighthere]{%
  \ifboolexpr{
    test \ifuseauthor
    and
    not test {\ifnameundef{author}}
  }
    {\usebibmacro{bbx:dashcheck}
       {\printtext{\bibnamedash}}
       {\usebibmacro{bbx:savehash}%
        \printnames{author}}%
        \iffieldundef{authortype}
          {\setunit{\printdelim{nameyeardelim}}}
          {\setunit{\printdelim{authortypedelim}}}%
     \iffieldundef{authortype}
       {}
       {\usebibmacro{authorstrg}%
        \setunit{\printdelim{nameyeardelim}}}}%
    {\global\undef\bbx@lasthash
     \usebibmacro{labeltitle}%
     \setunit*{\printdelim{nonameyeardelim}}}}%
  \usebibmacro{date+extradate}}
\let\ExecuteBibliographyOptions\@gobble
\nocite{elk:bronto,elk:einio,appleby:abl,appleby:civ}
\begin{bibexample}[title={\kvopt{dashed}{false}}]
\csuse{extblx@opt@dashed@false}
\printbibliography[heading=none]
\end{bibexample}

\begin{bibexample}[title={\kvopt{dashed}{bibnamehash}}]
\csuse{extblx@opt@dashed@bibnamehash}
\printbibliography[heading=none]
\end{bibexample}

\begin{bibexample}[title={\kvopt{dashed}{fullhash}}]
\csuse{extblx@opt@dashed@fullhash}
\printbibliography[heading=none]
\end{bibexample}
\makeatother
\end{refsection}
With \kvopt{dashed}{bibnamehash} the dash replaces the name list if they are
indistinguishable in the bibliography, while with \kvopt{dashed}{fullhash} the
lists are only replaced if they are indistinguishable in the data source.



\optitem[false]{introcite}{\optval{false}, \optval{plain}, \optval{label}}

\begin{warnbox}
The \opt{introcite} option is available for the bibliography styles of the
\hyperblxstyle[ext-authoryear]{authoryear} and
\hyperblxstyle[ext-authortitle]{authortitle} family.
It can not be used with citation styles of the
\hyperblxstyle[ext-verbose]{verbose} family, where
it may produce undesired output or errors.
The option is available globally and on a per-type and per-entry level, where
only the values \optval{plain} (and the default \optval{false}) are supported.
\end{warnbox}

This option controls whether or not the citation label is repeated in the
bibliography.
There are two possible output formats.
\begin{valuelist}
\item[false] Do not show the citation label in the bibliography.
\item[plain] Show the citation label at the beginning of an entry.
\item[label] Show the citation label as the label of a list similar to the
             \hyperblxstyle[ext-numeric]{numeric} or
             \hyperblxstyle[ext-alphabetic]{alphabetic} styles.
\end{valuelist}

The difference between \optval{plain} and \optval{label} is that the former
simply prints the citation label at the beginning of the entry, while the latter
prints the citation label similar to the item labels in a list or the numeric
labels in a \hyperblxstyle[ext-numeric]{numeric} bibliography.

\begingroup
\togglefalse{bbx:doi}
\setlength{\introcitewidth}{5.5\biblabelsep}
\DeclareFieldFormat{bbx@introcite}{\highlight{#1}}
\renewcommand*{\introcitepunct}{\highlight{\addcolon}\space}
\makeatletter
\begin{bibexample}[title={\kvopt{introcite}{false}}]
\csuse{extblx@opt@dashed@false}
\csletcs{extblx@introcite}{extblx@opt@introcite@false}
\exampleprintbib{sigfridsson,knuth:ct:a,knuth:ct:b}
\end{bibexample}

\begin{bibexample}[title={\kvopt{introcite}{plain}}]
\csuse{extblx@opt@dashed@false}
\csletcs{extblx@introcite}{extblx@opt@introcite@plain}
\exampleprintbib{sigfridsson,knuth:ct:a,knuth:ct:b}
\end{bibexample}

\begin{bibexample}[title={\kvopt{introcite}{label}}]
\csuse{extblx@opt@dashed@false}
\csletcs{extblx@introcite}{extblx@opt@introcite@label}
\exampleprintbib{sigfridsson,knuth:ct:a,knuth:ct:b}
\end{bibexample}
\makeatother
\endgroup


The label produced by the \optval{plain} option can be customised as follows.
\begin{ltxsyntax}
\csitem{introcitepunct}

The punctuation inserted between the label and the
rest of the entry with \kvopt{introcite}{plain}.
The default value is a colon followed by a space.
\begin{bibexample}
\togglefalse{bbx:doi}
\renewcommand*{\introcitepunct}{\highlight{\textbf{\addcolon}\textvisiblespace}}
\csuse{extblx@opt@dashed@false}
\csletcs{extblx@introcite}{extblx@opt@introcite@plain}
\exampleprintbib{sigfridsson}
\end{bibexample}
\end{ltxsyntax}

\begin{keymarglistbook}
\keyitem[bbx:introcite:plain:keeprelated] This toggle controls whether or
  not the citation label is also repeated for default related entries.
  The default value \optval{false} suppresses the label for related entries.
\begin{bibexample}[title={\gencode{\string\togglefalse\{%
  bbx:introcite:plain:keeprelated\}} (default)}]
\togglefalse{bbx:doi}
\csletcs{extblx@introcite}{extblx@opt@introcite@plain}
\exampleprintbib{vizedom:related}
\end{bibexample}
\begin{bibexample}[title={\gencode{\string\toggletrue\{%
  bbx:introcite:plain:keeprelated\}}}]
\toggletrue{bbx:introcite:plain:keeprelated}
\renewbibmacro*{related:init}{%
  \csundef{bbx:relatedloop}%
  \iftoggle{bbx:introcite:plain:keeprelated}{%
    \DeclareFieldFormat{bbx@introcite}{\highlight{##1}}%
    \renewcommand*{\introcitepunct}{\highlight{\addcolon}\space}%
  }{\renewbibmacro{introcite:plain}{}}}
\csuse{extblx@opt@dashed@false}
\csletcs{extblx@introcite}{extblx@opt@introcite@plain}
\exampleprintbib{vizedom:related}
\end{bibexample}
Note that the implementation of some related types automatically suppresses
the introcite label~-- or rather, does not add the label in the first place.
That means that the resulting bibliography should be checked for consistency
if the toggle is set to \optval{true}.
\end{keymarglistbook}

The \optval{label} option can be configured to not allow the label to run into
the remaining bibliography entry thus creating the appearance of a tabular-like
bibliography.
The citation label is not broken across lines, instead it moves the entry text
into the next line with \cmd{introcitebreak} if the width of the citation is
greater than \len{introcitewidth}.

\begin{ltxsyntax}
\lenitem{introcitewidth} The maximum width of the citation label.
  The initial value is 8 times \len{biblabelsep}.
\lenitem{introcitesep} This length sets the minimal space between the end of the
  citation label and the beginning of the rest of the entry.
  The initial value is \len{biblabelsep}.
\csitem{introcitebreak} The command to execute if a citation label exceeds
  \len{introcitewidth}. The default is \cs{leavevmode}\cs{newline}.
\end{ltxsyntax}

\makeatletter
\togglefalse{bbx:doi}
\begingroup
\setlength{\introcitewidth}{4.2\biblabelsep}
\setlength{\introcitesep}{3.8\biblabelsep}
\begin{bibexample}[enhanced, title={Lengths for \kvopt{introcite}{label}},
overlay={%
  \draw[highlight1, line width=.24mm,|-|] (frame.west)++(5.5mm,-.38cm) --
    node [at end, below=1pt] {\len{introcitewidth}} ++
    (\introcitewidth,0);
  \draw[highlight2, line width=.24mm,|-|] (frame.west)++
    (5.5mm+\introcitewidth,-.8mm) -- node [at start, above=2pt]
    {\len{introcitesep}} ++ (\introcitesep,0);
  \draw[highlight2, line width=.24mm,|-|] (frame.west)++
    (5.5mm+\introcitewidth,2.32cm) -- ++ (\introcitesep,0);
}
]
\csuse{extblx@opt@dashed@false}
\csletcs{extblx@introcite}{extblx@opt@introcite@label}
\exampleprintbib{sigfridsson,coleridge,geer}
\end{bibexample}
\endgroup

\begin{bibexample}[enhanced, title={\kvopt{introcite}{label} with empty
  \cs{introcitebreak}}]
\renewcommand*{\introcitebreak}{}
\setlength{\introcitewidth}{1.8cm}
\csuse{extblx@opt@dashed@false}
\csletcs{extblx@introcite}{extblx@opt@introcite@label}
\exampleprintbib{coleridge,geer}
\end{bibexample}

\begin{bibexample}[enhanced, title={\kvopt{introcite}{label} with
  \len{introcitewidth} set to zero and \len{introcitesep} equal to
  \len{bibhang}}]
\setlength{\introcitewidth}{0pt}
\setlength{\introcitesep}{\bibhang}
\csuse{extblx@opt@dashed@false}
\csletcs{extblx@introcite}{extblx@opt@introcite@label}
\exampleprintbib{sigfridsson,geer}
\end{bibexample}
\makeatother

In order to reproduce the citation label accurately in the bibliography,
some initialisation and housekeeping needs to be done.
\begin{ltxsyntax}
\cmditem{AtIntrocite}{code}
\cmditem*{AtIntrocite}*{code}

Appends the \prm{code} to an internal hook executed when initializing
\cmd{introcite}.
The starred variant of the command clears the initialisation hook,
so the defaults can be overwritten.
The default settings execute the internal macro
\cs{extblx@introcite@init}, which makes sure that the label produced
for introcite takes into account the citation settings and suppresses
unwanted citation tracking and hyperlinks.
\end{ltxsyntax}

The appearance of the citation label can be customised mostly as if it were
produced by a true citation command called \cmd{bbx@introcite}.
The delimiter context is \gencode{bbx@introcite}, the inner citation delimiters
can be accessed as \gencode{bbx@introcite} as well.
The label does not have outer citation delimiters, you can use the wrapper
field format \gencode{bbx@introcite} instead. In fact this approach is more
versatile than the outer citation delimiter feature (see the discussion in
\cref{sec:opt:citedelims}).
The default settings for \cmd{bbx@introcite} emulate the output of \cmd{cite}.

\begin{bibexample}[title={Example customisations for \kvopt{introcite}{plain}}]
\begin{lstlisting}[style=extblxstylelatex]
\DeclareFieldFormat{bbx@introcite}{\mkbibbrackets{#1}}
\DeclareDelimFormat[bbx@introcite]{nameyeardelim}{\addcomma\space}
\UndeclareInnerCiteDelims{bbx@introcite}
\renewcommand*{\introcitepunct}{\quad}
\end{lstlisting}
\tcblower
\DeclareFieldFormat{bbx@introcite}{\mkbibbrackets{#1}}
\DeclareDelimFormat[bbx@introcite]{nameyeardelim}{\addcomma\space}
\UndeclareInnerCiteDelims{bbx@introcite}
\renewcommand*{\introcitepunct}{\quad}
\csuse{extblx@opt@dashed@false}
\csletcs{extblx@introcite}{extblx@opt@introcite@plain}
\exampleprintbib{sigfridsson}
\end{bibexample}

\begin{bibexample}[title={Example customisations for \kvopt{introcite}{label}}]
\begin{lstlisting}[style=extblxstylelatex]
\DeclareFieldFormat{bbx@introcite}{\mkbibbold{#1}}
\DeclareDelimFormat[bbx@introcite]{nameyeardelim}{\addspace}
\DeclareInnerCiteDelims{bbx@introcite}{\bibopenparen}{\bibcloseparen}
\setlength{\introcitewidth}{0pt}
\setlength{\introcitesep}{\bibhang}
\end{lstlisting}
\tcblower
\DeclareFieldFormat{bbx@introcite}{\mkbibbold{#1}}
\DeclareDelimFormat[bbx@introcite]{nameyeardelim}{\addspace}
\DeclareInnerCiteDelims{bbx@introcite}{\bibopenparen}{\bibcloseparen}
\setlength{\introcitewidth}{0pt}
\setlength{\introcitesep}{\bibhang}
\csuse{extblx@opt@dashed@false}
\csletcs{extblx@introcite}{extblx@opt@introcite@label}
\exampleprintbib{coleridge,geer}
\end{bibexample}
\csgundef{blx@delimcontextalias@bbx@introcite}
% because \DeclareDelimcontextAlias is global ... at the moment, this can go
% with biblatex >= 3.12

If you are using an author-year citation style together with
\kvopt{introcite}{label} or \kvopt{introcite}{label}, you may be interested in
combining this with \kvopt{bibstyle}{ext-authortitle} instead to move the year
back to the end of the entry.
\makeatletter
\begin{bibexample}[enhanced, title={\kvopt{introcite}{label} with
  \kvopt{style}{ext-authoryear} and \kvopt{bibstyle}{ext-authortitle}}]
\setlength{\introcitewidth}{0pt}
\setlength{\introcitesep}{\bibhang}
\bbx@opt@mergedate@false
\renewbibmacro*{author}{%
  \ifboolexpr{
    test \ifuseauthor
    and
    not test {\ifnameundef{author}}
  }
    {\usebibmacro{bbx:dashcheck}
       {\bibnamedash}
       {\printnames{author}%
        \setunit{\printdelim{authortypedelim}}%
        \usebibmacro{bbx:savehash}}%
     \usebibmacro{authorstrg}}
    {\global\undef\bbx@lasthash}}
\csuse{extblx@opt@dashed@false}
\csletcs{extblx@introcite}{extblx@opt@introcite@label}
\exampleprintbib{sigfridsson,geer}
\end{bibexample}
\makeatother
\end{optionlist}

\clearpage
\section{Further Customisations}\label{sec:opt:cust}
Aside from the new options mentioned in the last section the styles of this
bundle also offer additional field formats, punctuation and delimiter commands,
a new citation delimiter interface and a few new bibliography macros.

The citation delimiter interface is a novel feature of \sty{biblatex-ext},
but for the other subsections familiarity with the underlying \sty{biblatex}
concepts is assumed.
Some of this is easier understood by looking at the source code directly,
so it might not be a bad idea to have \filename{ext-standard.bbx} open when
perusing this section of the manual.

\subsection{Field formats}\label{sec:opt:field}
The styles of this bundle offer customisable field formats
in a few places where the standard styles employ hard-coded formatting
directives instead.
Field formats can be modified with \cmd{DeclareFieldFormat}.

\begin{bibfieldformatlist}
\bibfieldformatitem{biblabeldate} The format for the labeldate in the
  bibliography for \hyperblxstyle[ext-authoryear]{authoryear}-like styles.
  The default is to wrap the date in round brackets.
  \begin{bibexample}
  \togglefalse{bbx:doi}
  \DeclareFieldFormat{biblabeldate}{\highlight{\bibopenparen}\highlight[2]{#1}%
    \highlight{\bibcloseparen}}
  \exampleprintbib{sigfridsson}
  \end{bibexample}

\bibfieldformatitem{biblistlabeldate} Like \bibfieldformat{biblabeldate},
  but for bibliography lists created by \cmd{printbiblist}.
  The default is to use the same format as \hyperbibfieldformat{biblabeldate}.

\bibfieldformatitem{issuedate} The format of the \bibfield{issue} and
  \bibfield{date} information for \bibtype{article}s. By default this block is
  wrapped in round brackets.
  \begin{bibexample}
  \makeatletter\bbx@opt@mergedate@false\makeatother
  \togglefalse{bbx:doi}
  \DeclareFieldFormat{issuedate}{\highlight{\bibopenparen}\highlight[2]{#1}%
    \highlight{\bibcloseparen}}
  \exampleprintbib{sigfridsson}
  \end{bibexample}

\bibfieldformatitem{volumeof} The format for the \bibfield{volume} of a
  \bibfield{maintitle} used when \hyperkvopt{maintitleaftertitle}{true}.
  \begin{bibexample}
  \togglefalse{bbx:isbn}\toggletrue{bbx:maintitleaftertitle}
  \DeclareFieldFormat{volumeof}{\highlight{\bibstring{volume}}~%
    \highlight[2]{#1}}
  \exampleprintbib{knuth:ct:a}
  \end{bibexample}

\bibfieldformatitem{titlecase:title} The standard styles follow an
  all-or-nothing approach when it comes to title casing. The field format
  \bibfieldformat{titlecase} intended to enable sentence case with
  \cmd{MakeSentenceCase*} is applied to all title-like fields alike. Finer
  control over the title casing of each field could require involved code.%
  \fnurl{https://tex.stackexchange.com/a/22981/}
  The field format \bibfieldformat{titlecase:title} is applied to the fields
  \bibfield{title} and \bibfield{subtitle}.
  By default this field format is an alias for \bibfieldformat{titlecase}.

\bibfieldformatitem{titlecase:booktitle}
  Like \hyperbibfieldformat{titlecase:title}, but controls the title casing
  of the \bibfield{booktitle} and \bibfield{booksubtitle} fields.

\bibfieldformatitem{titlecase:maintitle}
  Like \hyperbibfieldformat{titlecase:title}, but controls the title casing
  of the \bibfield{maintitle} and \bibfield{mainsubtitle} fields.

\bibfieldformatitem{titlecase:journaltitle}
  Like \hyperbibfieldformat{titlecase:title}, but controls the title casing
  of the \bibfield{journaltitle} and \bibfield{journalsubtitle} fields.

\bibfieldformatitem{titlecase:issuetitle}
  Like \hyperbibfieldformat{titlecase:title}, but controls the title casing
  of the \bibfield{issuetitle} and \bibfield{issuesubtitle} fields.

The \bibfieldformat{citetitle} field format can be used to change the title
case in author-title citations, so there is no
\bibfieldformat{citetitle:labeltitle}.

\begin{bibexample}[title={Default output for \bibfieldformat{titlecase}
  field formats}]%
\DeclareFieldFormat{titlecase:title}{\highlight[1]{#1}}
\DeclareFieldFormat{titlecase:journaltitle}{\highlight[2]{#1}}
\togglefalse{bbx:doi}
\exampleprintbib{shore}
\end{bibexample}

\begin{bibexample}[title={Example changes to \bibfieldformat{titlecase} field
  formats}]%
\begin{lstlisting}[style=extblxstylelatex]
\DeclareFieldFormat{titlecase:title}{|1\MakeSentenceCase*{#1}1|}
\DeclareFieldFormat{titlecase:journaltitle}{|2#12|}
\end{lstlisting}
\tcblower
\DeclareFieldFormat{titlecase:title}{\highlight[1]{\MakeSentenceCase*{#1}}}
\DeclareFieldFormat{titlecase:journaltitle}{\highlight[2]{#1}}
\togglefalse{bbx:doi}
\exampleprintbib{shore}
\end{bibexample}
\end{bibfieldformatlist}

\subsection{Punctuation}\label{sec:opt:punct}
The package provides the following commands to modify the delimiters and
punctuation between fields.
Normal punctuation commands should be redefined with \cmd{renewcommand},
while context-sensitive commands marked with \CSdelim{} should be redefined
with \cmd{DeclareDelimFormat}.
A short overview over common punctuation commands defined by \sty{biblatex-ext}
as well as standard \sty{biblatex} in an example bibliography
can be found in \cref{sec:punctinuse}.
\begin{ltxsyntax}
\csitem{innametitledelim}\CSdelimMark
Similar to \cmd{nametitledelim}, but for names after the \enquote{in:} if
\hyperopt{innamebeforetitle} is \optval{true}.
The default value is that of \cmd{nametitledelim} for all contexts.
Since the definition of \cmd{nametitledelim} is different for the delimiter
contexts \optval{bib} and \optval{biblist}, you may have to use the optional
argument to redefine the delimiter.\footnote{The author admits that it is
somewhat pointless to make \cmd{innametitledelim} context sensitive.
But the obvious parallels with \cmd{nametitledelim} were too tempting.
It is probably too late now.}
\begin{bibexample}
\toggletrue{bbx:innamebeforetitle}
\togglefalse{bbx:isbn}
\DeclareFieldFormat{editortype}{\mkbibparens{#1}}
\DeclareDelimFormat{editortypedelim}{\addspace}
\DeclareDelimFormat[bib,biblist]{innametitledelim}{\highlight{\textbf{%
  \addperiod}\textvisiblespace}\bibsentence}% <- hacky & hard-coded!
\exampleprintbib{pines}
\end{bibexample}
Note that \bibfieldformat{editortype} and \cmd{editortypedelim}
were changed for this example to avoid punctuation clashes.

Since \cmd{nametitledelim} and \cmd{innametitledelim} are independent,
the following output is easily achieved.
Note that the optional argument to \cmd{DeclareDelimFormat} is used
to make sure the definitions apply to the bibliography and bibliography lists
contexts, this is necessary because these contexts have special pre-defined
values that would otherwise not be redefined.
\begin{bibexample}[title={Example customisations for \cs{innametitledelim}}]
\begin{lstlisting}[style=extblxstylelatex]
\ExecuteBibliographyOptions{innamebeforetitle=true}
\DeclareDelimFormat[bib,biblist]{nametitledelim}{|1\addcolon\space1|}
\DeclareDelimFormat[bib,biblist]{innametitledelim}{|2\addcomma\space2|}
\end{lstlisting}
\tcblower
\toggletrue{bbx:innamebeforetitle}
\DeclareDelimFormat[bib,biblist]{nametitledelim}{\highlightbf[1]{\addcolon}%
  \space}
\DeclareDelimFormat[bib,biblist]{innametitledelim}{\highlightbf[2]{\addcomma}%
  \space}
\exampleprintbib{gaonkar,gaonkar:in}
\end{bibexample}

\csitem{maintitletitledelim}
The punctuation between the \bibfield{maintitle} and \bibfield{title} or
\bibfield{booktitle} of a work if \hyperopt{maintitleaftertitle} is
\optval{false}.
The default is \cs{newunitpunct}.
\begin{bibexample}
\togglefalse{bbx:isbn}
\renewcommand*{\maintitletitledelim}{\highlight{\textbf{\addperiod}%
  \textvisiblespace}\bibsentence}% <- hacky & hard-coded!
\exampleprintbib{knuth:ct:a}
\end{bibexample}

\csitem{voltitledelim}
The punctuation between the \bibfield{volume} and \bibfield{title} or
\bibfield{booktitle} of a work if \hyperopt{maintitleaftertitle} is
\optval{false}.
The default is a colon followed by a space.
\begin{bibexample}
\togglefalse{bbx:isbn}
\renewcommand*{\voltitledelim}{\highlight{\textbf{\addcolon}%
  \textvisiblespace}\bibsentence}
\exampleprintbib{knuth:ct:a}
\end{bibexample}

\csitem{titlemaintitledelim}
The punctuation between the \bibfield{title} or \bibfield{booktitle} and
\bibfield{maintitle} of a work if \hyperopt{maintitleaftertitle} is
\optval{true}.
The default is \cmd{newunitpunct}.
\begin{bibexample}
\togglefalse{bbx:isbn}
\toggletrue{bbx:maintitleaftertitle}
\renewcommand*{\titlemaintitledelim}{\highlight{\textbf{\addperiod}%
  \textvisiblespace}\bibsentence}%<- hacky & hard-coded!
\exampleprintbib{knuth:ct:a}
\end{bibexample}

\csitem{titleaddonpunct}
The punctuation between \bibfield{title}\slash\bibfield{subtitle} and
\bibfield{maintitle} as well as other \bibfield{title}-like and
\bibfield{maintitle}-like fields. The default is \cmd{newunitpunct}
\begin{bibexample}
\renewcommand*{\titleaddonpunct}{\highlight{\textbf{\addperiod}%
  \textvisiblespace}\bibsentence}%<- hacky & hard-coded!
\exampleprintbib{salam}
\end{bibexample}

\csitem{jourvoldelim}
The delimiter between the \bibfield{journaltitle} and \bibfield{volume} fields
for \bibtype{article} if no \bibfield{series} is present.
The default is a space.
\begin{bibexample}
\togglefalse{bbx:doi}
\renewcommand*{\jourvoldelim}{\highlight{\textvisiblespace}}
\makeatletter\bbx@opt@mergedate@false\makeatother
\exampleprintbib{sigfridsson}
\end{bibexample}

\csitem{jourserdelim}
The delimiter between the \bibfield{journaltitle} and \bibfield{series} fields
for \bibtype{article}.
The default is \cmd{newunitpunct}.
\begin{bibexample}
\togglefalse{bbx:doi}
\renewcommand*{\jourserdelim}{\highlight{\textbf{\addperiod}\textvisiblespace}%
                              \bibsentence% <- hacky ...
                              }% <- hard-coded!
\makeatletter\bbx@opt@mergedate@false\makeatother
\exampleprintbib{reese,shore}
\end{bibexample}

\csitem{servoldelim}
The delimiter between the \bibfield{series} and \bibfield{volume} fields
for \bibtype{article}.
The default is \hypercmd{jourvoldelim}.
\begin{bibexample}
\togglefalse{bbx:doi}
\renewcommand*{\servoldelim}{\highlight{\textvisiblespace}}% <- hard-coded!
\makeatletter\bbx@opt@mergedate@false\makeatother
\exampleprintbib{reese,shore}
\end{bibexample}

\csitem{volnumdatedelim}
The delimiter between the \bibfield{volume}, \bibfield{number} block and the
date information for \bibtype{article}.
The default is a space.
\begin{bibexample}
\togglefalse{bbx:doi}
\renewcommand*{\volnumdatedelim}{\highlight{\textvisiblespace}}
\makeatletter\bbx@opt@mergedate@false\makeatother
\exampleprintbib{sigfridsson}
\end{bibexample}


\csitem{volnumdelim}
The delimiter between \bibfield{volume} and \bibfield{number} for
\bibtype{article}.
The default is a dot.
\begin{bibexample}
\togglefalse{bbx:doi}
\renewcommand*{\volnumdelim}{\highlightbf{\adddot}}
\exampleprintbib{sigfridsson}
\end{bibexample}

\csitem{sernumdelim}
The delimiter between \bibfield{series} and \bibfield{number} for
\bibtype{book}- and \bibtype{inbook}-like entries.
The default is a space.
\begin{bibexample}
\renewcommand*{\sernumdelim}{\highlight{\textvisiblespace}}
\exampleprintbib{coleridge}
\end{bibexample}

\csitem{locdatedelim}
The delimiter between \bibfield{location} and \bibfield{date}.
The default is a comma followed by a space.
\begin{bibexample}
\renewcommand*{\locdatedelim}{\highlight{\textbf{\addcomma}\textvisiblespace}}
\makeatletter\bbx@opt@mergedate@false\makeatother
\exampleprintbib{jaffe}
\end{bibexample}

\csitem{locpubdelim}
The delimiter between \bibfield{location} and \bibfield{publisher}\slash%
\bibfield{organization}\slash\bibfield{institution}.
The default is a colon followed by a space.
\begin{bibexample}
\makeatletter\bbx@opt@mergedate@false\makeatother
\renewcommand*{\locpubdelim}{\highlight{\textbf{\addcolon}\textvisiblespace}}
\exampleprintbib{knuth:ct:a}
\end{bibexample}


\csitem{publocdelim}
The delimiter between \bibfield{publisher}\slash
\bibfield{organization}\slash
\bibfield{institution} and \bibfield{location}.
The default is a comma followed by a space.
This delimiter is not used by the default style, since the standard
order of fields is \bibfield{location}, \bibfield{publisher}\slash
\bibfield{organization}\slash\bibfield{institution}, \bibfield{date}.
You could use \cmd{publocdelim} if you changed the order of these
fields to \bibfield{publisher}\slash\bibfield{organization}\slash
\bibfield{institution}, \bibfield{location} \bibfield{date} by
redefining the bibmacro~\bibmacro{pubinstorg+location+date} as follows.
\begin{biblatexcode}
\renewbibmacro*{pubinstorg+location+date}[1]{%
  \printlist{#1}%
  \setunit*{|1\publocdelim1|}%
  \printlist{location}%
  \setunit*{|2\locdatedelim2|}%
  \usebibmacro{date}%
  \newunit}
\end{biblatexcode}
\begin{bibexample}
\renewcommand*{\publocdelim}{\highlight[1]{\textbf{\addcomma}\textvisiblespace}}
\renewcommand*{\locdatedelim}{\highlight[2]{%
  \textbf{\addcomma}\textvisiblespace}}
\renewbibmacro*{pubinstorg+location+date}[1]{%
  \printlist{#1}%
  \setunit*{\publocdelim}%
  \printlist{location}%
  \setunit*{\locdatedelim}%
  \usebibmacro{date}%
  \newunit}
\makeatletter\bbx@opt@mergedate@false\makeatother
\exampleprintbib{knuth:ct:a}
\end{bibexample}


\csitem{pubdatedelim}
The delimiter between \bibfield{publisher}\slash\bibfield{organization}\slash
\bibfield{institution} and \bibfield{date}.
The default is a comma followed by a space.
\begin{bibexample}
\renewcommand*{\pubdatedelim}{\highlight{\textbf{\addcomma}\textvisiblespace}}
\makeatletter\bbx@opt@mergedate@false\makeatother
\exampleprintbib{knuth:ct:a}
\end{bibexample}


\csitem{extradateonlycompcitedelim}\CSdelimMark
Similar to \cmd{compcitedelim}, but indended for use between compressed
citations where the second is an \bibfield{extradate} only.
The default is a comma (\emph{not} followed by a space).
\begin{bibexample}
\DeclareDelimFormat{extradateonlycompcitedelim}{\highlightbf{\addcomma}}
\cite{knuth:ct:b,knuth:ct:c}
\end{bibexample}


\csitem{extradateonlycompciterangedelim}\CSdelimMark
Similar to \hypercmd{extradateonlycompcitedelim}, but indended for use in a
compressed list of \bibfield{extradate}s in the \blxstyle{-ecomp} styles.
The default is \cmd{bibrangedash}.
\begin{bibexample}
\DeclareDelimFormat{extradateonlycompciterangedelim}{%
  \highlightbf{\bibrangedash}}
\cite{knuth:ct:b,knuth:ct:c,knuth:ct:d}
\end{bibexample}
\end{ltxsyntax}

\subsection{Delimiters for citation commands}\label{sec:opt:citedelims}
The delimiters for citation commands provided by the styles of this bundle
offer a simple way to customise the bracketing of citation commands.

The citation commands \cmd{cite}, \cmd{parencite} and \cmd{textcite} come with
two sets of delimiters: A pair of \emph{outer delimiters} wrapped around the
resulting citation in its entirety and a pair of \emph{inner delimiters} that
sets off certains bits of the citation label from other information.
An example for outer delimiters would be the round brackets of \cmd{parencite}
for \hyperblxstyle[ext-authoryear]{authoryear}-like styles or the
square brackets of \cmd{cite} for \hyperblxstyle[ext-numeric]{numeric}- or
\hyperblxstyle[ext-alphabetic]{alphabetic}-like styles.
Inner delimiters would be the round brackets in \cmd{textcite} around
the year for \hyperblxstyle[ext-authoryear]{authoryear} or around the title for
\hyperblxstyle[ext-authortitle]{authortitle}.
See \cref{tab:citationdelims} for more details.

The delimiters are set up to work as paired delimiters, but you are free to
use non-matching pairs or to leave the opening or closing delimiter empty.
If you want to add punctuation, the context-sensitive delimiters
\gencode{nameyeardelim}, \gencode{nametitledelim},
\gencode{postnotedelim} and friends are more approriate.

\begin{remindbox}
Although the outer delimiters can be set up using \cmd{DeclareCiteCommand}'s
optional \prm{wrapper} argument for most styles, this is not possible for all
styles.
Similarly, inner delimiters can be set up with \cmd{DeclareFieldFormat} in some
styles, but other styles need more intricate implementations.
This means that the commands discussed here can be used to place the citations
between delimiters, but not natively to pass the result of a citation to a
wrapper command as an argument.
\end{remindbox}

\begin{table}[btph]
\centering
\caption[Outer and inner citation delimiters by style]{\highlight[1]{Outer} and
\highlight[2]{inner} citation delimiters by style. If the style does not use
the delimiters by default, \mpdl{} and \mpdr{} are substituted in the
appropriate place.}
\label{tab:citationdelims}
\begin{tabular}{@{}llll@{}}
\toprule
             & \multicolumn{3}{c}{Citation command}\\
             \cmidrule(lr){2-4}
Style family & \cmd{cite} & \cmd{parencite} & \cmd{textcite}\\
\midrule
\blxstyle{alphabetic} & \highlightbf[1]{[}Knu84\highlightbf[1]{]} &
  \highlightbf[1]{[}Knu84\highlightbf[1]{]} & \highlightbf[1]{\hmpdl}Knuth
  \highlightbf[2]{[}2\highlightbf[2]{]}\highlightbf[1]{\hmpdr}\\
\blxstyle{numeric} & \highlightbf[1]{[}2\highlightbf[1]{]} &
  \highlightbf[1]{[}2\highlightbf[1]{]} & \highlightbf[1]{\hmpdl}Knuth
  \highlightbf[2]{[}2\highlightbf[2]{]}\highlightbf[1]{\hmpdr}\\
\blxstyle{authortitle} & \highlightbf[1]{\hmpdl}Knuth,
  \highlightbf[2]{\hmpdl}\emph{\TeX book}\highlightbf[2]{\hmpdr}%
  \highlightbf[1]{\hmpdr} & \highlightbf[1]{(}Knuth,
  \highlightbf[2]{\hmpdl}\emph{\TeX book}\highlightbf[2]{\hmpdr}%
  \highlightbf[1]{)} & \highlightbf[1]{\hmpdl}Knuth \highlightbf[2]{(}%
  \emph{\TeX book}\highlightbf[2]{)}\highlightbf[1]{\hmpdr}\\
\blxstyle{authoryear} & \highlightbf[1]{\hmpdl}Knuth
  \highlightbf[2]{\hmpdl}1984\highlightbf[2]{\hmpdr}%
  \highlightbf[1]{\hmpdr} & \highlightbf[1]{(}Knuth
  \highlightbf[2]{\hmpdl}1984\highlightbf[2]{\hmpdr}%
  \highlightbf[1]{)} & \highlightbf[1]{\hmpdl}Knuth
  \highlightbf[2]{(}1984\highlightbf[2]{)}\highlightbf[1]{\hmpdr}\\
\bottomrule
\end{tabular}
\end{table}

\begin{ltxsyntax}
\cmditem{DeclareOuterCiteDelims}{cite command}{opening delimiter}
        {closing delimiter}

Sets up the outer delimiters for the citation command
\cmd{\prm{cite command}}. The name of the \prm{cite command}
is given without leading backslash in the argument, it normally corresponds to
the delimiter context.

Note that outer citation delimiters are often hard-coded so they usually do not
respond to delimiter context aliases.
Inner citation delimiters, on the other hand, usually follow delimiter context
aliases.
This may lead to initially surprising results if aliases are used.

You may use almost any input for \prm{opening delimiter} and
\prm{closing delimiter} as long as typesetting of
\enquote{\prm{opening delimiter}text\prm{closing delimiter}} does not lead to
errors when arbitrary grouping such as
\enquote{\{\prm{opening delimiter}\allowbreak text\}\allowbreak
\prm{closing delimiter}}
or \enquote{\prm{opening delimiter}\allowbreak text\allowbreak
\{\prm{closing delimiter}\}}
is introduced.
It can not be guaranteed that the opening and closing delimiters are executed
at the same level of grouping, let alone in the same group.
As mentioned above, this approach is not suitable to wrap the citation up in a
wrapper command, i.e.\ to pass the entire output of the citation command as
argument to a macro.

Instead of hard-coded \gencode{(}, \gencode{)}, \gencode{[} and \gencode{]}
their \sty{biblatex} counterparts \cmd{bibopenparen}, \cmd{bibcloseparen},
\cmd{bibopenbracket} and \cmd{bibclosebracket} are preferable, since these
commands respond to nesting and check if opening brackets are always closed.

\cmditem{DeclareOuterCiteDelimsAlias}{cite alias}{cite command}
\cmditem*{DeclareOuterCiteDelimsAlias*}{cite alias}{cite command}

Use the outer delimiters of \cmd{\prm{cite command}} for
\cmd{\prm{cite alias}} as well.
The unstarred version uses \cmd{def} assignment while the starred version uses
\cmd{let}. This means that the starred version copies the values of the
definitions at the time of executing the aliasing command,
whereas the alias created by the unstarred version will only evaluate the
delimiters whenever the citation command is called.

\cmditem{UndeclareOuterCiteDelims}{cite command}

Completely remove the definitions of the outer delimiters for
\cmd{\prm{cite command}}.

\cmditem{DeclareInnerCiteDelims}{cite command}{opening delimiter}
        {closing delimiter}

Sets up the inner delimiters for the citation command
\cmd{\prm{cite command}}.

This command is similar to \hypercmd{DeclareOuterCiteDelims} and the same
restrictions for the arguments apply.

Note that inner citation delimiters usually follow context delimiter aliases.
Outer citation delimiters, on the other hand, do not respond to context
delimiter aliases.
This may lead to initially surprising results if aliases are used.

\cmditem{DeclareInnerCiteDelimsAlias}{cite alias}{cite command}
\cmditem*{DeclareInnerCiteDelimsAlias*}{cite alias}{cite command}

Use the inner delimiters of \cmd{\prm{cite command}} for
\cmd{\prm{cite alias}} as well.
The unstarred version uses \cmd{def} assignment while the starred version uses
\cmd{let}. This means that the starred version copies the values of the
definitions at the time of executing the aliasing command,
whereas the alias created by the unstarred version will only evaluate the
delimiters whenever the citation command is called.

\cmditem{UndeclareInnerCiteDelims}{cite command}

Completely remove the definitions of the inner delimiters for
\cmd{\prm{cite command}}.

\cmditem{RegisterCiteDelims}{modifier}{cite command}

Register a pair of \prm{modifier} citation delimiters for
\cmd{\prm{cite command}}.
This command will define a delimiter wrapper command
\cmd{mk\prm{modifier}\prm{cite command}s} that places its argument between
the opening delimiter
\cs{ext\allowbreak blx@\allowbreak cite\allowbreak delim@\allowbreak
  \prm{cite command}@\allowbreak\prm{modifier}@\allowbreak open} and
the closing delimiter
\cs{ext\allowbreak blx@\allowbreak cite\allowbreak delim@\allowbreak
  \prm{cite command}@\allowbreak\prm{modifier}@\allowbreak close}.
No error will be raised if the opening or closing delimiters are not defined,
but an error will be raised if \cmd{mk\prm{modifier}\prm{cite command}s} is
already defined.

The styles of this bundle only use the values \optval{outer} and \optval{inner}
for \prm{modifier}, but other values are possible. With \optval{outer} and
\optval{inner} you can use \hypercmd{DeclareOuterCiteDelims} or
\hypercmd{DeclareInnerCiteDelims} and friends to define the opening and closing
delimiters, otherwise you need to define them manually.

All styles of \sty{biblatex-ext} define the wrapper commands listed in
\cref{tab:delimwrappers}.
\begin{table}[btph]
\centering
\caption{Delimiter wrappers defined by \sty{biblatex-ext}}
\label{tab:delimwrappers}
\begin{minipage}{\linewidth}
\renewcommand\footnoterule{}
\centering
\begin{tabular}{@{}lll@{}}
\toprule
                 & \multicolumn{2}{c}{Modifier} \\
                 \cmidrule(lr){2-3}
Citation command & Outer & Inner\\
\midrule
\cmd{cite} & \cmd{mkoutercitedelims} & \cmd{mkinnercitedelims}\\
\cmd{parencite} & \cmd{mkouterparencitedelims} & \cmd{mkinnerparencitedelims}\\
\cmd{textcite} & \cmd{mkoutertextcitedelims} & \cmd{mkinnertextcitedelims}\\
\cmd{footcite} & \cmd{mkouterfootcitedelims} & \cmd{mkinnerfootcitedelims}\\
\cmd{supercite}\footnote{If defined by the citation style.}
 & \cmd{mkoutersupercitedelims} & ---\\
\cmd{bbx@xrefcite} & \cmd{mkouterbbx@xrefcitedelims} &
  \cmd{mkinnerbbx@xrefcitedelims}\\
\enquote{\cmd{bbx@introcite}} & ---%
    \footnote{Use the field format \bibfieldformat{bbx@introcite} instead,
      see \cref{sec:opt:style}.} & \cmd{mkinnerbbx@introcitedelims}\\
\bottomrule
\end{tabular}
\end{minipage}
\end{table}


The \hyperblxstyle[ext-authoryear]{authoryear} and
\hyperblxstyle[ext-authortitle]{authortitle} styles have \cmd{parencite},
e.g.\ \parencite{knuth:ct:a}, \parencite{sigfridsson}, set up with
\begin{biblatexcode}
\DeclareOuterCiteDelims{parencite}{\bibopenparen}{\bibcloseparen}
\DeclareInnerCiteDelims{parencite}{}{}
\end{biblatexcode}
and \cmd{textcite}, e.g.\ \textcite{knuth:ct:a}, \textcite{sigfridsson}, with
\begin{biblatexcode}
\DeclareOuterCiteDelims{textcite}{}{}
\DeclareInnerCiteDelims{textcite}{\bibopenparen}{\bibcloseparen}
\end{biblatexcode}
If you wanted \cmd{parencite} of \hyperblxstyle[ext-authoryear]{authoryear}
to look like \enquote{[Sigfridsson and Worman (1998)]} you would use
\begin{biblatexcode}
\DeclareOuterCiteDelims{parencite}{|1\bibopenbracket1|}{|1\bibclosebracket1|}
\DeclareInnerCiteDelims{parencite}{|2\bibopenparen2|}{|2\bibcloseparen2|}
\end{biblatexcode}
\citereset
\begin{bibexample}
\DeclareOuterCiteDelims{parencite}{\highlightbf[1]{\bibopenbracket}}{%
  \highlightbf[1]{\bibclosebracket}}
\DeclareInnerCiteDelims{parencite}{\highlightbf[2]{\bibopenparen}}{%
  \highlightbf[2]{\bibcloseparen}}
\parencite{sigfridsson}\quad\parencite{worman,geer}\quad
\parencite{knuth:ct:a,knuth:ct:b,knuth:ct:c}
\end{bibexample}
\end{ltxsyntax}

\subsection{Selected bibliography macros}\label{sec:opt:bibmacros}
The following macros are defined in \filename{ext-standard.bbx} and may make
certain things easier to customise.
Many of these macros are replacements for bare \cmd{printfield} or
\cmd{printlist} in the bibliography drivers, or pack a frequently-used
sequence of commands into one central place.

\begin{keymarglistbook}
\keyitem[barevolume+volumes]
A bibliography macro to print the \bibfield{volume}, \bibfield{part} and
\bibfield{volumes} fields for \bibtype{mvbook}-, \bibtype{book}- and
\bibtype{inbook}-like entry types.
If \bibfield{maintitle} is defined, the \bibfield{volume} and \bibfield{part}
fields will be printed by \bibmacro{maintitle+title} or
\bibmacro{maintitle+booktitle} instead.

\begin{bibexample}
\renewbibmacro*{barevolume+volumes}{%
  \printtext[highlight1]{%
    \iffieldundef{maintitle}
      {\printfield{volume}%
       \printfield{part}}
      {}%
    \newunit
    \printfield{volumes}}}
\exampleprintbib{knuth:ct,matuz:doody}
\end{bibexample}

\keyitem[edition]
A bibliography macro to print the \bibfield{edition} field, this avoids a direct
\lstinline|\printfield{edition}| in the bibliography drivers.

\keyitem[version]
Similar to the \hyperbibmacro{edition} bibliography macro, \bibmacro{version}
prints the \bibfield{version} field, this avoids a direct
\lstinline|\printfield{version}| in the bibliography drivers.

\keyitem[pagetotal]
Similar to the \hyperbibmacro{edition} bibliography macro, \bibmacro{pagetotal}
prints the \bibfield{pagetotal} field, this avoids a direct
\lstinline|\printfield{pagetotal}| in the bibliography drivers.

\keyitem[in:editor(+others)]
The bibliography macro to print the \bibfield{editor} before the
\bibfield{booktitle} for \bibtype{inbook}, \bibtype{incollection} and
\bibtype{inproceedings} when \hyperopt{innamebeforetitle} is set to
\optval{true}.
The \cmd{printname} uses the name format \bibfield{ineditor}.

\keyitem[language]
A bibliography macro to print the \bibfield{language} field, this avoids a
direct \lstinline|\printlist{language}| in the bibliography drivers.

\keyitem[note]
A bibliography macro to print the \bibfield{note} field, this avoids a direct
\lstinline|\printfield{note}| in the bibliography drivers.

\keyitem[pubinstorg+location+date]
A general-purpose bibliography macro to catch
\bibmacro{publisher+location+date},
\bibmacro{institution+location+date} and \bibmacro{organization+location+date}.
This bibliography macro has one mandatory argument: the name of a list field,
sensible values are \bibfield{publisher}, \bibfield{institution} and
\bibfield{organization}.
By default the macro is defined and used as follows. 
\begin{biblatexcode}
\newbibmacro*{pubinstorg+location+date}[1]{%
  \printlist{location}%
  \iflistundef{|#1|}
    {\setunit*{\locdatedelim}}
    {\setunit*{\locpubdelim}}%
  \printlist{|#1|}%
  \setunit*{\pubdatedelim}%
  \usebibmacro{date}%
  \newunit}

\renewbibmacro*{|publisher|+location+date}{%
  \usebibmacro{pubinstorg+location+date}{|publisher|}}

\renewbibmacro*{|institution|+location+date}{%
  \usebibmacro{pubinstorg+location+date}{|institution|}}

\renewbibmacro*{|organization|+location+date}{%
  \usebibmacro{pubinstorg+location+date}{|organization|}}
\end{biblatexcode}

\keyitem[type+number]
A bibliography macro to print the \bibfield{type} and \bibfield{number} fields.
\end{keymarglistbook}

\clearpage
\section{Tabular Bibliographies}\label{sec:tabularbib}
\label{exltd@itm@biblatex-ext-tabular}
An experimental addition to this bundle is the package
\sty{biblatex-ext-tabular} which allows you to typeset tabular
(two-column) bibliographies.
The package must be loaded after \sty{biblatex} and has no further options.
\begin{biblatexcode}
\usepackage[(*@\prm{options}@*)]{biblatex}
(*@\dots @*)
\usepackage{biblatex-ext-tabular}
\end{biblatexcode}
If it turns out that the tabular bibliography package is useful on its own,
it may be split off from the rest of this bundle in the future.
In that case the name \sty{biblatex-ext-tabular} may be deprecated in favour
of something more snappy.
The old name should then continue to work, but you may have to install a new
package to be able to use it.

The package \sty{biblatex-ext-tabular} can be used independently of the styles
provided in this bundle.
But no matter what style you use, you will probably have to apply some tweaks
here and there to make the output look acceptable.

The code for the tabular bibliographies is based on Audrey's%
\fnurl{https://tex.stackexchange.com/users/4483/audrey}
answer to a question%
\fnurl{https://tex.stackexchange.com/q/71088/35864}
on \tex{}~--~\latex{} Stack Exchange.
The code has been modernised to keep in sync with changes to internal macros
and modified to allow for slightly more flexible setups, so any bugs you find
are mine and not Audrey's. Please report them via the channels mentioned above
in~\cref{sec:feedback}.

Before tabular bibliographies can be used, a suitable definition
must be supplied with \hypercmd{defbibtabular}.

\begin{ltxsyntax}
\cmditem{printbibtabular}[key=value, \dots]

This commands prints a tabular bibliography.
It accepts the same options as \cmd{printbibliography}, but note that
\optval{env} must be a tabular bibliography declared with
\hypercmd{defbibtabular}.
The default value for \optval{env} is \optval{bibtabular}.

\begin{warnbox}
Note that there is no default definition for the
\opt{bibtabular} environment, so you must define it with
\hypercmd{defbibtabular} before use.
An error will be issued if you fail to provide a definition
for \optval{bibtabular} before you use \cmd{printbibtabular} with the default
\optval{env} value and no bibliography will be printed.
\end{warnbox}

\cmditem{defbibtabular}{name}{begin code}{end code}{row code}

This command defines a tabular bibliography.
The \prm{name} is an identifier passed to the \optval{env} option of
\hypercmd{printbibtabular} when selecting the environment.
The \prm{begin code} is \latex{} code to be executed at the beginning of the
environment, usually it will begin a tabular environment.
The \prm{end code} is executed at the end of the environment and would
usually end the tabular environment.
The \prm{row code} holds the code for one row in the tabular bibliography,
it should have no more columns than ordered in \prm{begin code} and should
end with \lstinline|\\|.

\begin{remindbox}
Please keep in mind that the tabular bibliography is subject to all limitations
of the particular tabular environment you decide to use.
\end{remindbox}

Within the \prm{row code} argument the following macros are defined and can be
used to typeset the result of bibliography macros.
Each cell should normally be wrapped into exactly one of the following macros
since they are responsible for providing the necessary entry data to the
bibliography macros.

\cmditem{plain}{code}

Typesets \prm{code} with the entry data of the current entry.

\cmditem{plainlang}{code}

Like \cmd{plain}, but typesets \prm{code} in the language of the current
item if the language switching options are enabled.

\cmditem{anchor}{code}

Like \cmd{plain}, but additionally sets a link anchor for the current entry.
That link anchor is used when a citation links to the bibliography entry.
There should only be one \cmd{anchor} or \hypercmd{anchorlang} per entry (and
hence per row),
and for proper linking there should be at least one.
Usually \cmd{anchor} or \hypercmd{anchorlang} is the first cell in a row.

\cmditem{anchorlang}{code}

Like \hypercmd{anchor}, but in the entry language like \hypercmd{plainlang}.

\cmditem{driver}{precode}

Typesets the bibliography entry using the correct driver.
This always uses the entry language if language switching is enabled.
The \prm{precode} argument can be used to execute code before the driver
is executed.
Usually \cmd{driver} would be issued in the last (and widest) cell in a row,
duplicate information can be cleaned up with help of the \prm{precode} argument.

Recall that standard \env{tabular} can't break across pages, so you may want
to give \env{longtable} a try.
Additionally, the standard column specifiers \gencode{l}, \gencode{r} and
\gencode{c} define flexible-sized columns which might not be appropriate here;
\gencode{p} defines fixed-width columns but imposes justification, which does
not look good if the columns are too narrow.
You may want to look into fixed-width flush-left or flush-right columns%
\fnurl{https://tex.stackexchange.com/q/12703/35864}.
\end{ltxsyntax}

The package also defines two helpful bibliography macros.
\begin{keymarglistbook}
  \item[tabular:sortname] Prints the labelname/sortname.
  \item[tabular:omitsortname] Suppresses the labelname/sortname in the current
    group.
\end{keymarglistbook}

A simple example of a useful tabular bibliography setup is
\begin{biblatexcode}
\usepackage{longtable}
\usepackage{array}
\newcolumntype{L}[1]{%
  >{\raggedright\let\newline\\\arraybackslash\hspace{0pt}}p{#1}}

\defbibtabular{bibtabular}
  {\setlength{\LTpre}{0pt}%
   \setlength{\LTpost}{0pt}%
   \renewcommand*{\arraystretch}{2}%
   \begin{longtable}{%
     @{}
     L{\dimexpr0.3\textwidth-\tabcolsep\relax}
     L{\dimexpr0.7\textwidth-\tabcolsep\relax}
     @{}}}
  {\end{longtable}}
  {\anchorlang{\usebibmacro{tabular:sortname}} &
   \driver{\usebibmacro{tabular:omitsortname}} \\}
\end{biblatexcode}
This defines a two-column bibliography where the first column shows the
primary name associated with the work (the labelname/sortname) and the
second colum shows the bibliography data with the primary name suppressed
to avoid duplication.
The anchor is set to the first column, so that a viewer jumps there
if a link to the entry is followed.

\begin{bibexample}
  \defbibtabular{bibtabular}
    {\setlength{\LTpre}{0pt}%
     \setlength{\LTpost}{0pt}%
     \renewcommand*{\arraystretch}{2}%
     \begin{longtable}{%
       @{}
       L{\dimexpr0.3\textwidth-\tabcolsep\relax}
       L{\dimexpr0.7\textwidth-\tabcolsep\relax}
       @{}}}
    {\end{longtable}}
    {\anchorlang{\usebibmacro{tabular:sortname}} &
     \driver{\usebibmacro{tabular:omitsortname}} \\}

  \exampleprintbibtab{sigfridsson,geer,nussbaum}
\end{bibexample}

For the common case of two-column tabular bibliographies there is a thin
wrapper around \hypercmd{defbibtabular} that might turn out a bit easier to use.
\begin{ltxsyntax}
  \cmditem{defbibtabulartwocolumn}{name}{begin code}{end code}%
  {anchor code}{driver precode}

  This defines a two-column tabular bibliography environment with \prm{name},
  \prm{begin code} and \prm{end code} as for \hypercmd{defbibtabular}.

  The first column is an \hypercmd{anchorlang} column that executes
  \prm{anchor code}, the second column executes a pre-code and
  the bibliography driver.

\begin{biblatexcode}
\defbibtabulartwocolumn{(*@\prm{name}@*)}
  {(*@\prm{begin code}@*)}{(*@\prm{end code}@*)}
  {(*@\prm{anchor code}@*)}{(*@\prm{driver precode}@*)}
\end{biblatexcode}
is equivalent to
\begin{biblatexcode}
\defbibtabular{(*@\prm{name}@*)}
  {(*@\prm{begin code}@*)}{(*@\prm{end code}@*)}
  {\anchorlang{(*@\prm{anchor code}@*)} & \driver{(*@\prm{driver precode}@*)} \\}
\end{biblatexcode}

Using \cmd{defbibtabulartwocolumn} the example from above could be written as
\begin{biblatexcode}
\usepackage{longtable}
\usepackage{array}
\newcolumntype{L}[1]{%
  >{\raggedright\let\newline\\\arraybackslash\hspace{0pt}}p{#1}}

\defbibtabulartwocolumn{bibtabular}
  {\setlength{\LTpre}{0pt}%
   \setlength{\LTpost}{0pt}%
   \renewcommand*{\arraystretch}{2}%
   \begin{longtable}{%
     @{}
     L{\dimexpr0.3\textwidth-\tabcolsep\relax}
     L{\dimexpr0.7\textwidth-\tabcolsep\relax}
     @{}}}
  {\end{longtable}}
  {\usebibmacro{tabular:sortname}}
  {\usebibmacro{tabular:omitsortname}}
\end{biblatexcode}
\end{ltxsyntax}

\clearpage
\section{Open Access Symbols}\label{sec:oa}
\label{exltd@itm@biblatex-ext-oa}
The package \sty{biblatex-ext-oa} was inspired by a question%
\fnurl{https://tex.stackexchange.com/q/459449/35864}
by pluton\fnurl{https://tex.stackexchange.com/users/371/pluton} on TeX.SX
about promoting open access works by adding a small open access symbol into
the margin.

The package must be loaded after \sty{biblatex}.
It is not tied to use with styles of this bundle, though it can not be
guaranteed that everything will work out of the box with all other styles.
The package works as expected with the standard styles and
should work with most contributed styles at least to some extent.
For the actual typesetting of the symbol \sty{biblatex-ext-oa}
patches the bibliography macro \bibmacro{begentry} to include a call to the
bibliography macro \hyperbibmacro{oamark}.
\begin{biblatexcode}
\usepackage[(*@\prm{\sty{biblatex} options}@*)]{biblatex}
(*@\dots @*)
\usepackage[(*@\prm{\sty{biblatex-ext-oa} options}@*)]{biblatex-ext-oa}
\end{biblatexcode}

The package may load the support packages
\hypersty{biblatex-ext-oa-doiapi},
\hypersty{biblatex-ext-oasymb-pict2e},
\hypersty{biblatex-ext-oasymb-l3draw}
and \hypersty{biblatex-ext-oasymb-tikz}.
The preferred way to use those packages is to have them loaded
automatically via \sty{biblatex-ext-oa}'s package options.
It should usually not be necessary to load any of these support packages
directly with \cmd{usepackage}, but it is possible to use them as
stand-alone packages should the need arise in other situations.

\begin{warnbox}
Note that the macro \hypercmd{oasymbol} must be defined and should contain an
open access symbol that can be printed in the margin for the package
\sty{biblatex-ext-oa} to work properly.
The package \sty{biblatex-ext-oa} itself does not provide a definition for
\hypercmd{oasymbol} if the option \hyperopt{symbolpackage} is not used.
So you should either set that option or define command \hypercmd{oasymbol}
yourself.
\end{warnbox}

The package \sty{biblatex-ext-oa} provides the following load-time options.
\begin{optionlist}
\boolitem[true]{enable}

This option controls whether the package is only loaded (\optval{false}) or
loaded and active (\optval{true}).

\enquote{Active} means that the initial state of the global bibliography option
\hyperopt{openaccess} is \optval{auto} and that useful default settings are
applied for semi-automatic open access detection.
These useful defaults include patches of the bibmacro \bibmacro{begentry}
and, if \hyperopt{doiapi} is enabled, also \bibmacro{doi+eprint+url}.

If the package is loaded with \opt{enable} set to \optval{false} the global
bibliography option \hyperopt{openaccess} will be set to \optval{false} and no
sensible defaults are defined for the semi-automatic open access detection.

\optitem[false]{symbolpackage}{\optval{false}, \optval{pict2e}, \optval{l3draw},
  \optval{tikz}}

Load one of the available open access symbol support packages.
\begin{valuelist}
  \item[false] Don't load any symbol package.
    \begin{warnbox}
      Since \sty{biblatex-ext-oa} does not provide a default definition for
      \hypercmd{oasymbol}, you will have to make sure to define that command
      yourself if you don't load any of the symbol packages.
    \end{warnbox}
  \item[pict2e] Load \hypersty{biblatex-ext-oasymb-pict2e} to provide a
    definition for \hypercmd{oasymbol}.
    See \cref{sec:oa:symbol,sec:oa:symbol-pict2e}.
  \item[l3draw] Load \hypersty{biblatex-ext-oasymb-l3draw} to provide a
    definition for \hypercmd{oasymbol}.
    See \cref{sec:oa:symbol,sec:oa:symbol-l3draw}.
    \begin{remindbox}
      The package \hypersty{biblatex-ext-oasymb-l3draw} will load the entire
      \LaTeX3 core.
      Furthermore, \sty{l3draw} is still considered experimental.
    \end{remindbox}
  \item[tikz] Load \hypersty{biblatex-ext-oasymb-tikz} to provide a
    definition for \hypercmd{oasymbol}.
    See \cref{sec:oa:symbol,sec:oa:symbol-tikz}.
    \begin{remindbox}
      The package \hypersty{biblatex-ext-oasymb-tikz} will load \pgftikz.
    \end{remindbox}
\end{valuelist}

\begingroup
\makeatletter
% Avoid a multiply defined label.
% I know this is terrible, ...
\def\ltd@pdfbookmark#1#2{%
  \exltd@pdfbookmark{#1}{#2}}
\makeatother
\optitem{symbol}{\prm{symbol}}
\endgroup

Set \hypercmd{oasymbol} to output the symbol identified as \prm{symbol}.
The available options can be found in \cref{sec:oa:symbol}.

\begin{warnbox}
Note that this option is only available if the
option~\hyperopt{symbolpackage} has been set to a valid value different
from \optval{false} or one of the open access symbol packages has
been loaded manually (the latter is usually not recommended, the
option~\hyperopt{symbolpackage} is the preferred way to load those
packages).
\end{warnbox}

\boolitem[false]{doiapi}

Whether or not to load the support package \hypersty{biblatex-ext-oa-doiapi}
that provides a Lua module to query open access information
from Unpaywall\fnurl{https://unpaywall.org/} via the DOI of an entry.

\begin{warnbox}
The package \hypersty{biblatex-ext-oa-doiapi} and thus this option
can only be used with \lualatex.
\end{warnbox}
\end{optionlist}

\clearpage
The following macros in \sty{biblatex-ext-oa} are also of interest.
\begin{ltxsyntax}


\cmditem{LoadOASymbolPackage}[symbol]{package}

This command provides an alternative to \hyperopt{symbolpackage} for loading
a symbol package.
The mandatory argument \prm{package} can take the same values as the
\hyperopt{symbolpackage} option.
The optional argument \prm{symbol} has the same meaning as the possible values
for the \hyperopt{symbol} package option.

The command can only be used in the preamble.
It may only be used once and only if none of the symbol packages have been
loaded either explicitly or via the \hyperopt{symbolpackage} option.

This command is related to \hypercmd{DefineOASymbol} (see \cref{sec:oa:symbol}).

\cmditem{oasymbol}

This commands holds the little symbol that is going to be printed in the margin
to mark open access publications.

This command is initially undefined and must be defined for the
package to work.
One way of giving this command a useful definition is via the
\hyperopt{symbolpackage} option of \sty{biblatex-ext-oa}.
\end{ltxsyntax}

An entry will be marked as open access if it has a \enquote{best open access
URL}.
This best open access URL will be determined according to the rules set out by
the \hyperopt{openaccess} bibliography option (see \cref{sec:oa:detect}) in the
\cmd{AtDataInput} hook,
it is stored in the newly defined field \hyperbibfield{bestoaurl}.
\begin{fieldlist}
\fielditem{bestoaurl}{literal}
This field holds the \enquote{best open access URL} as determined by the open
access detection outlined in \cref{sec:oa:detect}.

Note that this field need not hold the URL in expanded form.
\end{fieldlist}

Depending on the load-time options the package may define the following two
bibliography macros.
\begin{keymarglistbook}
\keyitem[oamark]

Prints the open access symbol if a best open access URL is available.
This bibliography macro is available if the package is loaded with
\hyperopt{enable} set to \optval{true}

\keyitem[api:bestoaurl]

Display the best open access URL according to the Unpaywall API.
The URL is not shown if it coincides with the DOI, the check for that is
performed by \hypercmd{OpenaccessURLisDOI} (see \cref{sec:oa:doiapi}).
This macro is only available if both the \hyperopt{enable} and the
\hyperopt{doiapi} options are set to \optval{true}.

This macro is useful in connection with the Unpaywall API since the best
open access URL may not be amongst the fields that are already displayed
in the entry.
\end{keymarglistbook}

\begin{remindbox}
With \opt{enable} set to \optval{true} the package will furthermore patch
the bibliography macro \bibmacro{begentry} to execute the bibliography macro
\hyperbibmacro{oamark}.
If additionally \hyperopt{doiapi} is \optval{true}, the bibliography macro
\bibmacro{doi+eprint+url} will be patched to include a call to
\hyperbibmacro{api:bestoaurl}.
Please keep that in mind should you need to redefine \bibmacro{begentry}
or \bibmacro{doi+eprint+url}.
\end{remindbox}

\subsection{Open access detection}\label{sec:oa:detect}
The heart of the package \sty{biblatex-ext-oa} is the bibliography option
\hyperopt{openaccess}, which can be set globally, per type or per entry.

A longer example document exploring various aspects of open access
detection can be found at
\url{https://github.com/moewew/biblatex-ext/blob/master/oaslinks.tex}.

\begin{optionlist}
\optitem[auto {\normalfont(\optval{none})}]{openaccess}{\optval{auto},
  \optval{doiapi}, \optval{forcedoiapi}, \optval{true}, \optval{false},
  \optval{none}, \prm{open access field}}

This option controls how the best open access URL is detected.
The initial value is \optval{auto} unless the package is loaded with the
\hyperopt{enable} option set to \optval{false} in which case the initial
value changes to \optval{none}.

All of the following values are valid at all three levels (global,
per-type and per-entry), but not all options may be useful at all
available levels.
\begin{valuelist}
\item[auto] Use a semi-automatic heuristic to find the best open access URL.

\item[all] Assume that all known \enquote{open access fields} contain an open
  access URL.

  Note that an entry still needs at least one field that open access field
  for the open access symbol to show up.

\item[none] Assume that no known \enquote{open access field} contains an open
  access URL.

  The value \optval{none} may be used to disable the \opt{openaccess} feature.

\item[doiapi] Use API queries from the \sty{biblatex-ext-oa-doiapi} package
  to determine the best open access link from the DOI.

  \begin{warnbox}
  This option requires that the \hyperopt{doiapi} package option be
  set to \optval{true} and only works with \lualatex.
  Furthermore it needs a DOI in the \bibfield{doi} field.
  If the \bibfield{doi} field is empty, this option falls back to \optval{auto}.
  \end{warnbox}

\item[onlydoiapi] Like \optval{doiapi}, but does not fall back to \optval{auto}
  in case no DOI is present.

\item[forcedoiapi] Like \optval{doiapi}, but overrides more specific option
  levels that would say otherwise.

  By default more specific options take precedence over more general options,
  that means that a per-entry option trumps a per-type option, which in turn
  overrides the global setting.
  If \optval{forcedoiapi} is set it will prevail over more specific options that
  would say otherwise. A common use case is to override per-entry options with
  a global \optval{forcedoiapi}.

\item[forceonlydoiapi] A combination of \optval{onlydoiapi} and
  \optval{forcedoiapi}.
  This options overrides all more specific values and does not fall back to
  \optval{auto} in case the DOI is not given.

\item[true] An alias for \optval{all}.
\item[false] An alias for \optval{none}.

\item[\prm{open access field}] Specifies that the open access version is given
  by \prm{open access field}.
  This value would normally be used only on a per-entry level.

  Any field for which \hypercmd{DeclareOpenAccessFieldUrl} has declared a URL
  specification is a valid \enquote{open access field}.
\end{valuelist}

\end{optionlist}

The heuristics to find the best open access link are governed
by the following commands and rely on information present in the \filename{.bib}
file (ideally in field annotations).

\begin{ltxsyntax}
\cmditem{DeclareOpenAccessFieldUrl}[openness]{field}{url spec}

Declares \prm{field} to be an \enquote{open access field}.
The code for \prm{url spec} should expand to the full URL under which the entry
can be accessed.
The optional \prm{openness} argument indicates whether the URL will always be
open access (\texttt{always}) or whether it is only open access under certain
conditions (\texttt{conditional});
if it is not present \texttt{conditional} is assumed.

If the package is loaded with \opt{enable} set to \optval{true},
the following declarations are active.
\begin{biblatexcode}
\DeclareOpenAccessFieldUrl{url}{\thefield{url}}
\DeclareOpenAccessFieldUrl{doi}{https://doi.org/\thefield{doi}}
\end{biblatexcode}

\cmditem{UndeclareOpenAccessFieldUrl}{field}

Disable \prm{field} as \enquote{open access field} and remove its mapping
to a URL.

\cmditem{DeclareOpenAccessEprintUrl}[openness]{eprinttype}{url spec}

Declares the \bibfield{eprinttype} \prm{eprinttype} to be an
\enquote{open access eprint type}.
The code for \prm{url spec} should expand to the full URL under which the entry
can be accessed.

The optional \prm{openness} argument indicates whether the URL will always be
open access (\texttt{always}) or whether it is only open access under certain
conditions (\texttt{conditional});
if it is not present \texttt{conditional} is assumed.

The \texttt{always} decision can be overridden on a per-entry basis by
giving the field annotation \texttt{closedaccess}.
If a field has openness \texttt{conditional} it is only considered for a
particular entry if the field annotation \texttt{openaccess} is set.

If the package is loaded with \opt{enable} set to \optval{true},
the following declarations are active.
\begin{biblatexcode}
\DeclareOpenAccessEprintUrl{jstor}{%
  http://www.jstor.org/stable/\thefield{eprint}}
\DeclareOpenAccessEprintUrl{hdl}{%
  http://hdl.handle.net/\thefield{eprint}}
\DeclareOpenAccessEprintUrl{pubmed}{%
  http://www.ncbi.nlm.nih.gov/pubmed/\thefield{eprint}}
\DeclareOpenAccessEprintUrl[always]{arxiv}{%
  https://arxiv.org/\abx@arxivpath/\thefield{eprint}}
\end{biblatexcode}
Since articles on the arXiv are always open access, the declaration
for \bibfield{arxiv} features the optional \prm{openness} argument
\texttt{always}.

\cmditem{DeclareOpenAccessEprintAlias}{alias}{eprinttype}

Handle the open access URL detection for \prm{alias} exactly as that for
\prm{eprinttype}.

If the package is loaded with \opt{enable} set to \optval{true},
the following declarations are active.
\begin{biblatexcode}
\DeclareOpenAccessEprintAlias{JSTOR}{jstor}
\DeclareOpenAccessEprintAlias{HDL}{hdl}
\DeclareOpenAccessEprintAlias{PubMed}{pubmed}
\DeclareOpenAccessEprintAlias{arXiv}{arxiv}
\end{biblatexcode}

\cmditem{UndeclareOpenAccessEprintUrl}{eprinttype}

Disable \prm{eprinttype} as \enquote{open access eprint type} and remove its
mapping to a URL.

\cmditem{DeclareOpenAccessUrlFieldPriority}{fields}

Define the priority order of \enquote{open access fields} if several fields
contain valid open access URLs.
The fields processed in the given order and the last field prevails, meaning
that fields later in the list have higher priority.

The default definition if the package is active is
\begin{biblatexcode}
\DeclareOpenAccessUrlFieldPriority{url,eprint,doi}
\end{biblatexcode}
which means that the DOI is given preference over eprints, which in turn
prevail over the \bibfield{url} field.
\end{ltxsyntax}

The \optval{auto} detection works as follows.
\begin{itemize}
\item The best known open access URL is cleared.
\item For each non-empty field listed in
  \hypercmd{DeclareOpenAccessUrlFieldPriority} the URL specification is
  obtained.
  \begin{itemize}
    \item If the field has \prm{openness} \texttt{always}, the best known open
    access URL is updated with the new value, unless the field has been
    explicitly marked as closed acces with the field annotation
    \texttt{closedaccess}.
    \item If the field has \prm{openness} \texttt{conditional}, the best known
    open access URL is updated with the new value if the field has been marked
    with the field annotation \texttt{openaccess}.
  \end{itemize}
\end{itemize}

For \optval{auto}-detection, an open access article with DOI would
ideally use the field annotation \texttt{openaccess}
(\lstinline[style=extblxstylebibtex]|{doi+an = {=openaccess},|).
% cipriani
\inputexamplebibfile[listing options={linerange={27-38},
                                      style = extblxstylebibtex}]
                    {biblatex-ext-examples.bib}
Preprints on the arXiv, on the other hand, need no field annotation, because
they are always open access.
% meckes
\inputexamplebibfile[listing options={linerange={40-48},
                                      style = extblxstylebibtex}]
                    {biblatex-ext-examples.bib}
\makeatletter
\renewcommand{\extblx@oa@marginmark@do}{%
  \vbox to 0pt{%
    \kern-3\dp\strutbox
    \strut
    \hfill
    \rlap{\kern.6em
      \ifhyperref
        {\href{\thefield{bestoaurl}}}
        {\@firstofone}%
      {\oasymbol}}
    \vss
  }%
}
\makeatother
\begin{bibexample}
\exampleprintbib{cipriani,meckes,sigfridsson}
\end{bibexample}

If the \opt{openaccess} option is set to one specific field only, only
that field will be taken into consideration and it will be assumed that the
field contains an open access link unless the field annotation
\texttt{closedaccess} is set.
In particular the \prm{openness} of each field does not play a role.


\subsection{Open access detection with the Unpaywall API}\label{sec:oa:doiapi}
\label{exltd@itm@biblatex-ext-oa-doiapi}
Inspired by Eric Marsden's%
\fnurl{https://tex.stackexchange.com/users/29621/eric-marsden}
answer to the open access symbol question on TeX.SX, the package
\sty{biblatex-ext-oa-doiapi}, which can be loaded via the \hyperopt{doiapi}
option of \sty{biblatex-ext-oa}, provides means to automatically find the
best open access link to a given article via its DOI by querying the
Unpaywall API\fnurl{https://unpaywall.org/}.

\begin{warnbox}
This package can only be used with \lualatex{} and needs online
access.
Furthermore the Unpaywall API requires an email address and asks to limit
requests to 100,000 calls per day\fnurl{https://unpaywall.org/products/api}.
\end{warnbox}

To avoid too many calls to the API, the package caches its information in the
temporary \filename{.oai} file.
It is safe to delete that file at any point, but if you do so the next
\lualatex{} run may take a while, because all information has to be retrieved
from the APIs.
To avoid old information sticking around for too long you can set a cache
expiration period.

\begin{ltxsyntax}
\cmditem{SetDOIAPIMail}{email}

Define the email address that is going to be used for API queries that require
an email.

\begin{warnbox}
There is no predefined default email address.
If the package is loaded and no email was provided until the
beginning of the document (as tested in an \cmd{AtEndPreamble} hook)
you will receive a warning.
A fatal error will be thrown if you try to query the Unpaywall database
without an email address.
\end{warnbox}

\cmditem{SetDOIAPICacheExpiration}{days}

Set the expiration period for the cache.
If the information for an entry in the cache is older than \prm{days} days
it will be queried again.
The default expiration period is seven days.

\cmditem{IsOpenaccess}{DOI}{true}{false}

Expands to \prm{true} if the DOI is an open access work according to the
Unpaywall database and to \prm{false} otherwise.

\cmditem{GetOpenaccessURLWrapped}{DOI}

Obtain the best open access URL as returned by the Unpaywall API wrapped in
the \cmd{url} macro.

\cmditem{OpenaccessURLisDOI}{DOI}{true}{false}

Expands to \prm{true} if the best open access URL as returned by the Unpaywall
API is the DOI and to \prm{false} otherwise.
\end{ltxsyntax}


\subsection{Open access symbols}\label{sec:oa:symbol}
Currently, this bundle offers three methods of drawing the open access
symbols.
Each method is packed up in a little sub-package, end users are not
expected to load these packages manually, instead they should use
the option~\hyperopt{symbolpackage} of \hypersty{biblatex-ext-oa}.

The package \hypersty{biblatex-ext-oasymb-pict2e} draws the symbols using
\sty{pict2e}'s \env{picture} environment.
The package \hypersty{biblatex-ext-oasymb-tikz}
uses \pgftikz's \env{tikzpicture}.
The package \hypersty{biblatex-ext-oasymb-l3draw} finally
uses \LaTeX3's \sty{l3draw} package.
The main purpose of these packages is to define the macro \cmd{oasymbol}
for use with \sty{biblatex-ext-oa}.

All packages define a similar set of macros, so they are not indended to 
be loaded together.
If used as stand-alone packages, all packages only have one option.

\begin{optionlist}
\optitem[plos]{symbol}{\optval{plos}, \optval{oanet}}

This options determines which of the two available symbols is going to
be used for \hypercmd{oasymbol}.

The possible values are
\begin{valuelist}
\item[plos] Use the open access symbol by PLOS | Public Library of Science%
  \fnurl{https://www.plos.org/} \csuse{extblx@oasymb@pict@plos}.

  The drawing is based on a CC0%
  \fnurl{https://creativecommons.org/publicdomain/zero/1.0/deed.en}
  work by PLoS (now PLOS | Public Library of Science),
  modified by Wikipedia users Nina, Beao, and JakobVoss%
\fnurl{https://commons.wikimedia.org/wiki/File:Open_Access_logo_PLoS_white.svg}.

\item[oanet] Use the open access symbol by open-access.net%
  \fnurl{https://open-access.net/}
  \csuse{extblx@oasymb@pict@oanet}.

  The drawing is based on the logo uploaded to Wikimedia%
  \fnurl{https://commons.wikimedia.org/wiki/File:Open_access.svg}
  and originally from open-access.net%
  \fnurl{https://open-access.net/fileadmin/logos/oa.svg}.
  Wikimedia claims the content does not meet the threshold of originality
  for copyright protection.
  Even if that is not the case, the legal notices on open-access.net%
  \fnurl{https://open-access.net/DE-EN/provider-legal-notices-and-terms/}
  place all contents under a Creative Commons 4.0 Attribution International
  (CC BY 4.0) Licence\fnurl{https://creativecommons.org/licenses/by/4.0/}
  unless stated otherwise.
  The logo designer is listed as
  \emph{Medien + Design\fnurl{http://www.cedis.fu-berlin.de/mediendesign/},
  Center for Digital Systems,
  Competence Center for E-Learning and Multimedia, Freie Universität Berlin}.

  There is no implication that the open-access.net logo is not covered by
  the CC BY 4.0 terms.
  Indeed, the German \enquote{about us} page explicitly states
  \foreignquote{ngerman}{%
    Das Logo der Informationsplattform darf nachgenutzt werden,
    gerne mit einem Link zu open-access.net.}%
  \fnurl{https://open-access.net/ueber-uns/}
  The English version has \enquote{The logo of the information platform
  may be reused with a link to open-access.net}%
  \fnurl{https://open-access.net/DE-EN/about-us/}.
  While the German version leaves some leeway as to whether or not a link is
  required as attribution, the English version seems more explicit.

  That means that if you do not follow Wikimedia in considering the symbol
  not protected by copyright in your~-- and all other relevant jurisdictions%
  ~-- due to its failure to cross the threshold of originality,
  you should probably treat it as CC BY 4.0 and treat the link to
  \url{https://open-access.net} as attribution requested by the Licensor
  according to Section 3(a)(1)(A) of the license.
  If you can't reasonably meet the requirements of the license in your work,
  you may want to consider using the PLOS logo, which is licensed under CC0.
\end{valuelist}
\end{optionlist}

Apart from \hypercmd{oasymbol} the packages only define one other user-level
command.
\begin{ltxsyntax}
\cmditem{DefineOASymbol}[symbol]

This command redefines \hypercmd{oasymbol} to point to a different version
of the symbol.
The possible values for the \prm{symbol} argument are the values of the
\hyperopt{symbol} option.
\end{ltxsyntax}

\subsubsection[\sty{pict2e}]{Open access symbols with \sty{pict2e}}
\label{sec:oa:symbol-pict2e}
\label{exltd@itm@biblatex-ext-oasymb-pict2e}
The main purpose of the package \sty{biblatex-ext-oasymb-pict2e}
is to define the command \hypercmd{oasymbol}.
As mentioned above, usually this package is not going to be loaded as a
stand-alone package, instead it is loaded via
\hypersty[symbolpackage]{biblatex-ext-oa}.

The package requires and loads the packages
\sty{pict2e}\fnurl{https://ctan.org/pkg/pict2e} by Hubert Gäßlein,
Rolf Nie\-praschk and Josef Tkadlec
as well as
\sty{xcolor}\fnurl{https://ctan.org/pkg/xcolor} by Uwe Kern.

If used as a stand-alone package, \sty{biblatex-ext-oasymb-pict2e} has one
load-time option, namely \hyperopt{symbol}, and defines one command besides
\hypercmd{oasymbol}, namely \hypercmd{DefineOASymbol}.
The option determines whether \cmd{oasymbol} is defined to display
the PLOS symbol~(\csuse{extblx@oasymb@pict@plos})
or the open-access.net symbol~(\csuse{extblx@oasymb@pict@oanet}).
The command \hypercmd{DefineOASymbol} can be used to change the definition
of \cmd{oasymbol} on the fly.


\subsubsection[Ti\emph{k}Z]{Open access symbols with Ti\emph{k}Z}
\label{sec:oa:symbol-tikz}
\label{exltd@itm@biblatex-ext-oasymb-tikz}
The main purpose of the package \sty{biblatex-ext-oasymb-tikz}
is to define the command \hypercmd{oasymbol}.
As mentioned above, usually this package is not going to be loaded as a
stand-alone package, instead it is loaded via
\hypersty[symbolpackage]{biblatex-ext-oa}.

This package requires and loads the package
\sty{tikz}\fnurl{https://ctan.org/pkg/pgf} by Till Tantau as well as all
its dependencies.

If used as a stand-alone package, \sty{biblatex-ext-oasymb-tikz} has one
load-time option, namely \hyperopt{symbol}, and defines one command besides
\hypercmd{oasymbol}, namely \hypercmd{DefineOASymbol}.
The option determines whether \cmd{oasymbol} is defined to display
the PLOS symbol~(\csuse{extblx@oasymb@tikz@plos})
or the open-access.net symbol~(\csuse{extblx@oasymb@tikz@oanet}).
The command \hypercmd{DefineOASymbol} can be used to change the definition
of \cmd{oasymbol} on the fly.

This package used to be called \sty{biblatex-ext-oa-tikzsymbols},
for consistency reasons it has been renamed to \sty{biblatex-ext-oasymb-tikz}.

The command \cmd{DefineTikZOASymbol} has been renamed to the more generic
\hypercmd{DefineOASymbol}.
Backwards compatibility code for that change is in place.


\subsubsection[\sty{l3draw}]{Open access symbols with \sty{l3draw}}
\label{sec:oa:symbol-l3draw}
\label{exltd@itm@biblatex-ext-oasymb-l3draw}
The main purpose of the package \sty{biblatex-ext-oasymb-l3draw}
is to define the command \hypercmd{oasymbol}.
As mentioned above, usually this package is not going to be loaded as a
stand-alone package, instead it is loaded via
\hypersty[symbolpackage]{biblatex-ext-oa}.

This package requires and loads the full force of \LaTeX3.
Aside from the main package \sty{expl3}\fnurl{https://ctan.org/pkg/expl3} it
loads \sty{xparse}\fnurl{https://ctan.org/pkg/xparse},
and \sty{l3keys2e}\fnurl{https://ctan.org/pkg/l3keys2e}.
Drawing is performed with the (still experimental)
\sty{l3draw}\fnurl{https://ctan.org/pkg/l3experimental} and
\sty{l3color}\fnurl{https://ctan.org/pkg/l3experimental} packages.

If used as a stand-alone package, \sty{biblatex-ext-oasymb-l3draw} has one
load-time option, namely \hyperopt{symbol}, and defines one command besides
\hypercmd{oasymbol}, namely \hypercmd{DefineOASymbol}.
The option determines whether \cmd{oasymbol} is defined to display
the PLOS symbol~(\csuse{extblx@oasymb@draw@plos})
or the open-access.net symbol~(\csuse{extblx@oasymb@draw@oanet}).
The command \hypercmd{DefineOASymbol} can be used to change the definition
of \cmd{oasymbol} on the fly.




\clearpage
\section{Revision History}\label{sec:log}
The GitHub repository of this project uses release tags, so you can compare
the changes in source code there.\footnote{\url{\gitbaseurl/compare/}}
See also \filename{CHANGES.md}.

The numbers on the right indicate the relevant section of the manual.
\begin{changelog}
\begin{release}{<version>}{<date>}
  \item Fixed default value of \hypercmd{servoldelim}\see{sec:opt:punct}
\end{release}
\begin{release}{0.8}{2019-08-25}
  \item Added \hypercmd{AtIntrocite} \see{sec:opt:style}
  \item Require \sty{biblatex} v3.13 (2010/08/17)\see{sec:req}
\end{release}
\begin{release}{0.7a}{2019-04-04}
  \item Fixed bug in the \opt{env} option of \hypercmd{printbibtabular}%
    \see{sec:tabularbib}
\end{release}
\begin{release}{0.7}{2019-02-08}
  \item Added outer citation delimiters for \cmd{supercite}%
    \see{sec:opt:citedelims}
  \item Added \hypersty{biblatex-ext-oasymb-pict2e}\see{sec:oa:symbol-pict2e}
  \item Added \hypersty{biblatex-ext-oasymb-l3draw}\see{sec:oa:symbol-l3draw}
  \item Renamed  \sty{biblatex-ext-oa-tikzsymbols} to
    \hypersty{biblatex-ext-oasymb-tikz}\see{sec:oa:symbol-tikz}
  \item Deprecated option \opt{tikzsymbol} of \hypersty{biblatex-ext-oa} in
    favour of \hyperopt{symbolpackage} and \hyperopt{symbol}\see{sec:oa}
  \item Renamed and generalised \cmd{DefineTikZOASymbol} to
    \hypercmd{DefineOASymbol}\see{sec:oa:symbol}
  \item Renamed and generalised \cmd{LoadTikZOASymbol} to
    \hypercmd{LoadOASymbolPackage}\see{sec:oa}
\end{release}
\begin{release}{0.6b}{2019-01-11}
  \item Fixed issue loading \texttt{socket.url} module in
    \hypersty{biblatex-ext-oa-doiapi}\see{sec:oa:doiapi}
\end{release}
\begin{release}{0.6a}{2019-01-04}
  \item Minor improvements for \hypersty{biblatex-ext-oa}\see{sec:oa}
\end{release}
\begin{release}{0.6}{2018-11-23}
  \item Require \sty{biblatex} 3.12
  \item Made \hyperopt{dashed} and \hyperopt{introcite} options available on a
    per-entry and per-type basis\see{sec:opt:style}
  \item Added \hypercmd{titleaddonpunct}\see{sec:opt:punct}
  \item Added \hypersty{biblatex-ext-oa} package for open access detection and
    symbols\see{sec:oa}
\end{release}
\begin{release}{0.5a}{2018-10-09}
  \item Fixed a bug with \cmd{footcites} and \cmd{footcitetexts} in
    \blxstyle{ext-authoryear}
\end{release}
\begin{release}{0.5}{2018-10-02}
  \item New \hyperblxstyle[ext-authoryear-ecomp]{ext-authoryear-*ecomp} style
    subfamily\see{sec:styles:new}
  \item Added outer cite delimiters for \cmd{footcite}\see{sec:opt:citedelims}
  \item Added experimental tabular bibliography\see{sec:tabularbib}
\end{release}
\begin{release}{0.4}{2018-06-10}
  \item Renamed \cmd{DeclareOuterCiteDelim} and friends to
    \hypercmd{DeclareOuterCiteDelims}\see{sec:opt:citedelims}
  \item Added \hypercmd{RegisterCiteDelims}\see{sec:opt:citedelims}
  \item Renamed \enquote{virtual} cite command \cmd{bbx:introcite} to
    \cmd{bbx@introcite}\see{sec:opt:style}
  \item Renamed \opt{citeinxref} to \hyperopt{citexref}\see{sec:opt}
\end{release}
\begin{release}{0.3}{2018-06-04}
  \item Added \hyperbibfieldformat[titlecase:title]{titlecase:\dots title}
    field formats\see{sec:opt:field}
  \item Added \hypersty{bbx:introcite:plain:keeprelated} toggle%
    \see{sec:opt:style}
  \item Added \hypercmd{jourvoldelim}, \hypercmd{jourserdelim} and
    \hypercmd{servoldelim}\see{sec:opt:punct}
  \item Improve documentation
\end{release}
\begin{release}{0.2}{2018-03-28}
  \item Rework lengths for \hyperkvopt{introcite}{label}\see{sec:opt:style}
  \item Fixed meaning of \hypercmd{maintitletitledelim}\see{sec:opt:punct}
  \item Added \hypercmd{voltitledelim}\see{sec:opt:punct}
\end{release}
\begin{release}{0.1a}{2018-03-20}
  \item Fixed inner citation delimiters for \hyperblxstyle{ext-authoryear},
    \hyperblxstyle{ext-authortitle} and their \blxstyle{-ibid} versions%
    \see{sec:opt:citedelims}
  \item Fixed \cmd{smartcite} delimiters\see{sec:opt:citedelims}
\end{release}
\begin{release}{0.1}{2018-03-18}
  \item First public release
\end{release}
\end{changelog}

\begin{bibexample}[breakable]
\printbibliography[heading=bibintoc, title={Example \refname}]
\end{bibexample}

\clearpage
\appendix
\section{Punctuation Commands in Use}\label{sec:punctinuse}
The following example bibliography shows some common punctuation commands
in use.
Commands provided by standard \sty{biblatex} are marked in
\textit{\cmd{italics}}, new commands defined by \sty{biblatex-ext} are in
\textbf{\cmd{bold}} and linked to their documentation.
\begin{bibexample}[title={Punctuation and delimiters
  (defined by \textit{standard \sty{biblatex}} and
  \textbf{\sty{biblatex-ext}})},
  top=6mm,bottom=4mm,remember,
  overlay={
    % geer
    \punctarrow*{nameyeardelim}{-.8mm}{6mm}
    \punctarrow*{nametitledelim}{2.5mm}{6mm}
    % companion
    \punctarrow*{multinamedelim}{-2.5mm}{6mm}
    \punctarrow*{finalnamdelim}{2.5mm}{4mm}
    \punctarrow{locpubdelim}{-2.5mm}{-5mm}
    \punctarrow{pubdatedelim}{2.5mm}{-4mm}
    % knuth:ct:a
    \punctarrow{maintitletitledelim}{-2.5mm}{-.8cm}
    \punctarrow{voltitledelim}{2.5mm}{-4mm}
    % moore
    \punctarrow*{intitlepunct}{-4mm}{-4mm}
    \punctarrow*{bibpagespunct}{-2.5mm}{-4mm}
    \punctarrow*{bibrangedash}{.2cm}{-4mm}
    % sigfridsson
    \punctarrow{jourvoldelim}{-.2cm}{-4mm}
    \punctarrow{volnumdelim}{.2cm}{-4mm}
    % vizedom:related
    \punctarrow*{translatortypedelim}{2.5mm}{6mm}
    \punctarrow*{begrelateddelim}{.8cm}{-4mm}
    % westfahl:frontier
    \punctarrow*{editortypedelim}{2.5mm}{6mm}
    \punctarrow*{newunitpunct}{-.2cm}{-4mm}
    % westfahl:space
    \punctarrow*{bibnamedash}{2.5mm}{5mm}
    \punctarrow*{subtitlepunct}{2.5mm}{5mm}
  }
]
\frenchspacing
\list{}
  {\setlength{\leftmargin}{\bibhang}%
   \setlength{\itemindent}{-\leftmargin}%
   \setlength{\itemsep}{8mm}%
   \setlength{\parsep}{\bibparsep}}
\item\nocite{geer}
  Geer, Ingrid de\tikztextmark{nameyeardelim}{\textvisiblespace}(1985)%
  \tikztextmark{nametitledelim}{.} \enquote{Earl, Saint, Bishop, Skald --
  and Music. The Orkney Earldom of the Twelfth Century. A Musicological Study}.
  PhD thesis. Uppsala: Uppsala Universitet.

\item\nocite{companion}
  Goossens, Michel\tikztextmark{multinamedelim}{,} Frank Mittelbach
  \tikztextmark{finalnamdelim}{and} Alexander Samarin (1994).
  \emph{The LaTeX Companion}. 1st ed. Reading, Mass.%
  \tikztextmark{locpubdelim}{:} Addison-Wesley\tikztextmark{pubdatedelim}{,}
  1994. 528 pp.

\item\nocite{knuth:ct:a}
  Knuth, Donald E. (1984). \emph{Computers \& Typesetting}%
  \tikztextmark{maintitletitledelim}{.} Vol. A\tikztextmark{voltitledelim}{:}
  \emph{The \TeX{} book}. Reading, Mass.: Addison-Wesley, 1984.

\item\nocite{moore}
  Moore, Gordon E. (1965). \enquote{Cramming more components onto integrated
  circuits}. In\tikztextmark{intitlepunct}{:} \emph{Electronics} 38.8%
  \tikztextmark{bibpagespunct}{,} pp. 114\tikztextmark{bibrangedash}{--}117.

\item\nocite{sigfridsson}
  Sigfridsson, Emma and Ulf Ryde (1998). \enquote{Comparison of methods for
  deriving atomic charges from the electrostatic potential and moments}.
  In: \emph{Journal of Computational
  Chemistry}\tikztextmark{jourvoldelim}{\textvisiblespace}19%
  \tikztextmark{volnumdelim}{.}4 (1998), pp. 377--395.

\item\nocite{vizedom:related}
  Vizedom, Monika B. and Gabrielle L. Cafee%
  \tikztextmark{translatortypedelim}{,} trans. (1960).
  \emph{The Rites of Passage.} University of Chicago Press%
  \tikztextmark{begrelateddelim}{.} Trans. of Arnold van Gennep.
  \emph{Les rites de passage.}
  Paris: Nourry, 1909.

\item\nocite{westfahl:frontier}
  Westfahl, Gary\tikztextmark{editortypedelim}{,} ed. (2000a).
  \emph{Space and Beyond. The Frontier Theme in Science Fiction}%
  \tikztextmark{newunitpunct}{.} Westport, Conn. and London: Greenwood, 2000.

\item\nocite{westfahl:space}
  \tikztextmark{bibnamedash}{--} (2000b). \enquote{The True Frontier%
  \tikztextmark{subtitlepunct}{.} Confronting and Avoiding the Realities of
  Space in American Science Fiction Films}. In: \emph{Space and Beyond.
  The Frontier Theme in Science Fiction}. Ed. by Gary Westfahl. Westport, Conn.
  and London: Greenwood, 2000, pp. 55--65.
\endlist
\end{bibexample}

\end{document}
\endinput
