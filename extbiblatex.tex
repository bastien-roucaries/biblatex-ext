\documentclass{ltxdockit}
\usepackage{btxdockit}
\usepackage[utf8]{inputenc}
\usepackage[british]{babel}
\usepackage[strict]{csquotes}
\usepackage{libertine}
\usepackage[scaled=0.8]{beramono}
\usepackage{microtype}
\usepackage[style=ext-authoryear-icomp, backend=biber]{biblatex}
\addbibresource{biblatex-examples.bib}

\makeatletter
\blx@inputonce{ext-biblatex-aux.def}{auxiliary code for ext-biblatex}{}{}{}{}
\newcommand*{\biblatexversion}{\extblx@requiredbiblatexversion}
\makeatother

\newcommand*{\gitbaseurl}{https://github.com/moewew/extbiblatex}
\newcommand*{\extbiblatexversion}{0.1.0}
\newcommand*{\biber}{\sty{biber}}
  
\newrobustcmd*{\Deprecated}{%
  \textcolor{spot}{\margnotefont Deprecated}}
\newrobustcmd*{\DeprecatedMark}{%
  \leavevmode\marginpar{\Deprecated}}
\newcommand*{\gitissuelink}[1]{%
  \href{\gitbaseurl/issues/#1}{issue \##1 on github}}

\titlepage{%
  title={The \sty{extbiblatex} Package},
  subtitle={Traditional bibliography styles for \sty{biblatex}},
  url={\gitbaseurl},
  author={Moritz Wemheuer},
  email={\null},%
  revision={\extbiblatexversion},
  date={\today}}

\hypersetup{%
  pdftitle={The extbiblatex Package},
  pdfsubject={Traditional bibliography styles for biblatex},
  pdfauthor={Moritz Wemheuer},
  pdfkeywords={tex, latex, biblatex, bibtex, bibliography, references, citation}}

\hyphenation{%
  star-red
  bib-lio-gra-phy
  white-space
}

\begin{document}

\printtitlepage
\tableofcontents


\section{Introduction}\label{sec:int}
The \sty{extbiblatex} package provides an extended version of the standard
styles that come with \sty{biblatex}.
For each standard style this package provides a style with the same name
prefixed with \texttt{ext-} which can be used as a drop-in replacement for the
standard style -- for example, the replacement for \texttt{authoryear-icomp}
is called \texttt{ext-authoryear-icomp}.

The aim of this package is to offer a simple interface to change some of the
stylistic decisions made for the standard styles.
Additionally, some customisation features that were not deemed appropriate
for inclusion in the \sty{biblatex} kernel are provided.

\subsection{Requirements}
The use of the styles requires a current version of the \sty{biblatex} package.
At the time of writing the latest version of \sty{biblatex} is \biblatexversion,
that version is required for the styles to work properly.
A warning will be issued if you use an older version; you may choose to ignore
that warning, but the style can not be guaranteed to work properly in that case:
you might get other warnings, errors, or the style might fail silently.

Use of the \biber{} backend is strongly encouraged.
No features of this package require \biber{} explicitly in their implementation,
but many \sty{biblatex} standard features require \biber{} to function properly.

\subsection{License}
Permission is granted to copy, distribute and\slash or modify this software
under the terms of the \lppl, version 1.3c.%
\fnurl{https://www.latex-project.org/lppl.txt}
The current maintainer of this work is Moritz Wemheuer (\textcopyright 2017--2018).


\subsection{Feedback}\label{subsec:int:feb}

Please use the \sty{extbiblatex} project page on GitHub to report bugs and
submit feature requests.\footnote{\url{\gitbaseurl}}

If you do not want to report a bug or request a feature, but are simply in need
of assistance, you might want to consider posting your question on the
\texttt{comp.text.tex} newsgroup or \tex{} -- \latex Stack Exchange.%
\fnurl{https://tex.stackexchange.com/questions/tagged/biblatex}

\section{Usage}

\sty{extbiblatex} is not a standalone package, instead it is a collection of
\sty{biblatex} style files.

You can load the styles exactly as you would load the standard styles:
\begin{lstlisting}[style=latex,escapeinside={(*@}{@*)}]{}
\usepackage[style=(*@$\langle$\normalfont\emph{style}$\rangle$@*)]{biblatex}
\end{lstlisting}

The naming of the styles follows the scheme
\mbox{\texttt{ext-}$\langle$\emph{standard style}$\rangle$}, e.g.,
the style corresponding to \texttt{authoryear-icomp} is called
\texttt{ext-authoryear-icomp}.

\subsection{Styles}
Please refer to the \sty{biblatex} documentation%
\fnurl{http://mirrors.ctan.org/macros/latex/contrib/biblatex/doc/biblatex.pdf}
and the style examples%
\fnurl{http://mirrors.ctan.org/macros/latex/contrib/biblatex/doc/examples/}
for a more detailed description of each standard style.
The relations between the styles are exactly as in their standard counterparts.
The \texttt{ext-} styles only build on top of the standard files.
\begin{marglist}
\item[ext-numeric]
An extended version of the standard \texttt{numeric} style for citations with
numeric labels: \enquote{[1, 2, 3, 5, 7]}.

\item[ext-numeric-comp]
An extended version of the standard \texttt{numeric-comp} style.
The style is similar to \texttt{ext-numeric}, but citation labels are
compressed to give \enquote{[1--3, 5--7]} instead of
\enquote{[1, 2, 3, 5, 6, 7]}.

\item[ext-numeric-verb]
An extended version of the standard \texttt{numeric-verb} style.
This style is similar to the \texttt{numeric} style, but each citation label
is in its own set of brackets: \enquote{[1], [2], [3]}.

\item[ext-alphabetic]
An extended version of the standard \texttt{alphabetic} style for citations with
alphabetic labels derived from author name and year: \enquote{[SR98, Knu84c]}.

\item[ext-alphabetic-verb]
An extended version of the standard \texttt{alphabetic-verb} style.
This style is based on \texttt{ext-alphabetic}, but like
\texttt{ext-numeric-verb} places each citation label in its own pair of
brackets: \enquote{[SR98], [Knu84c]}.

\item[ext-authoryear]
An extended version of the standard \texttt{authoryear} style for citations
using author name and year: \enquote{Sigfridsson and Ryde 1998}.

\item[ext-authoryear-comp]
An extended version of the standard \texttt{authoryear-comp} style.
The style is based on the author-year citations of \texttt{ext-authoryear},
but several works by the same author are compressed by not displaying the
author multiple times: \enquote{Knuth 1984, 1986} instead of
\enquote{Knuth 1984; Knuth 1986}.

\item[ext-authoryear-ibid]
An extended version of the standard \texttt{authoryear-ibid} style.
This style is similar to \texttt{ext-authoryear}, but repeated citations are
replaced with \enquote{ibidem}.

\item[ext-authoryear-icomp]
An extended version of the standard \texttt{authoryear-icomp} style.
This style combines the two styles \texttt{ext-authoryear-comp} and
\texttt{ext-authoryear-ibid}.

\item[ext-authortitle]
An extended version of the standard \texttt{authortitle} style for citations
by author and title: \enquote{Maron, \emph{Animal Triste}}.

\item[ext-authortitle-comp]
An extended version of the standard \texttt{authortitle-comp} style.
Based on \texttt{ext-authortitle} this style compresses several citations
by the same author just like \texttt{ext-authoryear-comp}:
\enquote{Aristotle, \emph{Physics}, \emph{Poetics}} instead of
\enquote{Aristotle, \emph{Physics}; Aristotle, \emph{Poetics}}.

\item[ext-authortitle-ibid]
An extended version of the standard \texttt{authortitle-ibid} style.
This style is similar to \texttt{ext-authortitle}, but replaces repeated
citations of the same work with \enquote{ibidem}.

\item[ext-authortitle-icomp]
An extended version of the standard \texttt{authortitle-icomp} style.
This style combines \texttt{ext-authortitle-comp} and
\texttt{ext-authortitle-ibid}.

\item[ext-authortitle-terse]
An extended version of the standard \texttt{authortitle-terse} style.
This style is similar to \texttt{ext-authortitle}, but the title is omitted in
citations if there is only one work by the relevant author.

\item[ext-authortitle-tcomp]
An extended version of the standard \texttt{authortitle-tcomp} style.
This style combines \texttt{ext-authortitle-terse} and
\texttt{ext-authortitle-comp}.

\item[ext-authortitle-ticomp]
An extended version of the standard \texttt{authortitle-ticomp} style.
This style combines \texttt{ext-authortitle-terse},
\texttt{ext-authortitle-comp} and \texttt{ext-authortitle-ibid}.


\item[ext-verbose]
An extended version of the standard \texttt{verbose} style.
This style shows the full bibliographic reference the first time a work is
cited.

\item[ext-verbose-ibid]
An extended version of the standard \texttt{verbose-ibid} style.
Based on \texttt{ext-verbose}, repeated citations to the same work are replaced
with \enquote{ibidem}.

\item[ext-verbose-note]
An extended version of the standard \texttt{verbose-note} style.
Based on \texttt{ext-verbose} and intended for use in footnotes, subsequent
citations link back to the footnote the entry was cited at first and in full.

\item[ext-verbose-inote]
An extended version of the standard \texttt{verbose-inote} style.
Similar to \texttt{verbose-note}, but repeated citations to the same work are
replaced with \enquote{ibidem}.

\item[ext-verbose-trad1]
An extended version of the standard \texttt{verbose-trad1} style.
This style makes extensive use of scholarly abbreviations and is otherwise
similar to \texttt{ext-verbose}.

\item[ext-verbose-trad2]
An extended version of the standard \texttt{verbose-trad2} style.
The style is similar to \texttt{ext-\hspace{0pt}verbose-\hspace{0pt}inote} and
uses scholarly abbreviations to shorten citations.

\item[ext-verbose-trad3]
An extended version of the standard \texttt{verbose-trad3} style.
This style is similar to \texttt{ext-\hspace{0pt}verbose-\hspace{0pt}trad2}.
\end{marglist}


\subsection{Options}
All options of the \sty{biblatex} package are supported and each style supports
the options of its standard counterpart.

Additionally all styles support the following options in global, per-type and
per-entry scope.
The default values are such that the styles can be used as drop-in replacement
for the standard files without significant changes in output.

\begin{optionlist}
\boolitem[true]{articlein}
Whether or not to display \enquote{in:} before the journal information in
\bibtype{article} entries.

\boolitem[false]{innamebeforetitle}
Whether or not the \bibfield{editor} should be moved before the
\bibfield{booktitle} field for \bibtype{inbook}, \bibtype{incollection} and
\bibtype{inproceedings} entries.

\boolitem[false]{innameidem}
Whether or not the \bibfield{editor} of the \bibfield{booktitle}
for the entry types \bibtype{inbook}, \bibtype{incollection} and
\bibtype{inproceedings} is replaced by \enquote{idem} in case the
\bibfield{editor} and \bibfield{author} name lists coincide.

\boolitem[false]{inxref}
This option controls if \bibtype{inbook}, \bibtype{incollection} and
\bibtype{inproceedings} entries that are tied to a parent entry with
\bibfield{xref} or \bibfield{crossref} should cite their parent in the
bibliography if the parent is listed in the bibliography as an entry.
With the default setting \kvopt{inxref}{false} the parent is not cited, the
information is shown as in the standard styles. If the option is set to 
\texttt{true} the block following the \enquote{in:} is replaced by a citation
to the parent entry.
\end{optionlist}

\subsection{Field Formats}



\DeclareFieldFormat{issuedate}{\mkbibparens{#1}}

\subsection{Punctuation and Delimiters}
The package provides the following commands to modify the delimiters and
punctuation between fields.
\begin{ltxsyntax}
\csitem{volnumdatedelim}
The delimiter between the \bibfield{volume}, \bibfield{number} block and the
date information for \bibtype{article}.
The default is a space.

\csitem{volnumdelim}
The delimiter between \bibfield{volume} and \bibfield{number} for
\bibtype{article}.
The default is a dot.

\csitem{sernumdelim}
The delimiter between \bibfield{series} and \bibfield{number}.
The default is a space.

\csitem{locdatedelim}
The delimiter between \bibfield{location} and \bibfield{date}.
The default is a comma followed by a space.

\csitem{locpubdelim}
The delimiter between \bibfield{location} and \bibfield{publisher}/%
\bibfield{organization}/\bibfield{institution}.
The default is a colon followed by a space.

\csitem{publocdelim}
The delimiter between \bibfield{publisher}/\bibfield{organization}/%
\bibfield{institution} and \bibfield{location}.
The default is a comma followed by a space.

\csitem{pubdatedelim}
The delimiter between \bibfield{publisher}/\bibfield{organization}/%
\bibfield{institution} and \bibfield{date}.
The default is a comma followed by a space.

\csitem{extradateonlycompcitedelim}
Similar to \cmd{compcitedelim}, but indended for use between compressed
citations where the second is an \bibfield{extradate} only.
The default is a comma (\emph{not} followed by a space).
\end{ltxsyntax}

\subsection{Delimiters for Citation Commands}
The citation commands \cmd{cite}, \cmd{parencite} and \cmd{textcite} come with
two sets of delimiters: A pair of \emph{outer delimiters} wrapped around the
resulting citation in its entirety and a pair of \emph{inner delimiters} that
sets off certains bits of the citation label from other information.
\cmd{footcite} only supports the inner delimiters.

An example for outer delimiters would be the round brackets of \cmd{parencite}
for \sty{authoryear}-like styles or the square brackets of \cmd{cite} for 
\sty{numeric}- or \sty{alphabetic}-like styles.
Inner delimiters would be the round brackets in \cmd{textcite} around
the year for \sty{authoryear} or around the title for \sty{authortitle}.

\begin{ltxsyntax}
\cmditem{DeclarePairedCiteOuterDelim}{cite command}{opening delimiter}
        {closing delimiter}

Sets up the outer delimiters for the citation command
\cmd{$\langle$\emph{cite command}$\rangle$}.

For most styles \cmd{parencite} would be set up with
\begin{ltxexample}
\DeclarePairedCiteOuterDelim{parencite}{\bibopenparen}{\bibcloseparen}
\end{ltxexample}

\cmditem{DeclarePairedCiteOuterDelimAlias}{cite alias}{cite command}
\cmditem*{DeclarePairedCiteOuterDelimAlias*}{cite alias}{cite command}

Use the outer delimiters of \cmd{$\langle$\emph{cite command}$\rangle$} for
\cmd{$\langle$\emph{cite alias}$\rangle$} as well.
The unstarred version uses \cmd{def} assignment while the starred version uses
\cmd{let}. This means that the starred version copies the values of the
definitions at the time of executing the aliasing command,
whereas the alias created by the unstarred version will only evaluate the
delimiters whenever the citation command is called.


\cmditem{DeclarePairedCiteInnerDelim}{cite command}{opening delimiter}
        {closing delimiter}

Sets up the inner delimiters for the citation command
\cmd{$\langle$\emph{cite command}$\rangle$}.

For most styles \cmd{textcite} would be set up with
\begin{ltxexample}
\DeclarePairedCiteInnerDelim{textcite}{\bibopenparen}{\bibcloseparen}
\end{ltxexample}

\cmditem{DeclarePairedCiteInnerDelimAlias}{cite alias}{cite command}
\cmditem*{DeclarePairedCiteInnerDelimAlias*}{cite alias}{cite command}

Use the inner delimiters of \cmd{$\langle$\emph{cite command}$\rangle$} for
\cmd{$\langle$\emph{cite alias}$\rangle$} as well.
The unstarred version uses \cmd{def} assignment while the starred version uses
\cmd{let}. This means that the starred version copies the values of the
definitions at the time of executing the aliasing command,
whereas the alias created by the unstarred version will only evaluate the
delimiters whenever the citation command is called.
\end{ltxsyntax}

\section{Revision History}\label{apx:log}
\begin{changelog}
\begin{release}{0.1.0}{2018-02-02}
\item First release.
\end{release}
\end{changelog}
\end{document}

\endinput
