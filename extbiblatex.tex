\documentclass[DIV=9]{scrartcl}
\usepackage{ifxetex,ifluatex}
\newif\iffontspec
\ifxetex
  \fontspectrue
\else\ifluatex
  \fontspectrue
\else
  \fontspecfalse
\fi\fi
\iffontspec
  \usepackage{fontspec}
\else
  \usepackage[T1]{fontenc}
  \usepackage[utf8]{inputenc}
\fi
\usepackage[british]{babel}
\usepackage{csquotes}
\usepackage[useregional]{datetime2}
\usepackage{lmodern}
\usepackage[mono=false]{libertine}
\usepackage[libertine]{newtxmath}
\iffontspec
  \setmonofont[Scale=0.78]{Bitstream Vera Sans Mono}
\else
  \usepackage[scaled=0.78]{beramono}
\fi
\iffalse
\iffontspec
  \setmonofont[Scale=0.9387,StylisticSet=1,AutoFakeSlant]{Inconsolatazi4}
\else
  \usepackage[scale=0.9387]{zi4}
\fi
\usepackage[scaled=0.809]{GoMono}
\usepackage[scaled=0.814]{FiraMono}
\fi
\usepackage{microtype}
\usepackage{hyphenat}
\usepackage{bm}
\usepackage{multicol}
\usepackage[svgnames]{xcolor}
\usepackage[listings, breakable]{tcolorbox}%
\usepackage[style=ext-authoryear-icomp, backend=biber]{biblatex}
\addbibresource{biblatex-examples.bib}
\addbibresource{biblatex-ext-examples.bib}

\makeatletter
\def\newtextvisiblespace{%
  \raisebox{-1.2pt}{%
    \mbox{\kern.06em\vrule \@height.3ex \@width.14ex}%
    \vbox{\hrule \@width.3em \@height.14ex}%
    \hbox{\vrule \@height.3ex \@width.14ex}%
    \kern.06em}}
\makeatother


\usepackage{ltxdockit}
\usepackage{btxdockit}
\hypersetup{%
  colorlinks=true,
  allcolors=spot,
  bookmarksopen=false,
  bookmarksnumbered=false,
  plainpages=false}

\definecolor{highlight1}{RGB}{240, 0, 0}%{240,0,0}%{213,94,0}%{217,95,2}%{240,0,0}
\definecolor{highlight2}{RGB}{0,153, 153}%{50, 205, 50}%{50, 205, 50}%{45, 185, 45}%{0,158,115}%{27,158,119}%{34,139,34}
\definecolor{spot}{rgb}{0,0.2,0.6}%{RGB}{0,114,178}%{rgb}{0,0.2,0.6}
%{named}{DarkRed}%{rgb}{0,0.2,0.6}%{named}{MediumBlue}

\lstdefinestyle{extblxstyle}{%
  language     = {[LaTeX]TeX},
  basicstyle=\ttfamily\small,
  aboveskip    = {0\p@ \@plus 6\p@},
  belowskip    = {0\p@ \@plus 6\p@},
  tabsize      = 2,
  breaklines   = true,
  breakatwhitespace = true,
  keepspaces   = true,
  escapeinside = {(*@}{@*)},
  moredelim    = {[is][\ttfamily\bfseries\color{highlight1}]{|}{|}},
  moredelim    = {[is][\ttfamily\bfseries\color{highlight1}]{|1}{1|}},
  moredelim    = {[is][\ttfamily\bfseries\color{highlight2}]{|2}{2|}},
}


\lstdefinelanguage{extBibTeX}{%
  keywords={%
    @article,@artwork,@audio,@bibnote,@book,@bookinbook,@booklet,%
    @collection,@commentary,@customa,@customb,@customc,@customd,%
    @custome,@customf,@inbook,@incollection,@inproceedings,%
    @inreference,@image,@jurisdiction,@legal,@legislation,@letter,%
    @manual,@misc,@movie,@music,@mvcollection,@mvreference,%
    @mvproceedings,@mvbook,@online,@patent,@performance,@periodical,%
    @proceedings,@reference,@report,@review,@set,@software,@standard,%
    @suppbook,@suppcollection,@suppperiodical,@thesis,@unpublished,@video%
   },
   morekeywords={author,title,date},
   sensitive=false,
}

\newcommand*{\highlight}[2][1]{\textcolor{highlight#1}{#2}}

\newtcolorbox{bibexample}[1][]{colframe=spot!75!black, colback=spot!5!white, #1}
\newtcblisting{biblatexcode}{colframe=black!75!white, colback=black!5!white,
                             listing only,
                             listing options={style=extblxstyle}}
\newtcblisting{bibbibtexcode}{colframe=black!75!white, colback=black!5!white,
                             breakable, listing only,
                             listing options={style=extblxstyle,
                                              language=extBibTeX}}

\makeatletter
% this is taken from ltxdockit.cls, which is not loaded
\newrobustcmd*{\fnurl}[1][]{\hyper@normalise\ltd@fnurl{#1}}
\def\ltd@fnurl#1#2{\footnote{#1\hyper@linkurl{\Hurl{#2}}{#2}}}

\newrobustcmd*{\email}{\hyper@normalise\ltd@email}
\def\ltd@email#1{\href{mailto:#1}{#1}}

% title
\providecommand*{\titlepage}{}
\providecommand*{\titlefont}{}
\renewrobustcmd*{\titlepage}[1]{\setkeys{ltd@ttp}{#1}}
\renewcommand*{\titlefont}{\bfseries}
\define@key{ltd@ttp}{title}{\def\ltd@title@title{#1}}
\define@key{ltd@ttp}{subtitle}{\def\ltd@title@subtitle{#1}}
\define@key{ltd@ttp}{url}{\def\ltd@title@url{#1}}
\define@key{ltd@ttp}{author}{\def\ltd@title@author{#1}}
\define@key{ltd@ttp}{email}{\def\ltd@title@email{#1}}
\define@key{ltd@ttp}{revision}{\def\ltd@title@revision{#1}}
\define@key{ltd@ttp}{date}{\def\ltd@title@date{#1}}

\providecommand*{\printtitlepage}{}
\renewrobustcmd*{\printtitlepage}{%
  \begingroup
  \centering\titlefont
  \begingroup\LARGE
    \ifundef\ltd@title@url
      {\ltd@title@title}
      {\href{\ltd@title@url}{\ltd@title@title}}%
    \par
  \endgroup
  \vspace{0.25\baselineskip}
  \begingroup\large
    \ltd@title@subtitle\par
  \endgroup
  \vspace{0.25\baselineskip}
  \begin{multicols}{2}
  \raggedleft
    \ltd@title@author\par
    \expandafter\email\expandafter{\ltd@title@email}\par
  \raggedright
    Version \ltd@title@revision\par\ltd@title@date
  \end{multicols}
  \endgroup}

\BeforeStartingTOC[toc]{\begin{multicols}{2}}
\AfterStartingTOC[toc]{\end{multicols}}

\newrobustcmd*{\tex}{TeX}
\newrobustcmd*{\etex}{\mbox{e-TeX}}
\newrobustcmd*{\pdftex}{pdf\-\tex}
\newrobustcmd*{\xetex}{Xe\-\tex}
\newrobustcmd*{\luatex}{Lua\-\tex}
\newrobustcmd*{\latex}{La\kern-0.07em TeX}
\newrobustcmd*{\pdflatex}{pdf\-\latex}
\newrobustcmd*{\xelatex}{Xe\-\latex}
\newrobustcmd*{\lualatex}{Lua\-\latex}
\newrobustcmd*{\bibtex}{Bib\kern-0.07em TeX}
\newrobustcmd*{\lppl}{\latex{} Project Public License}
\newrobustcmd*{\pdf}{\acr{PDF}}
\newrobustcmd*{\utf}{\mbox{\acr{UTF}-8}}

\pdfstringdefDisableCommands{%
  \def\tex{TeX}%
  \def\etex{e-TeX}%
  \def\xetex{XeTeX}%
  \def\latex{LaTeX}%
  \def\xelatex{XeLaTeX}%
  \def\bibtex{BibTeX}%
  \def\lppl{LaTeX Project Public License}%
  \def\pdf{PDF}%
  \def\utf{UTF-8}%
}

\let\accentcolour\spotcolor

\addtokomafont{section}{\accentcolour}
\addtokomafont{subsection}{\accentcolour}
\addtokomafont{subsubsection}{\accentcolour}

\renewcommand*{\verbatimfont}{\ttfamily}
\renewcommand*{\displayverbfont}{\ttfamily}
\renewcommand*{\marglistfont}{\accentcolour\sffamily\small}
\renewcommand*{\margnotefont}{\sffamily\small}
\renewcommand*{\optionlistfont}{\accentcolour\sffamily\displayverbfont}
\renewcommand*{\ltxsyntaxfont}{\ttfamily}
\renewcommand*{\ltxsyntaxlabelfont}{\accentcolour\displayverbfont}
\renewcommand*{\changelogfont}{\normalfont}
\renewcommand*{\changeloglabelfont}{\accentcolour\sffamily\bfseries}
\newcommand*{\stylelistfont}{\accentcolour\sffamily\small}

\newenvironment*{stylelist}
  {\list{}{%
     \setlength{\labelwidth}{\marglistwidth}%
     \setlength{\labelsep}{\marglistsep}%
     \setlength{\leftmargin}{0pt}%
     \renewcommand*{\makelabel}[1]{\hss\stylelistfont##1}}%
   \def\styleitem##1{%
     \item[{##1}]%
     \ltd@pdfbookmark{##1}{##1}}}
  {\endlist}

\newenvironment*{keymarglistbook}
  {\marglist
   \setlength{\itemsep}{0pt}%
   \raggedright
   \let\@@item\item
   \def\item[##1]{%
     \@@item[{##1}]%
     \ltd@pdfbookmark{##1}{##1}}}
  {\endmarglist}

% modified for correct measurements
\def\ltd@option@i#1#2#3#4[#5]{%
  \item[#2]%
  \ltd@pdfbookmark{#1}{#1}%
  \begingroup\raggedright
  \ltd@textverb{=}%
  \settowidth\@tempdimb{\ltd@textverb{=}}%
  \settowidth\@tempdimc{\optionlistfont#2}%
  \ifdim\@tempdimc>\marglistwidth
    \@tempdimc=\dimexpr\@tempdimc-\marglistwidth\relax
  \else
    \@tempdimc=0pt
  \fi
  \@tempdima=\dimexpr\linewidth-\@tempdimb-\@tempdimc\relax
  \ifblank{#4}
    {}
    {\settowidth\@tempdimb{default: #4}%
     \@tempdima=\dimexpr\@tempdima-\@tempdimb-2em\relax}%
  \parbox[t]{\@tempdima}{\raggedright #3}%
  \ifblank{#4}
    {}
    {\hfill default:~#4}%
  \ifblank{#5}
    {}
    {\marginpar{\margnotefont #5}}%
  \par\endgroup
  \nobreak\vspace{\itemsep}}

\blx@inputonce{ext-biblatex-aux.def}{auxiliary code for ext-biblatex}{}{}{}{}
\newcommand*{\biblatexversion}{\extblx@requiredbiblatexversion}

\AtEndPreamble{%
  \deflength{\marglistwidth}{(\oddsidemargin+2cm)*85/100}}
\makeatother

% no \mbox here, we may have to break things
\def\sty{\texttt}

\newcommand*{\gitbaseurl}{https://github.com/moewew/biblatex-ext}
\newcommand*{\extblxversion}{0.1}
\newcommand*{\biber}{Biber}
\newcommand*{\gitissuelink}[1]{%
  \href{\gitbaseurl/issues/#1}{issue \##1 on github}}

\titlepage{%
  title={The \sty{biblatex-ext} Bundle},
  subtitle={Extensions for the \sty{biblatex} standard styles},
  url={\gitbaseurl},
  author={Moritz Wemheuer},
  email={\null},%
  revision={\extblxversion},
  date={\DTMDate{2018-02-16}}}

\hypersetup{%
  pdftitle={The biblatex-ext Bundle},
  pdfsubject={Extensions for the \sty{biblatex} standard styles},
  pdfauthor={Moritz Wemheuer},
  pdfkeywords={latex, biblatex, bibtex, bibliography, references, citation}}

\hyphenation{%
  star-red
  un-star-red
  bib-lio-gra-phy
  white-space
}

\begin{document}

\printtitlepage
\tableofcontents


\section{Introduction}\label{sec:int}
The \sty{biblatex-ext} bundle provides an extended version of the standard
styles that come with \sty{biblatex}.
For each standard style this bundle provides a style with the same name
prefixed with \texttt{ext-} which can be used as a drop-in replacement for the
standard style -- for example, the replacement for \texttt{authoryear-icomp}
is called \texttt{ext-authoryear-icomp}.

The aim of the styles of this bundle is to offer a simple interface to change
some of the stylistic decisions made for the standard styles that would
otherwise need cumbersome and tedious redefinitions.
Additionally, some customisation features that were not deemed appropriate
for inclusion in the \sty{biblatex} kernel are provided.
Other than that the styles are as close to the standard styles as possible.
The styles do not attempt to offer options or commands for customisations that
are already fairly simple to achieve with the tools provided by the standard
styles.\footnote{Of course the author reserves the right to decide what
exactly \enquote{simple} means and also the right to deviate from this principle
from time to time.}

A few words of warning:
The styles of this bundle are only really useful if you want to use one of their
features to avoid having to go through the lengthy and tedious redefinitions the
standard styles would require.
Most methods to modify the standard styles are also applicable to the styles of
this bundle, but you may break some of their features if you happen to redefine
something that the \sty{ext-} styles redefine themselves.
While many contributed \sty{biblatex} styles are examples of good \sty{biblatex}
style coding, this can not be said of all of the files included in this bundle.
Especially the citation styles have to go through great lengths to implement the
citation delimiter feature properly.
So if you want to get inspired for your \sty{biblatex} coding, be warned that
terrible things lurk in the \sty{.cbx} files.
The standard \sty{.cbx} files will give you a much better impression of how
things should be done.


\subsection{Requirements}
The use of the styles requires a current version of the \sty{biblatex} package.
At the time of writing the latest version of \sty{biblatex} is \biblatexversion,
that version is required for the styles to work properly.
A warning will be issued if you use an older version.
You may choose to ignore that warning, but the style can not be guaranteed to
work properly in that case:
you might get other warnings, errors, or the style might fail silently.

Use of the \biber{} backend is strongly encouraged.
No features of this package require \biber{} explicitly in their implementation,
but many \sty{biblatex} standard features require \biber{} to function properly.

\subsection{License}
Permission is granted to copy, distribute and\slash or modify this software
under the terms of the \lppl, version 1.3c%
\fnurl{https://www.latex-project.org/lppl/lppl-1-3c.txt}
or (at your option) any later version%
\fnurl{https://www.latex-project.org/lppl.txt}.
This bundle has the maintenance status \enquote{maintained}.
The current maintainer of this work is Moritz Wemheuer
(\textcopyright 2017--2018).


\subsection{Feedback}\label{subsec:int:feb}

Please use the \sty{biblatex-ext} project page on GitHub to report bugs and
submit feature requests.\footnote{\url{\gitbaseurl}}

If you do not want to report a bug or request a feature, but are simply in need
of assistance, you might want to consider posting your question on the
\texttt{comp.text.tex} newsgroup or \tex{} -- \latex{} Stack Exchange.%
\fnurl{https://tex.stackexchange.com/questions/tagged/biblatex}

\section{Usage}

The \sty{biblatex-ext} bundle is a collection of \sty{biblatex} style files.
You can load the styles exactly as you would load the standard styles:
\begin{biblatexcode}
\usepackage[style=(*@$\langle$\normalfont\emph{style}$\rangle$@*)]{biblatex}
\end{biblatexcode}
The naming of the styles follows the scheme
\mbox{\texttt{ext-}$\langle$\emph{standard style}$\rangle$}, e.g.,
the style corresponding to \texttt{authoryear-icomp} is called
\texttt{ext-authoryear-icomp}.

\subsection{Styles}
Please refer to the \sty{biblatex} documentation%
\fnurl{http://mirrors.ctan.org/macros/latex/contrib/biblatex/doc/biblatex.pdf}
and the style examples%
\fnurl{http://mirrors.ctan.org/macros/latex/contrib/biblatex/doc/examples/}
for a more detailed description of each standard style.
The relations between the styles are exactly as in their standard counterparts.
The \texttt{ext-} styles only build on top of the standard files.
\begin{stylelist}
\styleitem{ext-numeric}
An extended version of the standard \sty{numeric} style for citations with
numeric labels.
\begin{bibexample}[]
[1, 2, 3, 5, 6, 7]
\end{bibexample}

\styleitem{ext-numeric-comp}
An extended version of the standard \sty{numeric-comp} style.
Similar to \sty{ext-numeric}, but citation labels are
compressed to give \enquote{[1--3, 5--7]} instead of
\enquote{[1, 2, 3, 5, 6, 7]}.
\begin{bibexample}[]
[1--3, 5--7]
\end{bibexample}

\styleitem{ext-numeric-verb}
An extended version of the standard \sty{numeric-verb} style.
This style is similar to the \sty{numeric} style, but each citation label
is in its own set of brackets: \enquote{[1], [2], [3]}.
\begin{bibexample}[]
[1], [2], [3], [5], [6], [7]
\end{bibexample}

\styleitem{ext-alphabetic}
An extended version of the standard \sty{alphabetic} style for citations with
alphabetic labels derived from author name and year.
\begin{bibexample}[]
[SR98, Knu84c]
\end{bibexample}

\styleitem{ext-alphabetic-verb}
An extended version of the standard \sty{alphabetic-verb} style.
This style is based on \sty{ext-alphabetic}, but like
\sty{ext-numeric-verb} places each citation label in its own pair of
brackets: \enquote{[SR98], [Knu84c]}.
\begin{bibexample}[]
[SR98], [Knu84c]
\end{bibexample}

\styleitem{ext-authoryear}
An extended version of the standard \sty{authoryear} style for citations
using author name and year.
\begin{bibexample}
Sigfridsson and Ryde 1998
\end{bibexample}

\styleitem{ext-authoryear-comp}
An extended version of the standard \sty{authoryear-comp} style.
The style is based on the author-year citations of \sty{ext-authoryear},
but several works by the same author are compressed by not displaying the
author multiple times: \enquote{Knuth 1984, 1986} instead of
\enquote{Knuth 1984; Knuth 1986}.
\begin{bibexample}
Knuth 1984, 1986
\end{bibexample}

\styleitem{ext-authoryear-ibid}
An extended version of the standard \sty{authoryear-ibid} style.
This style is similar to \sty{ext-authoryear}, but repeated citations are
replaced with \enquote{ibidem}.
\begin{bibexample}
Knuth 1984\quad ibid.
\end{bibexample}

\styleitem{ext-authoryear-icomp}
An extended version of the standard \sty{authoryear-icomp} style.
This style combines the two styles \sty{ext-authoryear-comp} and
\sty{ext-authoryear-ibid}.
\begin{bibexample}
Knuth 1984, 1986 \quad Sigfridsson and Ryde 1998 \quad ibid.
\end{bibexample}

\styleitem{ext-authortitle}
An extended version of the standard \sty{authortitle} style for citations
by author and title.
\begin{bibexample}
Maron, \emph{Animal Triste}
\end{bibexample}

\styleitem{ext-authortitle-comp}
An extended version of the standard \sty{authortitle-comp} style.
This style is based on \sty{ext-authortitle} and compresses several citations
by the same author just like \sty{ext-author\-year-comp}:
\enquote{Aristotle, \emph{Physics}, \emph{Poetics}} instead of
\enquote{Aristotle, \emph{Physics}; Aristotle, \emph{Poetics}}.
\begin{bibexample}
Aristotle, \emph{Physics}, \emph{Poetics}
\end{bibexample}

\styleitem{ext-authortitle-ibid}
An extended version of the standard \sty{authortitle-ibid} style.
This style is similar to \sty{ext-authortitle}, but replaces repeated
citations of the same work with \enquote{ibidem}.
\begin{bibexample}
Maron, \emph{Animal Triste}\quad ibid.
\end{bibexample}

\styleitem{ext-authortitle-icomp}
An extended version of the standard \sty{authortitle-icomp} style.
This style combines \sty{ext-authortitle-comp} and
\sty{ext-authortitle-ibid}.
\begin{bibexample}
Aristotle, \emph{Physics}, \emph{Poetics}\quad Maron, \emph{Animal Triste}\quad ibid.
\end{bibexample}

\styleitem{ext-authortitle-terse}
An extended version of the standard \sty{authortitle-terse} style.
This style is similar to \sty{ext-authortitle}, but the title is omitted in
citations if there is only one work by the relevant author.
\begin{bibexample}
Sigfridsson and Ryde \quad Aristotle, \emph{Physics}; Aristotle, \emph{Poetics}
\end{bibexample}

\styleitem{ext-authortitle-tcomp}
An extended version of the standard \sty{authortitle-tcomp} style.
This style combines \sty{ext-authortitle-terse} and
\sty{ext-authortitle-comp}.
\begin{bibexample}
Sigfridsson and Ryde \quad Aristotle, \emph{Physics}, \emph{Poetics}
\end{bibexample}

\styleitem{ext-authortitle-ticomp}
An extended version of the standard \sty{authortitle-ticomp} style.
This style combines \sty{ext-authortitle-terse},
\sty{ext-authortitle-comp} and \sty{ext-authortitle-ibid}.
\begin{bibexample}
Aristotle, \emph{Physics}, \emph{Poetics} \quad Sigfridsson and Ryde \quad ibid.
\end{bibexample}

\styleitem{ext-verbose}
An extended version of the standard \sty{verbose} style.
This style shows the full bibliographic reference the first time a work is
cited.

\styleitem{ext-verbose-ibid}
An extended version of the standard \sty{verbose-ibid} style.
Based on \sty{ext-verbose}, repeated citations to the same work are replaced
with \enquote{ibidem}.

\styleitem{ext-verbose-note}
An extended version of the standard \sty{verbose-note} style.
Based on \sty{ext-verbose} and intended for use in footnotes, subsequent
citations link back to the footnote the entry was cited at first and in full.

\styleitem{ext-verbose-inote}
An extended version of the standard \sty{verbose-inote} style.
Similar to \sty{verbose-note}, but repeated citations to the same work are
replaced with \enquote{ibidem}.

\styleitem{ext-verbose-trad1}
An extended version of the standard \sty{verbose-trad1} style.
This style makes extensive use of scholarly abbreviations and is otherwise
similar to \sty{ext-verbose}.

\styleitem{ext-verbose-trad2}
An extended version of the standard \sty{verbose-trad2} style.
The style is similar to \sty{ext\hyp{}verbose\hyp{}inote} and
uses scholarly abbreviations to shorten citations.

\styleitem{ext-verbose-trad3}
An extended version of the standard \sty{verbose-trad3} style.
This style is similar to \sty{ext\hyp{}verbose\hyp{}trad2}.
\end{stylelist}


\subsection{Options}
All options of the \sty{biblatex} package are supported and each style supports
the options of its standard counterpart.

Additionally all styles support the following options in global, per-type and
per-entry scope.
The default values are such that the styles can be used as drop-in replacement
for the standard files without significant changes in output.

\begin{optionlist}
\boolitem[true]{articlein}
Whether or not to display \enquote{in:} before the journal information in
\bibtype{article} entries.
All other entry types are not affected by this option.

\begin{bibexample}[title={\kvopt{articlein}{true}}]
\makeatletter
\togglefalse{bbx:doi}
\renewbibmacro*{in:}{%
  \blx@begunit\blx@endunit
  \textcolor{highlight1}{%
    \printtext{%
      \bibstring{in}\intitlepunct}}}
\toggletrue{bbx:articlein}\fullcite{sigfridsson}
\renewbibmacro*{in:}{%
  \blx@begunit\blx@endunit
  \textcolor{highlight2}{%
    \printtext{%
      \bibstring{in}\intitlepunct}}}
\smallskip\par\fullcite{brandt}
\makeatother
\end{bibexample}

\begin{bibexample}[title={\kvopt{articlein}{false}}]
\togglefalse{bbx:doi}
\togglefalse{bbx:articlein}\fullcite{sigfridsson}
\makeatletter
\renewbibmacro*{in:}{%
  \blx@begunit\blx@endunit
  \textcolor{highlight2}{%
    \printtext{%
      \bibstring{in}\intitlepunct}}}
\smallskip\par\fullcite{brandt}
\makeatother
\end{bibexample}

\boolitem[false]{innamebeforetitle}
Whether or not the \bibfield{editor} should be moved before the
\bibfield{booktitle} field for \bibtype{inbook}, \bibtype{incollection} and
\bibtype{inproceedings} entries.

\begin{bibexample}[title={\kvopt{innamebeforetitle}{true}}]
\makeatletter
\renewbibmacro*{bbx:in:editor}[1]{%
  \ifboolexpr{
    test \ifuseeditor
    and
    not test {\ifnameundef{editor}}
  }
    {\textcolor{highlight1}{%
       \ifboolexpr{togl {bbx:innameidem} and test {\bbx@ineditoridem}}
         {\bibstring[\mkibid]{idem\thefield{gender}}}
         {\printnames[ineditor]{editor}}%
       \setunit{\printdelim{editortypedelim}}%
       \usebibmacro{#1}}%
     \clearname{editor}}
    {}}
\makeatother
\toggletrue{bbx:innamebeforetitle}\fullcite{pines}
\end{bibexample}

\begin{bibexample}[title={\kvopt{innamebeforetitle}{false}}]
\renewbibmacro*{byeditor+others}{%
  \ifnameundef{editor}
    {}
    {\textcolor{highlight1}{%
       \usebibmacro{byeditor+othersstrg}%
       \setunit{\addspace}%
       \printnames[byeditor]{editor}%
       \newunit}
     \clearname{editor}}%
  \usebibmacro{byeditorx}%
  \usebibmacro{bytranslator+others}}
\togglefalse{bbx:innamebeforetitle}\fullcite{pines}
\end{bibexample}

\boolitem[false]{innameidem}
Whether or not the \bibfield{editor} of the \bibfield{booktitle}
for the entry types \bibtype{inbook}, \bibtype{incollection} and
\bibtype{inproceedings} is replaced by \enquote{idem} in case the
\bibfield{editor} and \bibfield{author} name lists coincide.
This option only has an effect if \opt{innamebeforetitle} is set to \opt{true}.

\begin{bibexample}[title={\kvopt{innameidem}{true}}]
\togglefalse{bbx:isbn}
\makeatletter
\renewbibmacro*{bbx:in:editor}[1]{%
  \ifboolexpr{
    test \ifuseeditor
    and
    not test {\ifnameundef{editor}}
  }
    {\textcolor{highlight1}{%
       \ifboolexpr{togl {bbx:innameidem} and test {\bbx@ineditoridem}}
         {\bibstring[\mkibid]{idem\thefield{gender}}}
         {\printnames[ineditor]{editor}}%
       \setunit{\printdelim{editortypedelim}}%
       \usebibmacro{#1}%
       \clearname{editor}}}
    {}}
\makeatother
\toggletrue{bbx:innamebeforetitle}\toggletrue{bbx:innameidem}
\fullcite{gaonkar:in}
\end{bibexample}

\begin{bibexample}[title={\kvopt{innameidem}{false}}]
\togglefalse{bbx:isbn}
\makeatletter
\renewbibmacro*{bbx:in:editor}[1]{%
  \ifboolexpr{
    test \ifuseeditor
    and
    not test {\ifnameundef{editor}}
  }
    {\textcolor{highlight1}{%
       \ifboolexpr{togl {bbx:innameidem} and test {\bbx@ineditoridem}}
         {\bibstring[\mkibid]{idem\thefield{gender}}}
         {\printnames[ineditor]{editor}}%
       \setunit{\printdelim{editortypedelim}}%
       \usebibmacro{#1}%
       \clearname{editor}}}
    {}}
\makeatother
\toggletrue{bbx:innamebeforetitle}\togglefalse{bbx:innameidem}
\fullcite{gaonkar:in}
\end{bibexample}

\boolitem[false]{inxref}
This option controls if \bibtype{inbook}, \bibtype{incollection} and
\bibtype{inproceedings} entries that are tied to a parent entry with
\bibfield{xref} or \bibfield{crossref} should cite their parent in the
bibliography if the parent is listed in the bibliography as a separate entry.
With the default setting \kvopt{inxref}{false} the parent is not cited, the
entry is shown as in the standard styles. If the option is set to
\texttt{true}, the block following the \enquote{in:} is replaced by a citation
to the parent entry.
The option itself does \emph{not} cause the parent entry to be added to the
bibliography, this needs to happen either explicitly by citing the parent
(this includes \cmd{nocite}), or implicitly via the \opt{mincrossrefs} or
\opt{minxrefs} functionality.

\nocite{westfahl:frontier}
\begin{bibexample}[title={\kvopt{inxref}{true}}]
\makeatletter
\renewbibmacro*{crosscite}[1]{%
  \textcolor{highlight1}{%
    \iftoggle{bbx:inxref}
      {\iffieldundef{crossref}
         {\iffieldundef{xref}
            {\usebibmacro{#1}}
            {\printtext{\bbx@cite@xref{\thefield{xref}}}}}
         {\printtext{\bbx@cite@xref{\thefield{crossref}}}}}
      {\usebibmacro{#1}}}}
\makeatother
\toggletrue{bbx:inxref}\fullcite{westfahl:space}
\end{bibexample}

\begin{bibexample}[title={\kvopt{inxref}{false}}]
\makeatletter
\renewbibmacro*{crosscite}[1]{%
  \textcolor{highlight1}{%
    \iftoggle{bbx:inxref}
      {\iffieldundef{crossref}
         {\iffieldundef{xref}
            {\usebibmacro{#1}}
            {\printtext{\bbx@cite@xref{\thefield{xref}}}}}
         {\printtext{\bbx@cite@xref{\thefield{crossref}}}}}
      {\usebibmacro{#1}}}}
\makeatother
\togglefalse{bbx:inxref}\fullcite{westfahl:space}
\end{bibexample}

The \opt{dashed} option of the \sty{authoryear}- and \sty{authortitle}-like
bibliography styles allows for finer control over the dashes now.
\optitem[false]{dashed}{\opt{true}, \opt{false}, \opt{fullhash},
                        \opt{bibnamehash}}

This option controls whether or not recurring lists of authors/editors in the
bibliography are replaced with a dash.
The standard values \opt{true} and \opt{false} are still valid and give the
exact same output as in the standard styles.
The new values \opt{fullhash} and \opt{bibnamehash} differ in how exactly they
determine if a list of authors/editors is the same.
\begin{valuelist}
\item[true] An alias for \opt{fullhash}.
\item[false] Disable this feature.
\item[bibnamehash] Compare name lists using \bibfield{bibnamehash}, taking
                   into account only names that are actually listed in the
                   bibliography account.
\item[fullhash] Compare name lists using \bibfield{fullhash}, taking into
                account all names in the list, even those that are truncated
                and do not appear in the bibliography.
\end{valuelist}


\begin{refsection}
\makeatletter
Assuming \kvopt{maxnames}{1} and no name list disambiguation
(\kvopt{uniquelist}{false}), the four entries
\begin{bibbibtexcode}
@book{applebyone,
  author  = {Humphrey Appleby},
  title   = {On the Ablative in Greek},
  date    = {1982},
}

@book{applebytwo,
  author  = {Humphrey Appleby},
  title   = {On the Importance of the Civil Service},
  date    = {1980},
}

@book{elkone,
  author  = {Elk, Anne and Hacker, James and Rumpo, Syd},
  title   = {On the Theory of Brontosauruses},
  date    = {1972},
}

@book{elktwo,
  author  = {Elk, Anne and Hacker, James and Pode, Eric},
  title   = {On the Theory of Stegosauruses},
  date    = {1972},
}
\end{bibbibtexcode}
give
\let\ExecuteBibliographyOptions\@gobble
\nocite{elkone,elktwo,applebyone,applebytwo}
\begin{bibexample}[title={\kvopt{dashed}{false}}]
\csuse{extblx@opt@dashed@false}
\printbibliography[heading=none]
\end{bibexample}

\begin{bibexample}[title={\kvopt{dashed}{bibnamehash}}]
\csuse{extblx@opt@dashed@bibnamehash}
\printbibliography[heading=none]
\end{bibexample}

\begin{bibexample}[title={\kvopt{dashed}{fullhash}}]
\csuse{extblx@opt@dashed@fullhash}
\printbibliography[heading=none]
\end{bibexample}
\makeatother
\end{refsection}
With \kvopt{dashed}{bibnamehash} the dash replaces the name list if they are
indistinguishable in the bibliography, with \kvopt{dashed}{fullhash} the lists
are only replaced if they are exactly the same in the data source.
\end{optionlist}

\subsection{Field Formats}
At a few points where the standard styles employ hard-coded formatting directive
the styles of this bundle offer customisable formats instead.

\begin{keymarglistbook}
\item[bib(list)labeldate] The format for the labeldate in the bibliography or
  a bibliography list for author-year-like styles. The default is to wrap the
  date in round brackets.
  \begin{bibexample}
  \togglefalse{bbx:doi}
  \DeclareFieldFormat{biblabeldate}{\highlight{\bibopenparen}#1\highlight{\bibcloseparen}}
  \fullcite{sigfridsson}
  \end{bibexample}

\item[issuedate] The format of the \bibfield{issue} and \bibfield{date}
  information for \bibtype{article}s. By default this block is wrapped in
  round brackets.
  \begin{bibexample}
  \makeatletter\bbx@opt@mergedate@false\makeatother
  \togglefalse{bbx:doi}
  \DeclareFieldFormat{issuedate}{\highlight{\bibopenparen}#1\highlight{\bibcloseparen}}
  \fullcite{sigfridsson}
  \end{bibexample}
\end{keymarglistbook}

\DeclareFieldFormat{issuedate}{\mkbibparens{#1}}

\subsection{Punctuation and Delimiters}
The package provides the following commands to modify the delimiters and
punctuation between fields.
\begin{ltxsyntax}
\csitem{volnumdatedelim}
The delimiter between the \bibfield{volume}, \bibfield{number} block and the
date information for \bibtype{article}.
The default is a space.
\begin{bibexample}
\togglefalse{bbx:doi}
\renewcommand*{\volnumdatedelim}{\highlight{\textvisiblespace}}
\makeatletter\bbx@opt@mergedate@false\makeatother
\fullcite{sigfridsson}
\end{bibexample}


\csitem{volnumdelim}
The delimiter between \bibfield{volume} and \bibfield{number} for
\bibtype{article}.
The default is a dot.
\begin{bibexample}
\togglefalse{bbx:doi}
\renewcommand*{\volnumdelim}{\highlight{\textbf{\adddot}}}
\fullcite{sigfridsson}
\end{bibexample}

\csitem{sernumdelim}
The delimiter between \bibfield{series} and \bibfield{number}.
The default is a space.
\begin{bibexample}
\renewcommand*{\sernumdelim}{\highlight{\textvisiblespace}}
\fullcite{coleridge}
\end{bibexample}

\csitem{locdatedelim}
The delimiter between \bibfield{location} and \bibfield{date}.
The default is a comma followed by a space.
\begin{bibexample}
\renewcommand*{\locdatedelim}{\highlight{\bfseries\addcomma\textvisiblespace}}
\makeatletter\bbx@opt@mergedate@false\makeatother
\fullcite{jaffe}
\end{bibexample}

\csitem{locpubdelim}
The delimiter between \bibfield{location} and \bibfield{publisher}\slash%
\bibfield{organization}\slash\bibfield{institution}.
The default is a colon followed by a space.
\begin{bibexample}
\makeatletter\bbx@opt@mergedate@false\makeatother
\renewcommand*{\locpubdelim}{\highlight{\bfseries\addcolon\textvisiblespace}}
\fullcite{knuth:ct:a}
\end{bibexample}


\csitem{publocdelim}
The delimiter between \bibfield{publisher}\slash\bibfield{organization}\slash
\bibfield{institution} and \bibfield{location}.
The default is a comma followed by a space.
This delimiter is not used by the default style, since the standard order of
fields is \bibfield{location}, \bibfield{publisher}\slash
\bibfield{organization}\slash\bibfield{institution}, \bibfield{date}.
You could use \cs{publocdelim} if you changed the order of these fields to
\bibfield{publisher}\slash\bibfield{organization}\slash\bibfield{institution},
\bibfield{location} \bibfield{date} with
\begin{biblatexcode}
\renewbibmacro*{pubinstorg+location+date}[1]{%
  \printlist{#1}%
  \setunit*{|1\publocdelim1|}%
  \printlist{location}%
  \setunit*{|2\locdatedelim2|}%
  \usebibmacro{date}%
  \newunit}
\end{biblatexcode}
\begin{bibexample}
\renewcommand*{\publocdelim}{\highlight[1]{\bfseries\addcomma\textvisiblespace}}
\renewcommand*{\locdatedelim}{\highlight[2]{\bfseries\addcomma\textvisiblespace}}
\renewbibmacro*{pubinstorg+location+date}[1]{%
  \printlist{#1}%
  \setunit*{\publocdelim}%
  \printlist{location}%
  \setunit*{\locdatedelim}%
  \usebibmacro{date}%
  \newunit}
\makeatletter\bbx@opt@mergedate@false\makeatother
\fullcite{knuth:ct:a}
\end{bibexample}


\csitem{pubdatedelim}
The delimiter between \bibfield{publisher}\slash\bibfield{organization}\slash
\bibfield{institution} and \bibfield{date}.
The default is a comma followed by a space.
\begin{bibexample}
\renewcommand*{\pubdatedelim}{\highlight{\bfseries\addcomma\textvisiblespace}}
\makeatletter\bbx@opt@mergedate@false\makeatother
\fullcite{knuth:ct:a}
\end{bibexample}

\csitem{extradateonlycompcitedelim}
Similar to \cmd{compcitedelim}, but indended for use between compressed
citations where the second is an \bibfield{extradate} only.
The default is a comma (\emph{not} followed by a space).
\end{ltxsyntax}
\begin{bibexample}
\renewcommand*{\extradateonlycompcitedelim}{\highlight{\bfseries\addcomma}}
\cite{knuth:ct:b,knuth:ct:c}
\end{bibexample}


\subsection{Delimiters for Citation Commands}
The citation commands \cmd{cite}, \cmd{parencite} and \cmd{textcite} come with
two sets of delimiters: A pair of \emph{outer delimiters} wrapped around the
resulting citation in its entirety and a pair of \emph{inner delimiters} that
sets off certains bits of the citation label from other information.
An example for outer delimiters would be the round brackets of \cmd{parencite}
for \texttt{author\-year}-like styles or the square brackets of \cmd{cite} for
\sty{numeric}- or \sty{alphabetic}-like styles.
Inner delimiters would be the round brackets in \cmd{textcite} around
the year for \sty{authoryear} or around the title for \sty{authortitle}.

\begin{ltxsyntax}
\cmditem{DeclarePairedCiteOuterDelim}{cite command}{opening delimiter}
        {closing delimiter}

Sets up the outer delimiters for the citation command
\cmd{$\langle$\emph{cite command}$\rangle$}.

You may use almost arbitrary input for \prm{opening delimiter} and
\prm{closing delimiter} as long as typesetting of
\enquote{\prm{opening delimiter}text\prm{closing delimiter}} does not lead to
errors.
Instead of hard-coded \texttt{(}, \texttt{)}, \texttt{[} and \texttt{]} their
\sty{biblatex} counterparts \cmd{bibopenparen}, \cmd{bibcloseparen},
\cmd{bibopenbracket} and \cmd{bibclosebracket} are preferable, since these
commands can change the style when nesting and check if opening brackets
are always closed.\footnote{That means that you can not have \cmd{bibopenparen}
as opening delimiter and \cmd{bibclosebracket} as closing delimiter, though:
The closing delimiter will automatically match the opening one.}

For most styles \cmd{parencite} would be set up with
\begin{biblatexcode}
\DeclarePairedCiteOuterDelim{parencite}{\bibopenparen}{\bibcloseparen}
\end{biblatexcode}

\cmditem{DeclarePairedCiteOuterDelimAlias}{cite alias}{cite command}
\cmditem*{DeclarePairedCiteOuterDelimAlias*}{cite alias}{cite command}

Use the outer delimiters of \cmd{$\langle$\emph{cite command}$\rangle$} for
\cmd{$\langle$\emph{cite alias}$\rangle$} as well.
The unstarred version uses \cmd{def} assignment while the starred version uses
\cmd{let}. This means that the starred version copies the values of the
definitions at the time of executing the aliasing command,
whereas the alias created by the unstarred version will only evaluate the
delimiters whenever the citation command is called.


\cmditem{DeclarePairedCiteInnerDelim}{cite command}{opening delimiter}
        {closing delimiter}

Sets up the inner delimiters for the citation command
\cmd{$\langle$\emph{cite command}$\rangle$}.

For most styles \cmd{textcite} would be set up with
\begin{biblatexcode}
\DeclarePairedCiteInnerDelim{textcite}{\bibopenparen}{\bibcloseparen}
\end{biblatexcode}

\cmditem{DeclarePairedCiteInnerDelimAlias}{cite alias}{cite command}
\cmditem*{DeclarePairedCiteInnerDelimAlias*}{cite alias}{cite command}

Use the inner delimiters of \cmd{$\langle$\emph{cite command}$\rangle$} for
\cmd{$\langle$\emph{cite alias}$\rangle$} as well.
The unstarred version uses \cmd{def} assignment while the starred version uses
\cmd{let}. This means that the starred version copies the values of the
definitions at the time of executing the aliasing command,
whereas the alias created by the unstarred version will only evaluate the
delimiters whenever the citation command is called.

If you wanted \cmd{parencite} of \sty{authoryear} to look like
\enquote{[Sigfridsson and Worman $\langle$1998$\rangle$]} you would use
\begin{biblatexcode}
\DeclarePairedCiteOuterDelim{parencite}{|1\bibopenbracket1|}{|1\bibclosebracket1|}
\DeclarePairedCiteInnerDelim{parencite}{|2$\langle$2|}{|2$\rangle$2|}
\end{biblatexcode}
\begin{bibexample}
\DeclarePairedCiteOuterDelim{parencite}{\highlight[1]{\bibopenbracket}}{\highlight[1]{\bibclosebracket}}
\DeclarePairedCiteInnerDelim{parencite}{\highlight[2]{${\langle}$}}{\highlight[2]{${\rangle}$}}
\parencite{sigfridsson}\quad\parencite{worman,geer}\quad
\parencite{knuth:ct:a,knuth:ct:b,knuth:ct:c}
\end{bibexample}
\end{ltxsyntax}

\subsection{Selected Bibliography Macros}
\begin{keymarglistbook}
\item[edition]
A bibliography macro to print the \bibfield{edition} field, this avoids a direct
\lstinline|\printfield{edition}| in the bibliography drivers.

\item[in:editor(+others)]
The bibliography macro to print the \bibfield{editor} before the
\bibfield{booktitle} for \bibtype{inbook}, \bibtype{incollection} and
\bibtype{inproceedings} when \opt{innamebeforetitle} is set to \opt{true}.
The \cmd{printname} uses the name format \texttt{ineditor}.

\item[pubinstorg+location+date]
A general-purpose bibliography macro to catch \texttt{publisher+location+date},
\texttt{institution+location+date} and \texttt{organization+location+date}.
As parameter it takes the name of a list field, sensible values are
\bibfield{publisher}, \bibfield{institution} and \bibfield{organization}.

\begin{biblatexcode}
\newbibmacro*{pubinstorg+location+date}[1]{%
  \printlist{location}%
  \iflistundef{|#1|}
    {\setunit*{\locdatedelim}}
    {\setunit*{\locpubdelim}}%
  \printlist{|#1|}%
  \setunit*{\pubdatedelim}%
  \usebibmacro{date}%
  \newunit}

\renewbibmacro*{|publisher|+location+date}{%
  \usebibmacro{pubinstorg+location+date}{|publisher|}}

\renewbibmacro*{|institution|+location+date}{%
  \usebibmacro{pubinstorg+location+date}{|institution|}}

\renewbibmacro*{|organization|+location+date}{%
  \usebibmacro{pubinstorg+location+date}{|organization|}}
\end{biblatexcode}
\end{keymarglistbook}

\section{Revision History}\label{apx:log}
\begin{changelog}
\begin{release}{<version>}{<date>}
\item First release.
\end{release}
\end{changelog}

\begin{bibexample}[breakable]
\printbibliography
\end{bibexample}

\end{document}
\endinput
